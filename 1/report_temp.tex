\documentclass[a4paper,11pt]{ltjsarticle}
\usepackage{geometry}
\geometry{top=25mm,bottom=25mm,left=25mm,right=25mm}
\usepackage{graphicx}
\usepackage{amsmath,amssymb,bm}
\usepackage{hyperref}
\usepackage{float}

\title{数理工学実験レポート\\\vspace{5pt}\large 第X章(章タイトル)}
\author{京都大学 工学部情報学科 数理工学コース\\学籍番号 1029366161\quad 中塚 一瑳}
\date{\today}

\begin{document}

% 表紙(1ページ)
\begin{titlepage}
\centering
{\Large 数理工学実験レポート\par}
\vspace{4mm}
{\large 第X章(章タイトル)\par}
\vspace{15mm}
{\large 京都大学 工学部情報学科 数理工学コース\par}
{\large 学年:2回生\par}
{\large 学籍番号:1029366161\quad 中塚 一瑳\par}
\vspace{10mm}
\begin{tabular}{@{}ll}
科目名: & 数理工学実験 \\
実験テーマ: & (記入) \\
実験の実施年月日: & (記入) \\
実験の実施場所: & (記入) \\
レポート提出年月日: & \today \\
\end{tabular}
\vfill
\end{titlepage}

\section*{要旨(Abstract)}
本実験の目的,扱う手法,主要な結果と結論を簡潔に述べる。

\section{はじめに}
本実験の目的と全体像を述べる。第2章では連立一次方程式と固有値問題に対する数値解法を扱う。

% ===== 課題ごとの構成 =====
\section{課題1:消去法}
\subsection{原理・方法}
ガウス消去法(主成分選択の有無を明記)。記号は初出で定義する。
\subsection{結果}
\begin{figure}[H]
  \centering
  % \includegraphics[width=0.62\linewidth]{fig_t1.pdf}
  \caption{課題1の結果。横軸:(単位),縦軸:(単位)。}
  \label{fig:t1}
\end{figure}
\subsection{考察}
精度・計算量・安定性で評価。図\ref{fig:t1}参照。

\section{課題2:LU分解}
\subsection{原理・方法}
$A=LU$ 分解。前進・後退代入。ピボット戦略の有無を明記。
\subsection{結果}
\begin{figure}[H]
  \centering
  % \includegraphics[width=0.62\linewidth]{fig_t2.pdf}
  \caption{課題2の結果。横軸:(単位),縦軸:(単位)。}
  \label{fig:t2}
\end{figure}
\subsection{考察}
課題1と比較(速度・誤差)。条件数の影響。

\section{課題3:LU分解による解法}
\subsection{原理・方法}
同じ右辺に対する複数回解法の効率。分解再利用。
\subsection{結果}
\begin{figure}[H]
  \centering
  % \includegraphics[width=0.62\linewidth]{fig_t3.pdf}
  \caption{課題3の結果。横軸:(単位),縦軸:(単位)。}
  \label{fig:t3}
\end{figure}
\subsection{考察}
ガウス法との差。安定性と時間削減。

\section{課題4:べき乗法}
\subsection{原理・方法}
最大固有値・固有ベクトルの反復。正規化と収束判定。
\subsection{結果}
\begin{figure}[H]
  \centering
  % \includegraphics[width=0.62\linewidth]{fig_t4.pdf}
  \caption{課題4の結果。横軸:反復回数(回),縦軸:誤差(–)。}
  \label{fig:t4}
\end{figure}
\subsection{考察}
初期ベクトル依存性と収束率。真値との差。

\section{課題5:逆反復法}
\subsection{原理・方法}
シフト$\mu$付き線形解法で固有ベクトルを精密化。正規化と収束条件。
\subsection{結果}
\begin{figure}[H]
  \centering
  % \includegraphics[width=0.62\linewidth]{fig_t5.pdf}
  \caption{課題5の結果。横軸:反復回数(回),縦軸:誤差(–)。}
  \label{fig:t5}
\end{figure}
\subsection{考察}
シフト選択と収束性。べき乗法との比較。

\section{課題6:QR分解}
\subsection{原理・方法}
ハウスホルダー/グラム–シュミットで $A=QR$。直交性検証手順。
\subsection{結果}
\begin{figure}[H]
  \centering
  % \includegraphics[width=0.62\linewidth]{fig_t6.pdf}
  \caption{課題6の結果。横軸:サイズ(次元),縦軸:直交誤差 $\lVert Q^\mathsf{T}Q-I\rVert$(–)。}
  \label{fig:t6}
\end{figure}
\subsection{考察}
古典/改良GSとハウスホルダーの安定性差。

\section{課題7:QR法}
\subsection{原理・方法}
$A_{k+1}=R_kQ_k$ の反復。シフト・ヘッセンベルク化の有無を明記。
\subsection{結果}
\begin{figure}[H]
  \centering
  % \includegraphics[width=0.62\linewidth]{fig_t7.pdf}
  \caption{課題7の結果。横軸:反復回数(回),縦軸:$\max_{i\ne j}|(A_k)_{ij}|$(–)。}
  \label{fig:t7}
\end{figure}
% \begin{figure}[H]
%   \centering
%   \includegraphics[width=0.62\linewidth]{fig_t7.pdf}
%   \caption{課題7の結果。横軸:反復回数(回),縦軸:$\max_{i\ne j}|(A_k)_{ij}|$(–)。}
%   \label{fig:t7}
% \end{figure}
\subsection{考察}
シフト導入の効果。収束速度と計算コストのトレードオフ。

\section{結論}
主要な知見を要約。今後の改良点を簡潔に述べる。

\section*{参考文献(References)}
\begin{thebibliography}{99}
\bibitem{exp2025} 数理工学実験(2025年度配布資料).
\end{thebibliography}

\section*{付録(Appendix)}
\begin{itemize}
  \item 使用したソースコード(Julia/Python)
  \item 追加ログ・出力データ
\end{itemize}

\end{document}
