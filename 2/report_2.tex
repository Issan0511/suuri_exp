\documentclass[a4paper,11pt]{ltjsarticle}
\usepackage{geometry}
\geometry{top=25mm,bottom=25mm,left=25mm,right=25mm}
\usepackage{graphicx}
\usepackage{amsmath,amssymb,bm}
\usepackage{hyperref}
\usepackage{float}
\usepackage{caption}
\usepackage{subcaption}
\usepackage{booktabs}
\usepackage{siunitx}
\usepackage{listings}
\usepackage{xcolor}
\usepackage{physics}
\usepackage[protrusion=false,expansion=false]{microtype}

\hypersetup{
  pdftitle={数理工学実験レポート},
  pdfauthor={中塚一瑳},
  colorlinks=true,
  linkcolor=blue,
  citecolor=blue,
  urlcolor=blue
}

% -------------------------
% ユーザ定義マクロ
% -------------------------
\newcommand{\StudentID}{1029366161}
\newcommand{\AuthorName}{中塚一瑳}
\newcommand{\CourseName}{数理工学実験}
\newcommand{\ChapterTitle}{第3章(最小二乗法)}
\newcommand{\ExperimentDate}{\today}

% 図・表フォルダパス(必要なら変更)
\newcommand{\graphdir}{graphs}

% 数式用の簡易マクロ
\newcommand{\R}{\mathbb{R}}
% \newcommand{\norm}[1]{\left\lVert#1\right\rVert}  % physics パッケージで既に定義済み

% listings 設定(R を例示)
\lstset{
  basicstyle=\ttfamily\footnotesize,
  breaklines=true,
  frame=single,
  numbers=left,
  numberstyle=\tiny,
  language=R,
  keywordstyle=\color{blue},
  commentstyle=\color{gray},
  stringstyle=\color{teal},
  showstringspaces=false
}



% キャプションのフォント調整
\captionsetup{font=small,labelfont=bf}

% -------------------------
% 本文
% -------------------------
\title{数理工学実験レポート\\[4pt]\large \ChapterTitle}
\author{学籍番号 1029366161 \quad 中塚一瑳}
\date{\ExperimentDate}

\begin{document}

\maketitle
\begin{abstract}
% 要旨をここに記述
\end{abstract}

\tableofcontents
\clearpage

% \section*{表記・略記}
% 本レポートにおける表記と略記を示す。必要に応じて編集すること。
% \begin{itemize}
%   \item $\R$:実数全体
%   \item $\|\cdot\|$:ユークリッドノルム(明示が必要な場合は添字を用いる)
%   \item 各種パラメータや記号は本文中で定義する
% \end{itemize}

% ===== 例:実験テンプレート =====
\section{はじめに}
本実験第2回では,最小二乗法の基礎とその実装手法を学ぶことを目的とする。具体的には,観測データからのパラメータ推定,重み付き・逐次最小二乗法,データ分割・推定値の合成,カルマンフィルタ,交互最小二乗法などの手法を理解し,それぞれのアルゴリズムを実装・評価することで,最小二乗法の応用範囲とその性能を比較検討する。


\section{最小二乗法とその評価方法}

\subsection{原理・方法}

観測モデルを
\begin{align}
 y_i = f(\theta, x_i) + w_i
\end{align}
とする.加法雑音 $w_i$ は平均 $0$,観測ごとに独立,同一分布であり,共分散は有限と仮定する.
ここでは $f(\theta, x) = \phi(x)\theta$ としてパラメータに対して線形とし,最小二乗問題
\begin{align}
 \min_{\theta}\sum_{i=1}^{N}\lVert y_i - \phi(x_i)\theta\rVert^2
\end{align}
を解く.解は
\begin{align}
 \hat{\theta}_N
  = \Big( \sum_{i=1}^{N} \phi_i^\top \phi_i \Big)^{-1}
    \sum_{i=1}^{N} \phi_i^\top y_i,
 \qquad \phi_i := \phi(x_i).
\end{align}\cite{exp2025}.

また雑音分散が未知の場合の推定量および推定誤差共分散は
\begin{align}
 \hat{\sigma}^2
 &= \frac{1}{N-n} \sum_{i=1}^{N} \lVert y_i - \phi_i \hat{\theta}_N \rVert^2, \\
 \widehat{\mathrm{Cov}}(\hat{\theta}_N)
 &= \hat{\sigma}^2 \Big( \sum_{i=1}^{N} \phi_i^\top \phi_i \Big)^{-1}
\end{align}
で与えられる.

当てはまりの評価には決定係数
\begin{align}
 C = \frac{\sum_{i=1}^{N}\lVert \phi_i \hat{\theta}_N - \bar{y} \rVert^2}
          {\sum_{i=1}^{N}\lVert y_i - \bar{y} \rVert^2},
 \qquad
 \bar{y} = \frac{1}{N}\sum_{i=1}^{N} y_i
\end{align}
を用いる.以上は資料3.2の線形最小二乗および評価に対応する.


本実験では $\phi(x)$ として設計した特徴量から行列
\begin{align}
 X = \begin{bmatrix}
  \phi(x_1)^\top \\
  \vdots \\
  \phi(x_N)^\top
 \end{bmatrix}
\end{align}
を構成し,観測データを
\begin{align}
 y = [y_1; \dots; y_N]
\end{align}
として用いる.パラメータ推定量は
\begin{align}
 \hat{\theta} = (X^\top X)^{-1} X^\top y
\end{align}
で与えられる.

残差は
\begin{align}
 r = y - X\hat{\theta}
\end{align}
とし,雑音分散推定量は
\begin{align}
 \hat{\sigma}^2 = \frac{\lVert r \rVert^2}{N - p}
\end{align}
($p$ は列数)とする.

パラメータ推定の誤差共分散は
\begin{align}
 \hat{\sigma}^2 (X^\top X)^{-1}
\end{align}
で与えられる.

決定係数は
\begin{align}
 C = \frac{\lVert X\hat{\theta} - \bar{y}\mathbf{1} \rVert^2}
          {\lVert y - \bar{y}\mathbf{1} \rVert^2}
\end{align}
と定義する.

以下の R 関数が上記推定を実装している.

\begin{lstlisting}[language=R]
regression_simple <- function(x, y){
    theta_hat  <- solve(t(x) %*% x) %*% t(x) %*% y
    sigma2_hat <- as.numeric(t(y - x %*% theta_hat) %*%
                             (y - x %*% theta_hat) /
                             (nrow(x) - ncol(x)))
    err_cov_mat <- sigma2_hat * solve(t(x) %*% x)
    det_coef <- sum((x %*% theta_hat - mean(y))^2) /
                sum((y - mean(y))^2)
    list(theta_hat=theta_hat, err_cov_mat=err_cov_mat,
         sigma2_hat=sigma2_hat, det_coef=det_coef)
}
\end{lstlisting}


\subsection{課題1(重回帰)}

\paragraph{モデル}
$y_i = x_i^\top \theta + w_i \ (i=1,\dots,N)$.$x_i\in\mathbb{R}^2$.$w_i$ は独立,同分散 $\sigma^2$,平均0.
$X=[x_1^\top;\dots;x_N^\top]\in\mathbb{R}^{N\times2}$,$y=[y_1;\dots,y_N]\in\mathbb{R}^N$.

\paragraph{推定量}
最小二乗推定量
\[
\hat\theta_N=(X^\top X)^{-1}X^\top y .
\]
残差 $r=y-X\hat\theta_N$ により
\[
\hat\sigma^2=\frac{\|r\|_2^2}{N-p},\quad p=2, \qquad
\widehat{\mathrm{Cov}}(\hat\theta_N)=\hat\sigma^2\,(X^\top X)^{-1}.
\]
決定係数
\[
R^2=\frac{\|X\hat\theta_N-\bar y\mathbf{1}\|_2^2}{\|y-\bar y\mathbf{1}\|_2^2},\quad
\bar y=\frac1N\sum_{i=1}^N y_i .
\]

\paragraph{実装}
データから設計行列 $X$ と観測ベクトル $y$ を構成し,$\hat\theta = (X^\top X)^{-1} X^\top y$ により推定量を計算する.
残差 $r = y - X\hat\theta$ から雑音分散推定量 $\hat\sigma^2 = \|r\|^2/(N-p)$ を求め,
推定誤差共分散 $\hat\sigma^2 (X^\top X)^{-1}$ および決定係数 $R^2$ を算出する関数を実装した.
以下がRによる実装例である。
\begin{lstlisting}[language=R]
# データの読み込み
data <- read.csv("datas/mmse_kadai1.csv", header=FALSE,
                 col.names = c("x1","x2","y"))
x <- as.matrix(data[,c("x1","x2")])
y <- as.matrix(data[,"y"])

# サンプルサイズ n で推定を実行する関数
exp1 <- function (x, y, n){
    x_n <- x[1:n,]
    y_n <- y[1:n,]
    result <- regression_simple(x_n, y_n)
    return(list(theta_hat = result$theta_hat, 
                err_cov_mat = result$err_cov_mat, 
                det_coef = result$det_coef))
}

N_max <- nrow(x)
print(exp1(x, y, N_max)$theta_hat)
print(exp1(x, y, N_max)$err_cov_mat)
print(exp1(x, y, N_max)$det_coef)

# 収束の可視化
plot_exp1 <- function(x, y){
    ns <- c(2,4,8,16,32,64,128,256,512,1024,2048,4096,8192)
    theta_hats <- matrix(0, nrow=length(ns), ncol=ncol(x))
    for (i in 1:length(ns)){
        result <- exp1(x, y, ns[i])
        theta_hats[i, ] <- as.vector(result$theta_hat)  
    }
    png("graphs/task1.png")
    matplot(ns, theta_hats, type="b", log="x", xlab="n", ylab="theta_hat",
            main="Convergence of theta_hat")
    dev.off()
}
plot_exp1(x, y)
\end{lstlisting}

\paragraph{収束確認}
$N\in\{2,4,8,\dots,2^{13}=8192\}$ で $\hat\theta_N$ を計算し,$N$ を横軸とする片対数図で各成分を同一図に描く.

\paragraph{結果(全データ $N=10000$)}
\[
\hat\theta_{N}=
\begin{bmatrix}
1.507\\
1.998
\end{bmatrix},\qquad
\widehat{\mathrm{Cov}}(\hat\theta_{N})=
\begin{bmatrix}
9.867 & -0.04081\\
-0.04081 & 10.05
\end{bmatrix}\times10^{-5}.
\]
決定係数
R^2=0.8630 .
\[
\]

\begin{figure}[H]
  \centering
  \includegraphics[width=0.75\linewidth]{graphs/task1.png}
  \caption{$\hat\theta_N$ の収束(横軸 $N=2,4,\dots,8192$ の片対数)}
  \label{fig:task1-conv}
\end{figure}

\paragraph {考察}
図\ref{fig:task1-conv}から,$\hat\theta_N$ は $N$ を増やすにつれて収束していることが確認できた.
また、共分散行列の非対角成分が対角成分に比べて非常に小さいことから,確かに各成分の推定誤差が独立であることも示唆された。

\subsection{課題2(多項式回帰)}

\paragraph{モデル}
$y_i=\varphi(x_i)\theta+w_i,\quad \varphi(x)=[\,1\ x\ x^2\ x^3\,],\ i=1,\dots,N.$
雑音 $w_i$ は独立,同分散 $\sigma^2$,平均0.分散9の正規分布.
$X=[\varphi(x_1);\dots;\varphi(x_N)]\in\mathbb{R}^{N\times4}$.

\paragraph{推定量}
推定量・共分散・決定係数は課題1と同じ形式($p=4$)で計算する.

\paragraph{実装}
入力 $x_i$ から特徴ベクトル $\varphi(x_i) = [1\ x_i\ x_i^2\ x_i^3]$ を構成し,設計行列 $X$ を作成する.
課題1と同様の関数を用いてパラメータ推定量,誤差共分散,決定係数を算出する.
$N\in\{4,8,16,\dots,8192\}$ の各点で推定量を計算し,片対数プロットで収束の様子を可視化した.

\paragraph{収束確認}
$N\in\{4,8,16,\dots,8192\}$ で $\hat\theta_N$ を計算し,$N$ を横軸とする片対数図で各成分を同一図に描画する.

\paragraph{結果(全データ $N=10000$)}
\[
\hat\theta_N=\begin{bmatrix}
-0.5090\\
\ \ 1.976\\
\ \ 0.1977\\
-0.09867
\end{bmatrix},\qquad
\widehat{\mathrm{Cov}}(\hat\theta_N)=\begin{bmatrix}
2.023 & -0.01186 & -0.1348 & 0.0003971\\
-0.01186 & 6.753 & -0.001899 & -37.95\\
-0.1348 & -0.001899 & 16.04 & 0.06352\\
0.0003971 & -37.95 & 0.06352 & 2.529
\end{bmatrix}\times10^{-3}.
\]


決定係数:
\[
R^2=0.4619 .
\]

\begin{figure}[H]
  \centering
  \includegraphics[width=0.75\linewidth]{graphs/task2.png}
  \caption{$\hat\theta_N$ の収束(横軸 $N=4,8,\dots,8192$ の片対数)}
  \label{fig:task2-conv}
\end{figure}

\paragraph {考察}
図\ref{fig:task2-conv}から,課題1と同様に,$\hat\theta_N$ は $N$ を増やすにつれて収束し、対角成分に比べて非対角成分が小さいかったものの、課題1よりも収束が遅く、非対角成分の割合がやや大きいことが確認できた。なにより,モデルの精度を表す決定係数 $R^2$ が課題1に比べて大幅に低下しており,モデルの当てはまりが悪化していることが分かる。
これについて、課題2では雑音分散が9と大きく設定されており、これは元データのxに対する影響が大きいため、モデルの当てはまりが悪化したと考えられる。特に定数項と1次項の収束が遅いことから、これらの成分が雑音よりも十分に大きくないのでその影響を受けやすいことが示唆される。


\subsection{課題3(分散の観測誤差:Cauchy)}

\paragraph{モデルと注意}
$y_i=x_i^\top\theta+w_i\ (i=1,\dots,N)$,$x_i\in\mathbb{R}^2$.
観測誤差 $w_i\sim \mathrm{Cauchy}(0,1)$ 独立同分布.

\paragraph{推定量}
推定量・共分散・決定係数は課題1と同じ形式($p=2$)で計算する.

\paragraph{実装}
課題1と同じ最小二乗推定の枠組みを用いて,$N\in\{2,4,\dots,8192\}$ の各サンプルサイズで推定量を計算した.


\paragraph{結果(全データ $N=10000$)}
\[
\hat\theta_N=
\begin{bmatrix}
2.373\\
1.537
\end{bmatrix},\qquad
\widehat{\mathrm{Cov}}(\hat\theta_N)=
\begin{bmatrix}
9.235 & -0.003642\\
-0.003642 & 9.264
\end{bmatrix}.
\]
参考:対角の平方根は
$\mathrm{SE}(\hat\theta_1)\approx 3.039,\ 
 \mathrm{SE}(\hat\theta_2)\approx 3.044$.
決定係数(参考値)は
\[
R^2=0.0002841 .
\]

\begin{figure}[H]
  \centering
  \includegraphics[width=0.75\linewidth]{graphs/task3.png}
  \caption{$\hat\theta_N$ の非収束例(横軸 $N=2,4,\dots,8192$ の片対数)}
  \label{fig:task3-conv}
\end{figure}

\paragraph{考察}
図\ref{fig:task3-conv}より,資料にあるように、Cauchy 誤差では外れ値の影響が支配的で,$\hat\theta_N$ は $N$ を増やしても収束せず、決定係数もほぼ0であることが確認できた。


\subsection{課題4(入力域の制約と設計)}

\paragraph{設定}
課題2と同じ $\varphi(x)=[\,1\ x\ x^2\ x^3\,]$, 真の $\theta$, 観測誤差 $w_i\sim\mathcal{N}(0,9)$ とする.
ただし入力は $x_i\in[0,1]$, $i=1,\dots,10000$.
$X=[\varphi(x_1);\dots;\varphi(x_N)]\in\mathbb{R}^{N\times4}$, $y=[y_1;\dots;y_N]$.

\paragraph{推定量}
推定量・共分散・決定係数は課題1と同じ形式($p=4$)で計算する.

\paragraph{実装}
課題2と同様に $\varphi(x)=[1\ x\ x^2\ x^3]$ の特徴ベクトルを用いて設計行列を構成し,
全データ $N=10000$ に対して最小二乗推定を行った.

\paragraph{結果($N=10000$)}
\[
\hat\theta_{(4)}=
\begin{bmatrix}
-0.5777\\
\ \ 2.097\\
-0.3950\\
\ \ 0.5491
\end{bmatrix},\quad
\widehat{\mathrm{Cov}}(\hat\theta_{(4)})=\begin{bmatrix}
0.1532 & -1.149 & \ \ 2.296 & -1.339\\
-1.149 & 11.46 & -25.75 & 16.01\\
\ \ 2.296 & -25.75 & 61.72 & -39.97\\
-1.339 & 16.01 & -39.97 & 26.62
\end{bmatrix},\quad
R^2=0.04084.
\]


\paragraph{課題2.1との比較($N=10000$)}
課題2.1の推定結果:
\[
\hat\theta_{(2.1)}=
\begin{bmatrix}
-0.5090\\
\ \ 1.976\\
\ \ 0.1977\\
-0.09867
\end{bmatrix},\quad
R^2=0.4619.
\]



\paragraph{考察}
精度が大幅に低下していることが確認できた。これは、入力 $x$ が $[0,1]$ に制約されることで、特徴量 $\varphi(x)$ の値域が狭まり、設計行列 $X$ の列間の相関が高くなったことが原因と考えられる。また,$x$ が $[0,1]$に制約されることは、高次項も小さな値に限定されることを意味し、雑音の分散が9であることを考慮すると、モデルの当てはまりがさらに悪化したと考えられる。
入力設計の工夫としては、
\begin{itemize}\setlength{\itemsep}{0pt}
\item xの分散を十分に大きくすることで、雑音の影響が相対的に小さくなるようにする。
\item xの値を[0,1]区間から大きく離すことで、特徴量同士の相関を低減させる。
\end{itemize}
が挙げられる。

\section{重み付き最小二乗法}

\subsection{原理と方法}

\paragraph{原理}
各観測 $i=1,\dots,N$ で $y_i\in\mathbb{R}^m$, $X_i\in\mathbb{R}^{m\times p}$ とし
\[
y_i = X_i\theta + w_i,\qquad \mathbb{E}[w_i]=0,\ \mathrm{Cov}(w_i)=V\ (\text{既知}), 
\]
を仮定する。$Q:=V^{-1}$ とおく。WLS は重み付き残差平方和
\[
J(\theta)=\sum_{i=1}^N (y_i-X_i\theta)^\top Q\,(y_i-X_i\theta)
\]
を最小化する推定で,正規方程式は
\[
S\hat\theta=b,\qquad 
S:=\sum_{i=1}^N X_i^\top QX_i,\quad 
b:=\sum_{i=1}^N X_i^\top Qy_i.
\]
したがって
\[
\hat\theta=(\sum_{i=1}^N X_i^\top QX_i)^{-1}\Big(\sum_{i=1}^N X_i^\top Qy_i\Big).
\]
$V$ が既知のとき
\[
\mathrm{Cov}(\hat\theta)=S^{-1}.
\]
$V=\sigma^2\Sigma$ のようにスケール未知なら
\[
\hat\sigma^2=\frac{\sum_{i=1}^N r_i^\top \Sigma^{-1} r_i}{Nm-p},\quad r_i:=y_i-X_i\hat\theta,\qquad
\widehat{\mathrm{Cov}}(\hat\theta)=\hat\sigma^2\,(\sum X_i^\top \Sigma^{-1}X_i)^{-1}.
\]

\paragraph{方法}
(1) 各 $i$ の設計行列 $X_i$ と観測 $y_i$ を用意($X_i$ は $m\times p$)。  
(2) $Q=V^{-1}$ を決めて $S=\sum X_i^\top QX_i,\ b=\sum X_i^\top Qy_i$ を計算。  
(3) $\hat\theta=S^{-1}b$。  

\paragraph{実装}
R による実装例を以下に示す。
\begin{lstlisting}
regression_multiple <- function(x, y, V = NULL, Q = NULL){
    # y: n×1行列(またはベクトル)
    if (is.null(ncol(y))) {
        y <- as.matrix(y, ncol = 1)
    }
    if (length(dim(x)) != 3) {
        stop("x must be a 3-dimensional array")
    }
    n <- dim(x)[3]
    m <- ncol(y)
    p <- dim(x)[1] # 特徴量数
    S <- matrix(0, p, p)
    b <- matrix(0, p, 1)
    T <- matrix(0, p, p)
    if (is.null(V)) {
        V <- diag(m)
    }
    if (is.null(Q)) {
        Q <- solve(V)
    }
    # m=1(スカラー出力)の場合はQ,Vをスカラー化
    if (m == 1) {
        Q <- as.numeric(Q)
        V <- as.numeric(V)
        for (i in 1:n) {
            xi <- as.matrix(x[,,i]) # (p,1)
            S <- S + xi %*% t(xi) * Q
            b <- b + xi * Q * y[i, 1]
            T <- T + xi %*% t(xi) * V
        }
    } else {
        for (i in 1:n) {
            xi <- as.matrix(x[,,i])
            yi <- matrix(y[i, ], nrow = m, ncol = 1)
            S <- S + t(xi) %*% Q %*% xi
            b <- b + t(xi) %*% Q %*% yi
            T <- T + t(xi) %*% V %*% xi
        }
    }
    theta_hat <- solve(S) %*% b
    err_cov_mat <- solve(S) %*% T %*% solve(S)
    return(list(theta_hat = theta_hat, err_cov_mat = err_cov_mat, S = S, b = b, T = T))
}
\end{lstlisting}

\subsection{課題5(2次元出力:重み付き最小二乗法)}

\paragraph{課題の内容}
入力 $x_i \in \mathbb{R}$($i=1,\dots,1000$)に対し,$m=2$ 次元出力の線形モデル
\[
  y_i \;=\; \phi(x_i)\,\theta \;+\; w_i,\qquad
  \phi(x) = 
  \begin{bmatrix}
    1 & x\\
    1 & x^2
  \end{bmatrix},
  \quad
  w_i \sim \mathcal{N}(0,V),\;
  V=\begin{bmatrix}100&0\\0&1\end{bmatrix},
\]
を考える.ここで $\theta\in\mathbb{R}^2$ を推定対象とする.

\paragraph{推定法}
最小二乗(OLS)の推定値は
\[
  \hat{\theta}_N \;=\;
  \Bigl(\sum_{i=1}^N \phi_i^\top \phi_i \Bigr)^{-1}
  \sum_{i=1}^N \phi_i^\top y_i,
\]
で与えられる($\phi_i=\phi(x_i)$).
観測雑音の共分散 $V$ が既知で,その影響を反映した重み付き最小二乗(WLS)を用いると,
$V=W^2$ を満たす $W$ に対し
\[
  \hat{\theta}_N \;=\;
  \Bigl(\sum_{i=1}^N \phi_i^\top W^{-2}\phi_i \Bigr)^{-1}
  \sum_{i=1}^N \phi_i^\top W^{-2} y_i
  \;=\;
  \Bigl(\sum_{i=1}^N \phi_i^\top Q\,\phi_i \Bigr)^{-1}
  \sum_{i=1}^N \phi_i^\top Q\, y_i,
\]
となる($Q=W^{-2}=V^{-1}$).

\paragraph{実装}
\texttt{datas/mmse\_kadai5.csv} を読み込み,$x$ から
\(
\phi(x)=\begin{bmatrix}1&x\\1&x^2\end{bmatrix}
\)
を構成して $x\in\mathbb{R}^{2\times 2\times N}$ の 3 次元配列とし,
$y\in\mathbb{R}^{N\times 2}$ を観測行列とした.
推定は関数 \texttt{regression\_multiple} を用い,
\textbf{OLS} は無重み($Q=I$),
\textbf{WLS} は $V=\mathrm{diag}(100,1)$ を渡し内部で $Q=V^{-1}$ を用いる設定とした.
また,$N\in\{4,8,16,32,64,128,256,512,1000\}$ に対する
$\hat{\theta}_N$ の収束の様子を片対数($x$ 軸のみ対数)で可視化した.

\paragraph{結果}
全データ($N=1000$)を用いた推定結果は次のとおりである.

\medskip
\noindent\textbf{OLS}
\[
  \hat{\theta}_{\mathrm{OLS}}=
  \begin{bmatrix}
    2.994567\\[-1mm]
   -2.068971
  \end{bmatrix},\qquad
  \widehat{\mathrm{Cov}}(\hat{\theta}_{\mathrm{OLS}})=
  \begin{bmatrix}
    5.7621\!\times\!10^{-4} & -1.5041\!\times\!10^{-4}\\
    -1.5041\!\times\!10^{-4} & 2.9684\!\times\!10^{-4}
  \end{bmatrix}.
\]

\noindent\textbf{WLS}
\[
  \hat{\theta}_{\mathrm{WLS}}=
  \begin{bmatrix}
    2.939084\\[-1mm]
   -1.986465
  \end{bmatrix},\qquad
  \widehat{\mathrm{Cov}}(\hat{\theta}_{\mathrm{WLS}})=
  \begin{bmatrix}
    2.39993\!\times\!10^{-1} & -9.66541\!\times\!10^{-2}\\
    -9.66541\!\times\!10^{-2} & 4.99126\!\times\!10^{-2}
  \end{bmatrix}.
\]

\begin{figure}[H]
  \centering
  \includegraphics[width=.82\linewidth]{graphs/task5_ols.png}
  \caption{$\hat{\theta}_N$ の収束(OLS)}
  \label{fig:task5_ols}
\end{figure}

\begin{figure}[H]
  \centering
  \includegraphics[width=.82\linewidth]{graphs/task5_wls.png}
  \caption{$\hat{\theta}_N$ の収束(WLS,$V=\mathrm{diag}(100,1)$)}
  \label{fig:task5_wls}
\end{figure}

\paragraph{考察}
% (ここに考察を記述)


\subsection{課題6(2次元出力・異分散2群に対する重み付き最小二乗)}

\paragraph{課題の内容}
入力 $x_i\in\mathbb{R}$ に対し,
\[
  y_i \;=\; \phi(x_i)\,\theta + w_i,\qquad
  \phi(x)=
  \begin{bmatrix}
    1 & x\\[1mm]
    1 & x^2
  \end{bmatrix},\quad
  \theta\in\mathbb{R}^2,
\]
という $m=2$ 次元出力の線形モデルを考える.雑音は $N$ サンプルのうち前半 $i=1,\dots,\texttt{num}$ が
$w_i\sim\mathcal{N}(0,V_1)$,後半 $i=\texttt{num}+1,\dots,N$ が $w_i\sim\mathcal{N}(0,V_2)$ に従うとする.
本課題では $V_1=\mathrm{diag}(100,1)$,$V_2=\mathrm{diag}(2,1)$,$\texttt{num}=500$ を既知として推定を行う.

\paragraph{推定法(GLS/WLS)}
$Q_k=V_k^{-1}$($k=1,2$)とおくと,一般化最小二乗(WLS)の正規方程式は
\[
  S\,\hat{\theta}_N \;=\; b,
  \quad
  S \;=\; \sum_{i=1}^{\texttt{num}}\!\phi_i^\top Q_1\phi_i \;+\; \sum_{i=\texttt{num}+1}^{N}\!\phi_i^\top Q_2\phi_i,
  \quad
  b \;=\; \sum_{i=1}^{\texttt{num}}\!\phi_i^\top Q_1 y_i \;+\; \sum_{i=\texttt{num}+1}^{N}\!\phi_i^\top Q_2 y_i,
\]
より
\[
  \hat{\theta}_N \;=\; S^{-1} b.
\]
標本ベースの誤差共分散推定は(実装に合わせて)
\[
  \widehat{\mathrm{Cov}}(\hat{\theta}_N)
  \;=\;
  S^{-1}\,T\,S^{-1},\qquad
  T \;=\; \sum_{i=1}^{\texttt{num}}\!\phi_i^\top V_1\phi_i \;+\; \sum_{i=\texttt{num}+1}^{N}\!\phi_i^\top V_2\phi_i .
\]
なお OLS は $Q_1=Q_2=I$ とした特別な場合である.

\paragraph{実装}
\texttt{datas/mmse\_kadai6.csv} から $x$ を読み,$\phi(x)$ を用いて
$x\in\mathbb{R}^{2\times2\times N}$,$y\in\mathbb{R}^{N\times2}$ を構成した.
推定は関数 \texttt{regression\_multiple\_different\_distributions} により,
前半 $V_1$,後半 $V_2$ の重みで $S,b,T$ を分割加算して $\hat{\theta}_N$ と
$\widehat{\mathrm{Cov}}(\hat{\theta}_N)$ を得た.
また $N$ を \texttt{ns = seq(4,1000,length.out=20)} で走らせ,
$\hat{\theta}_N$ の収束の様子を折れ線で可視化した($x$ 軸は線形目盛).

\paragraph{結果}
全データ($N=1000$,$\texttt{num}=500$)の推定結果は以下のとおりである.

\medskip
\noindent\textbf{OLS(比較)}
\[
  \hat{\theta}_{\mathrm{OLS}}=
  \begin{bmatrix}
    3.188356\\[-1mm]
   -2.092183
  \end{bmatrix},\qquad
  \widehat{\mathrm{Cov}}(\hat{\theta}_{\mathrm{OLS}})=
  \begin{bmatrix}
    5.692084\!\times\!10^{-4} & -1.402629\!\times\!10^{-4}\\
    -1.402629\!\times\!10^{-4} & 2.842674\!\times\!10^{-4}
  \end{bmatrix}.
\]

\noindent\textbf{WLS($V_1=\mathrm{diag}(100,1)$,$V_2=\mathrm{diag}(2,1)$)}
\[
  \hat{\theta}_{\mathrm{WLS}}=
  \begin{bmatrix}
    2.994202\\[-1mm]
   -2.014699
  \end{bmatrix},\qquad
  \widehat{\mathrm{Cov}}(\hat{\theta}_{\mathrm{WLS}})=
  \begin{bmatrix}
    6.019710\!\times\!10^{-2} & -2.227335\!\times\!10^{-2}\\
    -2.227335\!\times\!10^{-2} & 1.276759\!\times\!10^{-2}
  \end{bmatrix}.
\]

\begin{figure}[H]
  \centering
  \includegraphics[width=.82\linewidth]{graphs/task6_ols.png}
  \caption{$\hat{\theta}_N$ の収束(OLS)}
  \label{fig:task6_ols}
\end{figure}

\begin{figure}[H]
  \centering
  \includegraphics[width=.82\linewidth]{graphs/task6_wls.png}
  \caption{$\hat{\theta}_N$ の収束(WLS,$V_1$/$V_2$ 異分散)}
  \label{fig:task6_wls}
\end{figure}

\paragraph{考察}
% (ここに考察を記述)


\section{推定値の合成と逐次最小二乗法}


\subsection{原理と方法}

\paragraph{推定値の合成(Estimator Fusion)}
$f(\theta,x)=\phi(x)\theta$ の線形回帰で,$D_N$ と $D'_M$ から得た推定
\(
\hat\theta_N=\Phi_N \sum_{i=1}^N \phi_i^\top y_i,\;
\hat\theta'_M=\Phi'_M \sum_{j=1}^M \phi'_j{}^\top y'_j
\)
($\Phi_N=(\sum \phi_i^\top\phi_i)^{-1}$)を持つとき,結合推定は
\[
  \hat\theta_{N+M}
  =\bigl(\Phi_N^{-1}+\Phi_M'{}^{-1}\bigr)^{-1}
   \bigl(\Phi_N^{-1}\hat\theta_N+\Phi_M'{}^{-1}\hat\theta'_M\bigr)
\]
で与えられる。\cite{exp2025}

以下が R による実装例である。

\begin{lstlisting}
# 既存バッチN=6000とM=4000の合成(S = Φ^{-1} を情報行列とする)
# S_6000, S_4000 はそれぞれ Φ_6000^{-1}, Φ_4000^{-1}
# theta_6000, theta_4000 は各バッチの推定ベクトル
theta_6000_4000 <- solve(S_6000 + S_4000) %*%
  (S_6000 %*% theta_6000 + S_4000 %*% theta_4000)
\end{lstlisting}

\paragraph{逐次最小二乗法(RLS)}
逆行列補題を用いると,バッチ解は次の再帰で更新できる:
\[
\begin{aligned}
  \hat\theta_{N+1} &= \hat\theta_N + K_{N+1}\bigl(y_{N+1}-\phi_{N+1}\hat\theta_N\bigr),\\
  K_{N+1} &= \Phi_N \phi_{N+1}^\top\bigl(I_m + \phi_{N+1}\Phi_N\phi_{N+1}^\top\bigr)^{-1},\\
  \Phi_{N+1} &= \Phi_N - K_{N+1}\phi_{N+1}\Phi_N .
\end{aligned}
\]
忘却係数 $\gamma\in(0,1]$ を導入すると
\[
\begin{aligned}
  K_{\gamma,N+1} &= \Phi_{\gamma,N}\phi_{N+1}^\top
    \bigl(\gamma I_m+\phi_{N+1}\Phi_{\gamma,N}\phi_{N+1}^\top\bigr)^{-1},\\
  \Phi_{\gamma,N+1} &= \gamma^{-1}\!\left(\Phi_{\gamma,N}-K_{\gamma,N+1}\phi_{N+1}\Phi_{\gamma,N}\right),\\
  \hat\theta_{\gamma,N+1} &= \hat\theta_{\gamma,N} +
    K_{\gamma,N+1}\bigl(y_{N+1}-\phi_{N+1}\hat\theta_{\gamma,N}\bigr).
\end{aligned}
\]
\cite{exp2025}

以下が R による実装例である。
\begin{lstlisting}
update_regression_mat <- function(Phi_N, phi_N_plus_1, theta_N, y_N_plus_1, gamma = 1){
  I_m <- diag(1)  # m=1想定。m>1なら diag(nrow(phi_N_plus_1 %*% Phi_N %*% t(phi_N_plus_1))) に置換
  # 中間計算
  phi_T_Phi_phi <- phi_N_plus_1 %*% Phi_N %*% t(phi_N_plus_1)
  inv_term <- solve( (gamma * I_m) + phi_T_Phi_phi )
  K <- gamma * (Phi_N %*% t(phi_N_plus_1) %*% inv_term)
  # 更新
  Phi_N_plus_1 <- (Phi_N - K %*% phi_N_plus_1 %*% Phi_N ) / gamma
  theta_N_plus_1 <- theta_N + K %*% (y_N_plus_1 - phi_N_plus_1 %*% theta_N)
  list(Phi_N_plus_1 = Phi_N_plus_1, theta_N_plus_1 = theta_N_plus_1)
}
\end{lstlisting}


$\Phi_0=\varepsilon^{-1}I$( $\varepsilon$は小さい正の実数)で初期化し,逐次更新する。各ステップの計算は
$\dim(\theta)=p$ に対して $O(p^2)$ 程度で済む。\cite{exp2025}

\subsection{課題7(推定値の合成の検証:2分割データの統合)}

\paragraph{課題の内容}
$N=10000$ のデータを前半 $6000$ と後半 $4000$ に分割し,各ブロックで OLS 推定を行ったのち,情報行列の加法性
\[
  S_N \;=\; \sum_{i=1}^N \varphi_i \varphi_i^\top
\]
を用いた合成推定
\[
  \hat\theta_{\mathrm{fuse}}
  \;=\;
  \bigl(S_{6000}+S_{4000}\bigr)^{-1}
  \Bigl(S_{6000}\hat\theta_{6000}+S_{4000}\hat\theta_{4000}\Bigr)
\]
が,全データ一括推定 $\hat\theta_{10000}$ と一致することを確認する。ここでモデルは
\[
  y_i \;=\; \varphi(x_i)^\top \theta + w_i,\quad
  \varphi(x)=
  \begin{bmatrix}
    1\\[1mm]
    \exp\!\bigl(-\tfrac{(x-1)^2}{2}\bigr)\\[1mm]
    \exp\!\bigl(-(x+1)^2\bigr)
  \end{bmatrix}\!,
\quad \theta\in\mathbb{R}^3.
\]
理論上,独立同分布で $V=\sigma^2$ のとき $\hat\theta_{\mathrm{fuse}}=\hat\theta_{10000}$ となる。\cite{exp2025}

\paragraph{実装}
分割データから
\(
S_{6000},\hat\theta_{6000},\;S_{4000},\hat\theta_{4000}
\)
を得て,合成推定値を計算し,全データ一括推定と比較する。

\begin{lstlisting}
# 事前に x, y, x_6000, y_6000, x_4000, y_4000 を用意済み

exp7 <- function(){
  res6000 <- regression_multiple(x_6000, y_6000)
  res4000 <- regression_multiple(x_4000, y_4000)
  res10000 <- regression_multiple(x, y)

  S_6000 <- res6000$S
  S_4000 <- res4000$S
  theta_6000 <- res6000$theta_hat
  theta_4000 <- res4000$theta_hat

  # 推定値の合成(情報行列の和)
  theta_6000_4000 <- solve(S_6000 + S_4000) %*%
    (S_6000 %*% theta_6000 + S_4000 %*% theta_4000)

  cat("S6000:\n"); print(S_6000)
  cat("S4000:\n"); print(S_4000)
  cat("theta6000:\n"); print(theta_6000)
  cat("theta4000:\n"); print(theta_4000)
  cat("theta_fuse:\n"); print(theta_6000_4000)

  theta_10000 <- res10000$theta_hat
  cat("theta10000:\n"); print(theta_10000)

  # 一致性の確認(数値誤差内)
  cat("all.equal(theta_fuse, theta10000): ",
      all.equal(drop(theta_6000_4000), drop(theta_10000),
                tolerance = 1e-10), "\n")
}

exp7()
\end{lstlisting}

\paragraph{結果}
\[
S_{6000}=
\begin{bmatrix}
6000 & 1505 & 1033\\
1505 & 1063 & 226.4\\
1033 & 226.4 & 720.6
\end{bmatrix},\quad
S_{4000}=
\begin{bmatrix}
4000 & 971.2 & 716.3\\
971.2 & 679.5 & 149.7\\
716.3 & 149.7 & 510.2
\end{bmatrix},
\]
\[
\hat\theta_{6000}=
\begin{bmatrix}
-0.003879\\ 3.011\\ -1.989
\end{bmatrix},\;
\hat\theta_{4000}=
\begin{bmatrix}
-0.02538\\ 3.035\\ -1.978
\end{bmatrix},\;
\hat\theta_{\mathrm{fuse}}=
\begin{bmatrix}
-0.01250\\ 3.020\\ -1.985
\end{bmatrix}.
\]
\[
\hat\theta_{10000}=
\begin{bmatrix}
-0.01250\\ 3.020\\ -1.985
\end{bmatrix}
\]

\paragraph{考察}
上の結果から、たしかに $\hat\theta_{\mathrm{fuse}}$ と $\hat\theta_{10000}$ が一致することが確認できた。

\subsection{課題7(推定値の合成の検証:非線形基底・分割データ)}

\paragraph{課題の内容}
基底
\[
  \varphi(x)=
  \begin{bmatrix}
    1\\[1mm]
    \exp\!\bigl(-\tfrac{(x-1)^2}{2}\bigr)\\[1mm]
    \exp\!\bigl(-(x+1)^2\bigr)
  \end{bmatrix}\!,\qquad
  y_i=\varphi(x_i)^\top \theta+w_i,\ \theta\in\mathbb{R}^3
\]
で $N=10000$ 個のデータを前半 $6000$ と後半 $4000$ に分割する。
各ブロックの推定値 $(\hat\theta_{6000},\hat\theta_{4000})$ と情報行列
$S_k=\sum_{i\in\text{block }k}\varphi_i\varphi_i^\top$ を用いて
\[
  \hat\theta_{\mathrm{fuse}}
  =(S_{6000}+S_{4000})^{-1}\!\bigl(S_{6000}\hat\theta_{6000}+S_{4000}\hat\theta_{4000}\bigr)
\]
を計算し,全データ一括推定 $\hat\theta_{10000}$ と照合する。\cite{exp2025}

\paragraph{実装(OLS 版)}
\begin{lstlisting}
data <- read.csv("datas/mmse_kadai7.csv", header=FALSE, col.names=c("x","y"))
x0 <- rep(1, nrow(data))
x1 <- exp(-(data$x - 1)^2 / 2)
x2 <- exp(-(data$x + 1)^2)
n  <- nrow(data)
x  <- array(0, dim=c(3,1,n))
for(i in 1:n){ x[,,i] <- matrix(c(x0[i], x1[i], x2[i]), 3, 1) }
y <- as.matrix(data[,"y"])

x_6000 <- x[,,1:6000, drop=FALSE]; y_6000 <- y[1:6000, , drop=FALSE]
x_4000 <- x[,,6001:10000, drop=FALSE]; y_4000 <- y[6001:10000, , drop=FALSE]

exp7 <- function(){
  r6 <- regression_multiple(x_6000, y_6000)  # returns $S$ and $theta_hat$
  r4 <- regression_multiple(x_4000, y_4000)
  rall <- regression_multiple(x, y)

  S_6000 <- r6$S;  S_4000 <- r4$S
  th6 <- r6$theta_hat; th4 <- r4$theta_hat; th_all <- rall$theta_hat

  th_fuse <- solve(S_6000 + S_4000) %*% (S_6000 %*% th6 + S_4000 %*% th4)

  print(th6); print(th4); print(th_fuse); print(th_all)
  # 一致検証
  print(all.equal(drop(th_fuse), drop(th_all), tolerance=1e-12))
}
exp7()
\end{lstlisting}

\paragraph{実装(推定分散による重み付き合成)}
各ブロックの誤差分散を
\[
  \hat\sigma_k^2=\frac{\mathrm{RSS}_k}{N_k-p},\quad p=3
\]
で推定し,$Q_k=\hat\sigma_k^{-2}$ を用いた WLS 情報行列
$S_k=\sum_{i\in k}\varphi_i Q_k \varphi_i^\top=Q_k\sum_{i\in k}\varphi_i\varphi_i^\top$
で合成する($k\in\{6000,4000\}$)。コードは次のとおり。
\begin{lstlisting}
x_mat       <- t(matrix(x,       nrow=3, ncol=n))
x_6000_mat  <- t(matrix(x_6000,  nrow=3, ncol=6000))
x_4000_mat  <- t(matrix(x_4000,  nrow=3, ncol=4000))

exp8 <- function(){
  # OLS 推定
  th_all  <- regression_multiple(x,       y)$theta_hat
  th_6000 <- regression_multiple(x_6000,  y_6000)$theta_hat
  th_4000 <- regression_multiple(x_4000,  y_4000)$theta_hat

  # 分散推定(MSE)
  V_hat_full  <- sum((y       - x_mat      %*% th_all )^2)  /(nrow(x_mat)      - ncol(x_mat))
  V_hat_6000  <- sum((y_6000  - x_6000_mat %*% th_6000)^2) /(nrow(x_6000_mat)  - ncol(x_6000_mat))
  V_hat_4000  <- sum((y_4000  - x_4000_mat %*% th_4000)^2) /(nrow(x_4000_mat)  - ncol(x_4000_mat))

  # 各ブロックの WLS 情報行列と推定
  r6 <- regression_multiple(x_6000,  y_6000, V=V_hat_6000)
  r4 <- regression_multiple(x_4000,  y_4000, V=V_hat_4000)
  S_6000 <- r6$S;  S_4000 <- r4$S
  th6 <- r6$theta_hat; th4 <- r4$theta_hat

  # 合成推定(WLS 版)
  th_fuse <- solve(S_6000 + S_4000) %*% (S_6000 %*% th6 + S_4000 %*% th4)

  # 参考:全データにスカラー重み V_hat_full をかけた WLS(OLS と同値)
  th_full_wls <- regression_multiple(x, y, V=V_hat_full)$theta_hat

  print(th6); print(th4); print(th_fuse); print(th_full_wls)
}
exp8()
\end{lstlisting}

\paragraph{結果}
実行例(あなたの出力):
\[
\hat\theta_{6000}=\begin{bmatrix}
-0.003879\\ 3.011\\ -1.989
\end{bmatrix},\quad
\hat\theta_{4000}=\begin{bmatrix}
-0.02538\\ 3.035\\ -1.978
\end{bmatrix}.
\]
WLS 合成推定と全データ一括 WLS(スカラー重み):
\[
\hat\theta_{\mathrm{fuse}}=\begin{bmatrix}
-0.01245\\ 3.020\\ -1.985
\end{bmatrix},\qquad
\hat\theta_{10000}=\begin{bmatrix}
-0.01250\\ 3.020\\ -1.985
\end{bmatrix}.
\]
差分は
$\Delta\theta\approx(4.17\times10^{-5},-4.70\times10^{-5},-2.25\times10^{-5})^\top$。

\paragraph{考察}
ブロックごとに異なる重み($Q_{6000}\neq Q_{4000}$)を用いた合成と,全体に単一重みをかけた WLS は厳密には一致しないが,差は十分小さい。微小な差が生じる原因として,各ブロックでの分散推定 $\hat\sigma_k^2$ のばらつきが考えられる。



\subsection{課題9(逐次最小二乗法によるシステム同定)}

\paragraph{モデル}
離散化されたバネ・マス・ダンパ系($M=2,\ D=1,\ K=3,\ \Delta t=0.01$)
\[
y_k
= a_1\,y_{k-1}+a_2\,y_{k-2}+b\,F_{k-2}+w_k,\quad
\phi_k=\begin{bmatrix} y_{k-1}&y_{k-2}&F_{k-2}\end{bmatrix},
\]
\[
a_1=2-\tfrac{D}{M}\Delta t=1.995,\quad
a_2=-(1-\tfrac{D}{M}\Delta t+\tfrac{K}{M}\Delta t^2)=-0.99515,\quad
b=\tfrac{\Delta t^2}{M}=5.0\times10^{-5}.
\]
$w_k\sim\mathrm{Uniform}[-1,1]$。RLS 更新は \S3.3 の式に従う(忘却なし $\gamma=1$)\cite{exp2025}。

\paragraph{使用コード(共通)}
\begin{lstlisting}
w <- function(){ runif(1, min = -1, max = 1) }

update_tick <- function(y_k_1, y_k_2, F_k_2,
                        M=2, D=1, K=3, delta_t=0.01){
  (2 - (D/M)*delta_t) * y_k_1 -
  (1 - (D/M)*delta_t + (K/M)*delta_t^2) * y_k_2 +
  (delta_t^2/M) * F_k_2 + w()
}

# RLS(§3.3 の式に一致)
update_regression_mat <- function(Phi_N, phi_N_plus_1, theta_N, y_N_plus_1, gamma=1){
  I_m <- diag(1)
  phi_T_Phi_phi <- phi_N_plus_1 %*% Phi_N %*% t(phi_N_plus_1)
  inv_term <- solve((gamma * I_m) + phi_T_Phi_phi)
  K <- gamma * (Phi_N %*% t(phi_N_plus_1) %*% inv_term)
  Phi_N_plus_1 <- (Phi_N - K %*% phi_N_plus_1 %*% Phi_N) / gamma
  theta_N_plus_1 <- theta_N + K %*% (y_N_plus_1 - phi_N_plus_1 %*% theta_N)
  list(Phi_N_plus_1=Phi_N_plus_1, theta_N_plus_1=theta_N_plus_1)
}
\end{lstlisting}

\subsubsection*{(1) 入力 $F_k\equiv 1$(定数入力)}
\paragraph{内容}
$F_k=1$ を一定とし,$k=1,\dots,10000$ で RLS により
$\theta=[a_1,a_2,b]^\top$ を推定する。\cite{exp2025}

\paragraph{結果}
実行出力:
\[
\hat\theta=
\begin{bmatrix}
1.995154\\
-0.995324\\
2.151\times10^{-3}
\end{bmatrix}.
\]

\paragraph{考察}
$a_1,a_2$ は真値に近い。$b$ は真値 $5\times10^{-5}$ に対して過大。
定数入力で効果が極小かつ雑音が大きく,入力励起が不十分。
$F$ の情報が弱く $b$ の識別性が低い。

\paragraph{コード}
\begin{lstlisting}
exp9_1 <- function(){
  y_k <- y_k_1 <- y_k_2 <- 0; F_k_2 <- 1
  Phi <- diag(3) * 1000; theta <- matrix(0, nrow=3)
  for(i in 1:10000){
    y_k_2 <- y_k_1; y_k_1 <- y_k
    y_k <- update_tick(y_k_1, y_k_2, F_k_2)
    xi <- matrix(c(y_k_1, y_k_2, F_k_2), nrow=1)
    upd <- update_regression_mat(Phi, xi, theta, y_k)
    Phi <- upd$Phi_N_plus_1; theta <- upd$theta_N_plus_1
  }
  print(theta)
}
\end{lstlisting}

\subsubsection*{(2) 入力 $F_k=\sin(\pi k/5)$(小振幅正弦)}
\paragraph{内容}
$F$ を低振幅の正弦波にして同様に推定。\cite{exp2025}

\paragraph{結果}
実行出力:
\[
\hat\theta=
\begin{bmatrix}
1.995721\\
-0.995895\\
2.637\times10^{-3}
\end{bmatrix}.
\]

\paragraph{考察}
$F$ は時間変化するが振幅が小さく,$b$ の寄与は依然として
雑音に埋もれる。$b$ は真値より二桁以上大きい。
入力励起の大きさが推定精度の律速。

\paragraph{コード}
\begin{lstlisting}
exp9_2 <- function(){
  y_k <- y_k_1 <- y_k_2 <- 0
  Phi <- diag(3) * 1000; theta <- matrix(0, nrow=3)
  for(i in 1:10000){
    F_k_2 <- sin(pi * i / 5)
    y_k_2 <- y_k_1; y_k_1 <- y_k
    y_k <- update_tick(y_k_1, y_k_2, F_k_2)
    xi <- matrix(c(y_k_1, y_k_2, F_k_2), nrow=1)
    upd <- update_regression_mat(Phi, xi, theta, y_k)
    Phi <- upd$Phi_N_plus_1; theta <- upd$theta_N_plus_1
  }
  print(theta)
}
\end{lstlisting}

\subsubsection*{(3) 入力 $F_k=10000\sin(\pi k/5)$(大振幅正弦)}
\paragraph{内容}
入力振幅を $10^4$ 倍して強い励起を与える。\cite{exp2025}

\paragraph{結果}
実行出力:
\[
\hat\theta=
\begin{bmatrix}
1.995000\\
-0.995125\\
4.884\times10^{-5}
\end{bmatrix}.
\]

\paragraph{考察}
$b$ が真値 $5.0\times10^{-5}$ に一致。十分な励起で入出力比が改善し,
$F$ の係数が正しく同定できる。$a_1,a_2$ も誤差が小さい。
RLS はオンラインに有効だが,入力の持続励起とスケール設計が必要。

\paragraph{コード}
\begin{lstlisting}
exp9_3 <- function(){
  y_k <- y_k_1 <- y_k_2 <- 0
  Phi <- diag(3) * 1000; theta <- matrix(0, nrow=3)
  for(i in 1:10000){
    F_k_2 <- 10000 * sin(pi * i / 5)
    y_k_2 <- y_k_1; y_k_1 <- y_k
    y_k <- update_tick(y_k_1, y_k_2, F_k_2)
    xi <- matrix(c(y_k_1, y_k_2, F_k_2), nrow=1)
    upd <- update_regression_mat(Phi, xi, theta, y_k)
    Phi <- upd$Phi_N_plus_1; theta <- upd$theta_N_plus_1
  }
  print(theta)
}
\end{lstlisting}


\subsection{課題10(非定常時系列の追従:忘却係数付きRLS)}

\paragraph{内容}
非定常信号
\[
y_k=\theta_k+w_k,\quad \theta_k=\sin(10^{-4}k),\quad w_k\in\{-1,+1\}
\]
を 1 次元回帰 $\phi_k\equiv[1]$ で逐次推定する。忘却係数 $\gamma=0.99$ を用いる。\cite{exp2025}

\paragraph{方法}
忘却付き RLS の更新($m=1$):
\[
\begin{aligned}
K_k&=\Phi_{k-1}\bigl(\gamma+\phi_k\Phi_{k-1}\phi_k^\top\bigr)^{-1}\phi_k^\top,\\
\theta_k^{\text{hat}}&=\theta_{k-1}^{\text{hat}}+K_k\bigl(y_k-\phi_k\theta_{k-1}^{\text{hat}}\bigr),\\
\Phi_k&=\gamma^{-1}\!\left(\Phi_{k-1}-K_k\phi_k\Phi_{k-1}\right),
\end{aligned}
\]
初期化は $\theta_0^{\text{hat}}=0,\ \Phi_0=1000$。$\gamma=0.99$ は有効窓長の目安 $1/(1-\gamma)\approx 100$ に相当。\cite{exp2025}

\paragraph{実装}
\begin{lstlisting}
exp10 <- function(){
  y_k <- 0
  Phi <- matrix(1000, 1, 1)
  theta_hat <- matrix(0, 1, 1)
  theta_hats <- c()
  for(k in 1:10000){
    w_k <- sample(c(-1, 1), 1)              # ノイズ
    y_k <- sin(0.0001 * k) + w_k            # 観測
    xi <- matrix(1, 1, 1)                   # φ_k = 1
    upd <- update_regression_mat(Phi, xi, theta_hat, y_k, gamma=0.99)
    Phi <- upd$Phi_N_plus_1
    theta_hat <- upd$theta_N_plus_1
    theta_hats <- rbind(theta_hats, theta_hat)
  }
  dir.create("graphs", showWarnings=FALSE)
  png("graphs/task10.png")
  plot(1:10000, theta_hats, type="l", ylim=c(-1.5,1.5),
       xlab="k", ylab="theta_hat",
       main="Estimation of non-stationary time series (exp10)")
  lines(1:10000, sin(0.0001 * (1:10000)))
  legend("topright", legend=c("Estimated theta_hat", "True theta"), lty=1)
  dev.off()
}
exp10()
\end{lstlisting}

\paragraph{結果}
図\ref{fig:task10}。推定 $\theta_k^{\text{hat}}$ は真値 $\sin(10^{-4}k)$ を平滑追従。ノイズ $\pm 1$ のため短周期の揺れが残る。

\begin{figure}[H]
  \centering
  \includegraphics[width=.82\linewidth]{graphs/task10.png}
  \caption{忘却係数付きRLSによる非定常平均の追従($\gamma=0.99$)}
  \label{fig:task10}
\end{figure}

\paragraph{考察}
% 追従と平滑化のトレードオフ:$\gamma\downarrow$ で追従性↑ だが分散↑。$\gamma\uparrow$ で分散↓ だが遅延↑。
% 本設定は振幅1に対し雑音振幅1でSNRが低い。RMSEや相関係数で評価するとよい。
% スカラー回帰(φ_k=1)ではRLSは指数移動平均に近い挙動を示す。初期Φを大きくすると初期ゲインが大きく収束が速い。


\section {カルマンフィルタ、カルマンスムーサー}

\subsection{原理と方法}

\paragraph{モデル}
一次元の線形状態空間モデルを用いる:
\[
  \theta_k = a_k\,\theta_{k-1} + v_k,\tag{3.40}
\]
\[
  y_k = c_k\,\theta_k + w_k,\tag{3.41}
\]
ここで $v_k\sim\mathcal N(0,\sigma_v^2)$,$w_k\sim\mathcal N(0,\sigma_w^2)$ は互いに独立で白色,$a_k,c_k$ は既知とする\cite{exp2025}。目的は「時刻 $k$ までの観測 $\{y_\ell\}_{\ell=1}^k$ に基づく $\theta_k$ の最小平均二乗誤差(MMSE)推定」を行うこと\cite{exp2025}。

以下はこのモデルに基づくデータ生成のRによる実装例である。
\begin{lstlisting}[language=R,caption={データ生成関数}]
# データ生成関数(exp11とexp12で共通使用)
generate_kalman_data <- function(seed = 123, N = 100, a_k = 0.9, c_k = 2, 
                                  sigma2_v = 1, sigma2_w = 1) {
  # 乱数シードを設定
  set.seed(seed)
  
  # 初期値
  theta_0 <- rnorm(1, mean = 3, sd = sqrt(2))  # θ_0 ~ N(3, 2)
  V_0 <- 2
  
  # データ格納用
  theta_true <- numeric(N)
  y_obs <- numeric(N)
  
  # 初期化
  theta_k_1 <- theta_0
  
  # データ生成
  for (k in 1:N) {
    # 真の状態遷移 (3.40): θ_k = a_k * θ_{k-1} + v_k
    v_k <- rnorm(1, mean = 0, sd = sqrt(sigma2_v))
    theta_k <- a_k * theta_k_1 + v_k
    theta_true[k] <- theta_k
    
    # 観測 (3.41): y_k = c_k * θ_k + w_k
    w_k <- rnorm(1, mean = 0, sd = sqrt(sigma2_w))
    y_k <- c_k * theta_k + w_k
    y_obs[k] <- y_k
    
    # 次のステップへ
    theta_k_1 <- theta_k
  }
  
  # データを返す
  return(list(
    theta_true = theta_true,
    y_obs = y_obs,
    theta_0 = theta_0,
    V_0 = V_0,
    a_k = a_k,
    c_k = c_k,
    sigma2_v = sigma2_v,
    sigma2_w = sigma2_w,
    N = N
  ))
}
\end{lstlisting}

\paragraph{Kalman フィルタ}
平方完成による最適ゲイン $F_k$ と分散更新を導くと,予測分散 $X_k$,事後分散 $V_k$,推定値 $\hat\theta_k$ は
\[
\hat\theta_k = a_k\hat\theta_{k-1} + F_k\bigl(y_k - c_k a_k \hat\theta_{k-1}\bigr),\tag{3.36}
\]
\[
X_k = a_k^2 V_{k-1} + \sigma_v^2,\tag{3.37}
\]
\[
V_k = \frac{\sigma_w^2\, X_k}{c_k^2 X_k + \sigma_w^2},\tag{3.38}
\]
\[
F_k = \frac{c_k X_k}{c_k^2 X_k + \sigma_w^2}.\tag{3.39}
\]
これが Kalman フィルタであり,推定誤差分散を最小にする\cite{exp2025}。初期条件は $\hat\theta_0=\bar\theta,\;V_0=P$ とする\cite{exp2025}。

以下は、KalmanフィルタのRによる実装例である。

\begin{lstlisting}[language=R,caption={Kalmanフィルタ更新(式(3.36)–(3.39))}]
update_kalman_filter <- function(theta_k_1, V_k_1, y_k, a_k, c_k, sigma2_v, sigma2_w) {
  # (3.37) 予測分散
  X_k <- a_k^2 * V_k_1 + sigma2_v
  # (3.39) カルマンゲイン
  F_k <- (c_k * X_k) / (c_k^2 * X_k + sigma2_w)
  # (3.36) 推定値更新
  theta_k <- a_k * theta_k_1 + F_k * (y_k - c_k * a_k * theta_k_1)
  # (3.38) 事後分散更新
  V_k <- (sigma2_w * X_k) / (c_k^2 * X_k + sigma2_w)
  list(theta_k = theta_k, V_k = V_k, X_k = X_k, F_k = F_k)
}
\end{lstlisting}

\paragraph{Kalman スムーザ}
全データ $\{y_1,\dots,y_N\}$ を用いて各 $\theta_k$ を後処理で精密化する。Rauch–Tung–Striebel(RTS)型の退行更新は
\[
\hat\theta_k^{\,s}=\hat\theta_k + g_k\bigl(\hat\theta_{k+1}^{\,s}-a_k\hat\theta_k\bigr),\tag{3.42}
\]
\[
g_k=\frac{a_k V_k}{X_{k+1}},\tag{3.43}
\]
\[
V_k^{\,s}=V_k + g_k^2\bigl(V_{k+1}^{\,s}-X_{k+1}\bigr),\tag{3.44}
\]
終端条件は $\hat\theta_N^{\,s}=\hat\theta_N,\;V_N^{\,s}=V_N$。$\{\hat\theta_k^{\,s}\}$ は時刻 $N$ までの全情報に基づく MMSE 推定となる\cite{exp2025}。

以下は、KalmanスムーザのRによる実装例である。

\begin{lstlisting}[language=R,caption={Kalmanスムーザ更新(式(3.42)–(3.44))}]
update_kalman_smoother <- function(theta_k, theta_k_plus_1_s, V_k, V_k_plus_1_s, X_k_plus_1, a_k) {
  # (3.43) 後向きゲイン
  g_k <- a_k * V_k / X_k_plus_1
  # (3.42) 推定値の後向き補正
  theta_k_s <- theta_k + g_k * (theta_k_plus_1_s - a_k * theta_k)
  # (3.44) 分散の後向き補正
  V_k_s <- V_k + g_k^2 * (V_k_plus_1_s - X_k_plus_1)
  list(theta_k_s = theta_k_s, V_k_s = V_k_s, g_k = g_k)
}
\end{lstlisting}

\subsection{課題11(カルマンフィルタ:フォワード)}
\label{subsec:task11}

\paragraph{内容}
一次元状態空間モデルを対象とする.
\begin{align}
  \theta_k &= a_k\,\theta_{k-1} + v_{k-1}, \quad v_{k-1}\sim \mathcal{N}(0,\sigma_v^2),\\
  y_k &= c_k\,\theta_k + w_k, \quad w_k\sim \mathcal{N}(0,\sigma_w^2).
\end{align}
初期推定値は$\hat{\theta}_0=\theta_0$, 分散$V_0$を与える.
評価は平均二乗誤差
\begin{align}
  \mathrm{MSE}=\frac{1}{N}\sum_{k=1}^{N}\bigl(\theta_k^{\mathrm{true}}-\hat{\theta}_k\bigr)^2
\end{align}
で行う.

\paragraph{方法}
時刻$k$ごとに予測と更新を行う.
\begin{align}
  X_k &= a_k^2 V_{k-1} + \sigma_v^2,\\
  F_k &= \frac{c_k X_k}{c_k^2 X_k + \sigma_w^2},\\
  \hat{\theta}_k &= a_k \hat{\theta}_{k-1} + F_k\bigl(y_k - c_k a_k \hat{\theta}_{k-1}\bigr),\\
  V_k &= (1 - F_k c_k)\,X_k.
\end{align}
出力は$\{\hat{\theta}_k\}_{k=1}^N$と$\{V_k\}_{k=1}^N$.

\paragraph{実装}
関数\verb|exp11(data)|を用いる.各$k$で\verb|update_kalman_filter|を呼び,$X_k, F_k, \hat{\theta}_k, V_k$を更新する.推定系列と真値を重ね描画し\verb|graphs/task11.png|に保存する.

\paragraph{結果}
データ生成と実行条件は
\begin{lstlisting}
kalman_data <- generate_kalman_data(seed = 42, N = 100,
                                    a_k = 0.9, c_k = 2,
                                    sigma2_v = 1, sigma2_w = 1)
result11 <- exp11(kalman_data)
\end{lstlisting}
である.MSEは$0.1936$となった.図\ref{fig:task11_fig}に真値と推定の系列を示す.

\begin{figure}[H]
  \centering
  \includegraphics[width=0.8\linewidth]{graphs/task11.png}
  \caption{カルマンフィルタ(フォワード)による推定結果(赤:真値,青:推定)}
  \label{fig:task11_fig}
\end{figure}

\paragraph{考察}
% ここに記述


\subsection{課題12(カルマンスムーザー)}
\label{subsec:task12}

\paragraph{内容}
一次元状態空間モデル
\begin{align}
  \theta_k &= a_k\,\theta_{k-1} + v_{k-1},\quad v_{k-1}\sim\mathcal{N}(0,\sigma_v^2),\\
  y_k &= c_k\,\theta_k + w_k,\quad w_k\sim\mathcal{N}(0,\sigma_w^2)
\end{align}
に対し,時刻$1\!:\!N$の全観測を用いて各時刻の状態を後向きに精緻化する.

\paragraph{方法}
まず課題11と同じ前向きフィルタで
\begin{align}
  X_k &= a_k^2 V_{k-1}+\sigma_v^2,\quad
  F_k=\frac{c_k X_k}{c_k^2 X_k+\sigma_w^2},\\
  \hat\theta_k &= a_k\hat\theta_{k-1}+F_k\bigl(y_k-c_k a_k\hat\theta_{k-1}\bigr),\quad
  V_k=(1-F_k c_k)X_k
\end{align}
を得る.後向きではRTSスムーザーを用いる(スカラー).平滑化ゲイン
\begin{align}
  J_k &= \frac{V_k\,a_k}{X_{k+1}},
\end{align}
更新式
\begin{align}
  \hat\theta_k^{\,s} &= \hat\theta_k + J_k\bigl(\hat\theta_{k+1}^{\,s}-a_k\hat\theta_k\bigr),\\
  V_k^{\,s} &= V_k + J_k^2\bigl(V_{k+1}^{\,s}-X_{k+1}\bigr),
\end{align}
終端条件は$\hat\theta_N^{\,s}=\hat\theta_N,\;V_N^{\,s}=V_N$.

\paragraph{実装}
\verb|exp12(data)|を用いる.前向きで\verb|theta_hat, V_all, X_all|を保存し,後向きで$k=N-1$から$1$へ\verb|update_kalman_smoother|を適用して\verb|theta_smooth, V_smooth|を得る.真値と平滑化推定を描画し\verb|graphs/task12.png|に保存する.評価は\verb|mean((theta_true - theta_smooth)^2)|.

\paragraph{結果}
データ生成と実行
\begin{lstlisting}
kalman_data <- generate_kalman_data(seed = 42, N = 100,
                                    a_k = 0.9, c_k = 2,
                                    sigma2_v = 1, sigma2_w = 1)
result12 <- exp12(kalman_data)
\end{lstlisting}
出力は
\verb|mmse: 0.1716| であった.図\ref{fig:task12_fig}に真値と平滑化推定を示す.

\begin{figure}[H]
  \centering
  \includegraphics[width=0.8\linewidth]{graphs/task12.png}
  \caption{カルマンスムーザーの推定結果(赤:真値,緑:平滑化推定)}
  \label{fig:task12_fig}
\end{figure}

\paragraph{考察}
% ここに記述



\section{交互最小二乗法とK-平均法}


\subsection{原理と方法}

\paragraph{交互最小二乗法}
非線形最小二乗問題のうち,パラメータを二分割すると片方ずつは線形になる場合に適用する.
パラメータを $\theta=(\alpha,\beta)^\top$ と分割し,$g(\alpha,\beta,x)$ が $\alpha,\beta$ に関してそれぞれ線形(双線形など)であるとする.
観測 $(x_i,y_i)$ に対し,目的関数
\begin{align}
  J(\alpha,\beta) := \sum_{i=1}^N \|\,y_i - g(\alpha,\beta,x_i)\,\|^2
\end{align}
を最小化する.片方のパラメータを固定すれば通常の線形最小二乗(式(3.5)型)で解ける.
交互更新により $J$ は単調非増加で,反復に伴い悪化しない.ただし一般には局所解に収束しうるため初期値が重要である.


初期値 $\alpha^{(0)},\beta^{(0)}$ を与え,$j=1,2,\ldots$ について次を繰り返す.
\begin{enumerate}
  \item \textbf{Step A($\beta$ 固定)}:$\beta=\beta^{(j-1)}$ として
  \begin{align}
    \alpha^{(j)} := \arg\min_{\alpha}\ \sum_{i=1}^N \|\,y_i - g(\alpha,\beta^{(j-1)},x_i)\,\|^2.
  \end{align}
  これは線形最小二乗で解く.
  \item \textbf{Step B($\alpha$ 固定)}:$\alpha=\alpha^{(j)}$ として
  \begin{align}
    \beta^{(j)} := \arg\min_{\beta}\ \sum_{i=1}^N \|\,y_i - g(\alpha^{(j)},\beta,x_i)\,\|^2.
  \end{align}
  これも線形最小二乗で解く.
\end{enumerate}
収束判定は例えば $\|\alpha^{(j)}-\alpha^{(j-1)}\|^2+\|\beta^{(j)}-\beta^{(j-1)}\|^2<\varepsilon$ などとする.
更新ごとに $J(\alpha^{(j)},\beta^{(j)}) \le J(\alpha^{(j-1)},\beta^{(j-1)})$ が成り立つ性質を利用する.
初期値の多点スタートや正則化の併用が実務上有効である.

以下は、交互最小二乗法のRによる実装例である。

\begin{lstlisting}[language=R,caption={交互最小二乗法更新}]
update_alpha_beta <- function(alpha, beta, x, x_T, y, n) {
    # Step 1: alphaを固定してbetaを推定
    x_alpha <- matrix(0, nrow = n, ncol = 3)
    for (i in 1:n) {
        x_alpha[i, ] <- t(alpha) %*% x[, , i]
    }
    result <- regression_simple(x_alpha, y)
    beta_new <- result$theta_hat
    
    # Step 2: betaを固定してalphaを推定
    x_beta <- matrix(0, nrow = n, ncol = 2)
    for (i in 1:n) {
        x_beta[i, ] <- t(beta_new) %*% x_T[, , i]
    }
    result2 <- regression_simple(x_beta, y)
    alpha_new <- result2$theta_hat
    
    return(list(alpha_new = alpha_new, beta_new = beta_new))
}
\end{lstlisting}




\paragraph{K-平均法}
データ $\{x_i\}_{i=1}^N \subset \mathbb{R}^p$ を $K$ 個のクラスタに分割する.
クラスタ $V_\ell$ の代表点(重心)を
\begin{align}
  \mu(V_\ell) := \arg\min_{\mu\in\mathbb{R}^p} \sum_{i\in V_\ell} \|x_i-\mu\|^2
  = \frac{1}{|V_\ell|}\sum_{i\in V_\ell} x_i
\end{align}
と定めると,目的は
\begin{align}
  \min_{\{V_\ell\}}\ \sum_{\ell=1}^K \sum_{i\in V_\ell} \|x_i-\mu(V_\ell)\|^2
\end{align}
の最小化である.指示変数 $r_{i,\ell}\in\{0,1\}$($x_i$ がクラスタ $\ell$ に属すれば1)を用いると
\begin{align}
  \min_{\{\mu_\ell\},\{r_{i,\ell}\}}\ \sum_{\ell=1}^K \sum_{i=1}^N r_{i,\ell}\,\|x_i-\mu_\ell\|^2
\end{align}
と書ける.行列表示では $X=[x_1,\dots,x_N]$, $U=[\mu_1,\dots,\mu_K]$, $R=(r_{i,\ell})$ として
\begin{align}
  \|X-UR\|_F^2
\end{align}
の最小化に等価である.大域最適性は保証されず初期値依存性が強い.


初期重心 $\{\mu_\ell^{(0)}\}_{\ell=1}^K$ を与え,$j=1,2,\ldots$ について次を交互に実行する.
\begin{enumerate}
  \item \textbf{割当て(Assignment)}:各データを最も近い重心へ割り当てる.
  \begin{align}
    r_{i,\ell}^{(j)}=\begin{cases}
      1 & \text{if } \ell=\arg\min_{k=1,\dots,K}\ \|x_i-\mu_k^{(j-1)}\|^2,\\
      0 & \text{otherwise.}
    \end{cases}
  \end{align}
  \item \textbf{重心更新(Update)}:各クラスタの平均で重心を更新する.
  \begin{align}
    \mu_\ell^{(j)}=\frac{\sum_{i=1}^N r_{i,\ell}^{(j)}\,x_i}{\sum_{i=1}^N r_{i,\ell}^{(j)}}\quad (\sum_{i} r_{i,\ell}^{(j)} > 0).
  \end{align}
\end{enumerate}
停止条件は $\max_\ell \|\mu_\ell^{(j)}-\mu_\ell^{(j-1)}\|^2<\varepsilon$ 等とする.
目的関数は反復で非増加となるが,局所解に収束するため初期化戦略(例:多点スタート,$k$-means++)が重要である.

以下は、K-平均法のRによる実装例である。

\begin{lstlisting}[language=R,caption={K-平均法}]
k_means <- function(x, K, max_iters = 10000, tol = 1e-6) {
    n <- nrow(x)
    d <- ncol(x)
    
    # ランダムに初期クラスタ中心を選択
    centroids <- x[sample(1:n, K), ]
    
    cluster_assignments <- rep(0, n)
    for (iter in 1:max_iters) {
        # 各データポイントを最も近いクラスタ中心に割り当て
        for (i in 1:n) {
            distances <- apply(centroids, 1, function(centroid) sum((x[i, ] - centroid)^2))
            cluster_assignments[i] <- which.min(distances)
        }
        
        # 新しいクラスタ中心を計算
        new_centroids <- matrix(0, nrow = K, ncol = d)
        for (k in 1:K) {
            points_in_cluster <- x[cluster_assignments == k, , drop = FALSE]
            if (nrow(points_in_cluster) > 0) {
                new_centroids[k, ] <- colMeans(points_in_cluster)
            } else {
                new_centroids[k, ] <- centroids[k, ]  # クラスタにポイントがない場合は中心を維持
            }
        }
        
        # 収束判定
        if (max(sqrt(rowSums((new_centroids - centroids)^2))) < tol) {
            # 各データポイントから割り当てられた中心までの距離の合計を計算
            total_distance <- 0
            for (i in 1:n) {
                centroid_idx <- cluster_assignments[i]
                distance <- sum((x[i, ] - centroids[centroid_idx, ])^2)
                total_distance <- total_distance + distance
            }
            break
        }
        
        centroids <- new_centroids
    }
    
    return(list(
        centroids = centroids,
        cluster_assignments = cluster_assignments,
        total_distance = total_distance
    ))
}
\end{lstlisting}

\subsection{課題13(交互最小二乗法)}
\label{subsec:task13}

\paragraph{内容}
パラメータ $\alpha\in\mathbb{R}^{2}$, $\beta\in\mathbb{R}^{3}$を用いた双線形モデル
\[
y_i \approx \alpha^\top X_i \beta
\]
を最小二乗で推定する.特徴は $\phi_i=(1,\,x_i,\,x_i^2,\,x_i^3)$ とし,
\[
X_i=
\begin{pmatrix}
\phi_{i,1} & \phi_{i,2} & \phi_{i,3}\\
\phi_{i,2} & \phi_{i,3} & \phi_{i,4}
\end{pmatrix}
\in\mathbb{R}^{2\times 3}
\]
を用いる.目的は $\min_{\alpha,\beta}\sum_{i=1}^{n}\bigl(y_i-\alpha^\top X_i\beta\bigr)^2$.

\paragraph{方法}
交互最小二乗法(ALS)を用いる.反復 $j=1,2,\dots$ に対し
\begin{align}
\beta^{(j)}&:=\arg\min_{\beta}\sum_{i=1}^{n}\bigl(y_i-(\alpha^{(j-1)})^\top X_i\beta\bigr)^2,\\
\alpha^{(j)}&:=\arg\min_{\alpha}\sum_{i=1}^{n}\bigl(y_i-\alpha^\top X_i\beta^{(j)}\bigr)^2,
\end{align}
を交互に解く(いずれも線形最小二乗).収束判定は
\[
\max\bigl(\|\alpha^{(j)}-\alpha^{(j-1)}\|_2,\ \|\beta^{(j)}-\beta^{(j-1)}\|_2\bigr)<\varepsilon.
\]

\paragraph{実装}
CSV \verb|datas/mmse_kadai13.csv| を読み込み,$X_i$ を \verb|x[,,i]|,$X_i^\top$ を \verb|x_T[,,i]| に格納.
\verb|update_alpha_beta| で
\[
\underbrace{(\alpha^\top X_i)_{i=1:n}}_{\text{$n\times 3$設計行列}}\ \text{から}\ \beta,
\quad
\underbrace{(X_i\beta)_{i=1:n}}_{\text{$n\times 2$設計行列}}\ \text{から}\ \alpha
\]
を逐次推定する.多点初期化(乱数)で \verb|num_trials=10| を走らせ最良解を採用.収束履歴と
$\alpha$ の軌跡を \verb|graphs/task13_convergence.png|,\verb|graphs/task13_trajectory.png| に保存する.

\paragraph{結果}
全試行 $10$ 回の平均二乗誤差(MMSE)は
\[
\mathrm{MMSE}=0.3329
\]
であった(有効数字4桁).最良試行での推定結果は
\[
\alpha=
\begin{pmatrix}
0.6000\\[-2pt]
-1.200
\end{pmatrix},\quad
\beta=
\begin{pmatrix}
0.8441\\[-2pt]
-1.661\\[-2pt]
3.334
\end{pmatrix},
\]
\[
\theta=\alpha\beta^\top=
\left(
\begin{array}{ccc}
\ 0.5065 & -0.9967 & \ 2.000\\
-1.013 & \ 1.994 & -4.002
\end{array}
\right).
\]
収束挙動とパラメータ軌跡を図\ref{fig:task13_conv}および図\ref{fig:task13_traj}に示す.

\begin{figure}[H]
  \centering
  \includegraphics[width=0.78\linewidth]{graphs/task13_convergence.png}
  \caption{ALS による収束履歴(縦軸は $\max(\|\Delta\alpha\|_2,\|\Delta\beta\|_2)$,対数目盛)}
  \label{fig:task13_conv}
\end{figure}

\begin{figure}[H]
  \centering
  \includegraphics[width=0.78\linewidth]{graphs/task13_trajectory.png}
  \caption{$\alpha$ の推定軌跡(緑:初期,赤:収束点)}
  \label{fig:task13_traj}
\end{figure}

全試行の統計情報を表\ref{tab:task13_stats}に示す.

\begin{table}[H]
\centering
\caption{全試行 $10$ 回の統計情報}
\label{tab:task13_stats}
\begin{tabular}{|l|c|}
\hline
試行回数 & 10 \\
\hline
MMSE - 平均 & 0.3329 \\
\hline
MMSE - 標準偏差 & 0.0000 \\
\hline
MMSE - 最小 & 0.3329 \\
\hline
MMSE - 最大 & 0.3329 \\
\hline
\end{tabular}
\end{table}

$\theta=\alpha\beta^\top$ の各要素の範囲を表\ref{tab:task13_theta_range}に示す.

\begin{table}[H]
\centering
\caption{$\theta=\alpha\beta^\top$ の各要素の範囲}
\label{tab:task13_theta_range}
\begin{tabular}{|c|c|c|}
\hline
要素 & 最小値 & 最大値 \\
\hline
$\theta_{1,1}$ & 0.5065 & 0.5065 \\
\hline
$\theta_{1,2}$ & $-0.9967$ & $-0.9967$ \\
\hline
$\theta_{1,3}$ & 2.0003 & 2.0003 \\
\hline
$\theta_{2,1}$ & $-1.0132$ & $-1.0132$ \\
\hline
$\theta_{2,2}$ & 1.9939 & 1.9939 \\
\hline
$\theta_{2,3}$ & $-4.0017$ & $-4.0017$ \\
\hline
\end{tabular}
\end{table}



\paragraph{考察}
図\ref{fig:task13_conv}と、図\ref{fig:task13_traj} より,ALS は数十回の反復で高速に収束することが分かる.

また,表\ref{tab:task13_stats}および表\ref{tab:task13_theta_range}から,全試行にわたってMMSEや$\theta$の各要素が非常に安定していることも分かる.








\section{結論}
本実験では,最小二乗法とその応用手法について学んだ。各手法の理論的背景を理解し,実装を通じてその性能を評価することで,最小二乗法の多様な応用可能性を確認できた。特に,K平均法などの交互最小二乗法などは,工夫次第で様々な問題に適用できることが分かった。

\appendix
\section{使用コード一覧}
主要スクリプトと入手先を列挙する。
\begin{itemize}
  \item 実験コード(R):\url{https://github.com/Issan0511/suuri_exp/blob/aaf8c74b831aa8b3abb17d2e08713248fbf3de89/2/R1.ipynb}
  \item レポート作成、Rの実装の一部にGitHub Copilotの補完を使用
\end{itemize}


% ===== 参考文献 =====
\section*{参考文献}
\begin{thebibliography}{9}
\bibitem{exp2025} 数理工学実験(2025年度配布資料).
\end{thebibliography}

\end{document}
