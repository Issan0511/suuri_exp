

\subsection{交互最小二乗法}
\label{subsec:als}

\subsubsection{原理と方法}

\paragraph{原理}
非線形最小二乗問題のうち,パラメータを二分割すると片方ずつは線形になる場合に適用する.
パラメータを $\theta=(\alpha,\beta)^\top$ と分割し,$g(\alpha,\beta,x)$ が $\alpha,\beta$ に関してそれぞれ線形(双線形など)であるとする.
観測 $(x_i,y_i)$ に対し,目的関数
\begin{align}
  J(\alpha,\beta) := \sum_{i=1}^N \|\,y_i - g(\alpha,\beta,x_i)\,\|^2
\end{align}
を最小化する.片方のパラメータを固定すれば通常の線形最小二乗(式(3.5)型)で解ける.
交互更新により $J$ は単調非増加で,反復に伴い悪化しない.ただし一般には局所解に収束しうるため初期値が重要である.

\paragraph{方法}
初期値 $\alpha^{(0)},\beta^{(0)}$ を与え,$j=1,2,\ldots$ について次を繰り返す.
\begin{enumerate}
  \item \textbf{Step A($\beta$ 固定)}:$\beta=\beta^{(j-1)}$ として
  \begin{align}
    \alpha^{(j)} := \arg\min_{\alpha}\ \sum_{i=1}^N \|\,y_i - g(\alpha,\beta^{(j-1)},x_i)\,\|^2.
  \end{align}
  これは線形最小二乗で解く.
  \item \textbf{Step B($\alpha$ 固定)}:$\alpha=\alpha^{(j)}$ として
  \begin{align}
    \beta^{(j)} := \arg\min_{\beta}\ \sum_{i=1}^N \|\,y_i - g(\alpha^{(j)},\beta,x_i)\,\|^2.
  \end{align}
  これも線形最小二乗で解く.
\end{enumerate}
収束判定は例えば $\|\alpha^{(j)}-\alpha^{(j-1)}\|^2+\|\beta^{(j)}-\beta^{(j-1)}\|^2<\varepsilon$ などとする.
更新ごとに $J(\alpha^{(j)},\beta^{(j)}) \le J(\alpha^{(j-1)},\beta^{(j-1)})$ が成り立つ性質を利用する.
初期値の多点スタートや正則化の併用が実務上有効である.

以下は、交互最小二乗法のRによる実装例である。

\begin{lstlisting}[language=R,caption={交互最小二乗法更新}]
update_alpha_beta <- function(alpha, beta, x, x_T, y, n) {
    # Step 1: alphaを固定してbetaを推定
    x_alpha <- matrix(0, nrow = n, ncol = 3)
    for (i in 1:n) {
        x_alpha[i, ] <- t(alpha) %*% x[, , i]
    }
    result <- regression_simple(x_alpha, y)
    beta_new <- result$theta_hat
    
    # Step 2: betaを固定してalphaを推定
    x_beta <- matrix(0, nrow = n, ncol = 2)
    for (i in 1:n) {
        x_beta[i, ] <- t(beta_new) %*% x_T[, , i]
    }
    result2 <- regression_simple(x_beta, y)
    alpha_new <- result2$theta_hat
    
    return(list(alpha_new = alpha_new, beta_new = beta_new))
}
\end{lstlisting}


\subsection{K-平均法}
\label{subsec:kmeans}

\subsubsection{原理と方法}

\paragraph{原理}
データ $\{x_i\}_{i=1}^N \subset \mathbb{R}^p$ を $K$ 個のクラスタに分割する.
クラスタ $V_\ell$ の代表点(重心)を
\begin{align}
  \mu(V_\ell) := \arg\min_{\mu\in\mathbb{R}^p} \sum_{i\in V_\ell} \|x_i-\mu\|^2
  = \frac{1}{|V_\ell|}\sum_{i\in V_\ell} x_i
\end{align}
と定めると,目的は
\begin{align}
  \min_{\{V_\ell\}}\ \sum_{\ell=1}^K \sum_{i\in V_\ell} \|x_i-\mu(V_\ell)\|^2
\end{align}
の最小化である.指示変数 $r_{i,\ell}\in\{0,1\}$($x_i$ がクラスタ $\ell$ に属すれば1)を用いると
\begin{align}
  \min_{\{\mu_\ell\},\{r_{i,\ell}\}}\ \sum_{\ell=1}^K \sum_{i=1}^N r_{i,\ell}\,\|x_i-\mu_\ell\|^2
\end{align}
と書ける.行列表示では $X=[x_1,\dots,x_N]$, $U=[\mu_1,\dots,\mu_K]$, $R=(r_{i,\ell})$ として
\begin{align}
  \|X-UR\|_F^2
\end{align}
の最小化に等価である.大域最適性は保証されず初期値依存性が強い.

\paragraph{方法}
初期重心 $\{\mu_\ell^{(0)}\}_{\ell=1}^K$ を与え,$j=1,2,\ldots$ について次を交互に実行する.
\begin{enumerate}
  \item \textbf{割当て(Assignment)}:各データを最も近い重心へ割り当てる.
  \begin{align}
    r_{i,\ell}^{(j)}=\begin{cases}
      1 & \text{if } \ell=\arg\min_{k=1,\dots,K}\ \|x_i-\mu_k^{(j-1)}\|^2,\\
      0 & \text{otherwise.}
    \end{cases}
  \end{align}
  \item \textbf{重心更新(Update)}:各クラスタの平均で重心を更新する.
  \begin{align}
    \mu_\ell^{(j)}=\frac{\sum_{i=1}^N r_{i,\ell}^{(j)}\,x_i}{\sum_{i=1}^N r_{i,\ell}^{(j)}}\quad (\sum_{i} r_{i,\ell}^{(j)} > 0).
  \end{align}
\end{enumerate}
停止条件は $\max_\ell \|\mu_\ell^{(j)}-\mu_\ell^{(j-1)}\|^2<\varepsilon$ 等とする.
目的関数は反復で非増加となるが,局所解に収束するため初期化戦略(例:多点スタート,$k$-means++)が重要である.

以下は、K-平均法のRによる実装例である。

\begin{lstlisting}[language=R,caption={K-平均法}]
k_means <- function(x, K, max_iters = 10000, tol = 1e-6) {
    n <- nrow(x)
    d <- ncol(x)
    
    # ランダムに初期クラスタ中心を選択
    centroids <- x[sample(1:n, K), ]
    
    cluster_assignments <- rep(0, n)
    for (iter in 1:max_iters) {
        # 各データポイントを最も近いクラスタ中心に割り当て
        for (i in 1:n) {
            distances <- apply(centroids, 1, function(centroid) sum((x[i, ] - centroid)^2))
            cluster_assignments[i] <- which.min(distances)
        }
        
        # 新しいクラスタ中心を計算
        new_centroids <- matrix(0, nrow = K, ncol = d)
        for (k in 1:K) {
            points_in_cluster <- x[cluster_assignments == k, , drop = FALSE]
            if (nrow(points_in_cluster) > 0) {
                new_centroids[k, ] <- colMeans(points_in_cluster)
            } else {
                new_centroids[k, ] <- centroids[k, ]  # クラスタにポイントがない場合は中心を維持
            }
        }
        
        # 収束判定
        if (max(sqrt(rowSums((new_centroids - centroids)^2))) < tol) {
            # 各データポイントから割り当てられた中心までの距離の合計を計算
            total_distance <- 0
            for (i in 1:n) {
                centroid_idx <- cluster_assignments[i]
                distance <- sum((x[i, ] - centroids[centroid_idx, ])^2)
                total_distance <- total_distance + distance
            }
            break
        }
        
        centroids <- new_centroids
    }
    
    return(list(
        centroids = centroids,
        cluster_assignments = cluster_assignments,
        total_distance = total_distance
    ))
}
\end{lstlisting}

