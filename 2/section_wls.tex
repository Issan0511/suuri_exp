\subsection{3.2.2 原理と方法}

\paragraph{原理}
各観測 $i=1,\dots,N$ で $y_i\in\mathbb{R}^m$, $X_i\in\mathbb{R}^{m\times p}$ とし
\[
y_i = X_i\theta + w_i,\qquad \mathbb{E}[w_i]=0,\ \mathrm{Cov}(w_i)=V\ (\text{既知}), 
\]
を仮定する。$Q:=V^{-1}$ とおく。WLS は重み付き残差平方和
\[
J(\theta)=\sum_{i=1}^N (y_i-X_i\theta)^\top Q\,(y_i-X_i\theta)
\]
を最小化する推定で,正規方程式は
\[
S\hat\theta=b,\qquad 
S:=\sum_{i=1}^N X_i^\top QX_i,\quad 
b:=\sum_{i=1}^N X_i^\top Qy_i.
\]
したがって
\[
\hat\theta=(\sum_{i=1}^N X_i^\top QX_i)^{-1}\Big(\sum_{i=1}^N X_i^\top Qy_i\Big).
\]
$V$ が既知のとき
\[
\mathrm{Cov}(\hat\theta)=S^{-1}.
\]
$V=\sigma^2\Sigma$ のようにスケール未知なら
\[
\hat\sigma^2=\frac{\sum_{i=1}^N r_i^\top \Sigma^{-1} r_i}{Nm-p},\quad r_i:=y_i-X_i\hat\theta,\qquad
\widehat{\mathrm{Cov}}(\hat\theta)=\hat\sigma^2\,(\sum X_i^\top \Sigma^{-1}X_i)^{-1}.
\]

\paragraph{方法}
(1) 各 $i$ の設計行列 $X_i$ と観測 $y_i$ を用意($X_i$ は $m\times p$)。  
(2) $Q=V^{-1}$ を決めて $S=\sum X_i^\top QX_i,\ b=\sum X_i^\top Qy_i$ を計算。  
(3) $\hat\theta=S^{-1}b$。  
(4) 残差 $r_i=y_i-X_i\hat\theta$ を求め,$V$ 既知なら $\widehat{\mathrm{Cov}}(\hat\theta)=S^{-1}$。$V=\sigma^2\Sigma$ のときは上式で $\hat\sigma^2$ を推定し $\hat\sigma^2 S^{-1}$ を用いる。

\paragraph{実装}
R による実装例を以下に示す。
\begin{lstlisting}
# Weighted Least Squares for {y_i = X_i theta + w_i}
# x: array (m, p, n) with X_i = x[,,i]
# y: matrix (n, m) or vector length n*m? -> here matrix with row i = y_i^T
# V: m x m covariance of w_i (known)  ; Q = solve(V)
# If V = sigma^2 * Sigma, pass Sigma via V and set scale_unknown=TRUE.

regression_multiple <- function(x, y, V, scale_unknown = FALSE){
  if (length(dim(x)) != 3) stop("x must be an array (m, p, n)")
  m <- dim(x)[1]; p <- dim(x)[2]; n <- dim(x)[3]
  if (is.null(dim(y))) y <- matrix(y, nrow = n, ncol = m)
  if (!all(dim(y) == c(n, m))) stop("y must be n x m")
  Q <- solve(V)

  S <- matrix(0, p, p)
  b <- matrix(0, p, 1)
  for (i in 1:n){
    Xi <- x[,,i]              # m x p
    yi <- matrix(y[i,], m, 1) # m x 1
    S  <- S + t(Xi) %*% Q %*% Xi
    b  <- b + t(Xi) %*% Q %*% yi
  }
  theta_hat <- solve(S, b)

  # residuals
  SSEw <- 0
  for (i in 1:n){
    Xi <- x[,,i]; yi <- matrix(y[i,], m, 1)
    ri <- yi - Xi %*% theta_hat
    SSEw <- SSEw + t(ri) %*% Q %*% ri
  }
  if (scale_unknown){
    dof <- n*m - p
    sigma2_hat <- as.numeric(SSEw / dof)
    err_cov_mat <- sigma2_hat * solve(S)
  } else {
    err_cov_mat <- solve(S)               # V known exactly
    sigma2_hat <- NA_real_
  }
  list(theta_hat = theta_hat,
       err_cov_mat = err_cov_mat,
       sigma2_hat = sigma2_hat,
       S = S, b = b)
}

# m = 1(スカラー出力)で各点の重み w_i = 1/Var(w_i) がある場合の簡易呼び出し:
# ここでは X_i はベクトル(p×1)なので m=1, V = 1/w_i ではなく Var = 1/w_i ⇒ Q_i = w_i。
# 観測ごとに重みが異なるときは、対角ブロック Q = diag(w_1, ..., w_n) を展開するか、
# 上のループで各 i ごとに Q_i = w_i を使って S += w_i * X_i X_i^T, b += w_i * X_i y_i とすればよい。
\end{lstlisting}

