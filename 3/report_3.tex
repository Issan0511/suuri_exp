\documentclass[a4paper,11pt]{ltjsarticle}
\usepackage{geometry}
\geometry{top=25mm,bottom=25mm,left=25mm,right=25mm}
\usepackage{graphicx}
\usepackage{amsmath,amssymb,bm}
\usepackage{hyperref}
\usepackage{float}
\usepackage{caption}
\usepackage{subcaption}
\usepackage{booktabs}
\usepackage{siunitx}
\usepackage{listings}
\usepackage{xcolor}
\usepackage{physics}
\usepackage[protrusion=false,expansion=false]{microtype}

\hypersetup{
  pdftitle={数理工学実験レポート},
  pdfauthor={中塚一瑳},
  colorlinks=true,
  linkcolor=blue,
  citecolor=blue,
  urlcolor=blue
}

% -------------------------
% ユーザ定義マクロ
% -------------------------
\newcommand{\StudentID}{1029366161}
\newcommand{\AuthorName}{中塚一瑳}
\newcommand{\CourseName}{数理工学実験}
\newcommand{\ChapterTitle}{第4章(常微分方程式の数値的解法)}
\newcommand{\ExperimentDate}{\today}

% 図・表フォルダパス(必要なら変更)
\newcommand{\graphdir}{graphs}

% 数式用の簡易マクロ
\newcommand{\R}{\mathbb{R}}
% \newcommand{\norm}[1]{\left\lVert#1\right\rVert}  % physics パッケージで既に定義済み

% listings 設定(python を例示)
\lstset{
  basicstyle=\ttfamily\footnotesize,
  breaklines=true,
  frame=single,
  numbers=left,
  numberstyle=\tiny,
  language= Python,
  keywordstyle=\color{blue},
  commentstyle=\color{gray},
  stringstyle=\color{teal},
  showstringspaces=false
}



% キャプションのフォント調整
\captionsetup{font=small,labelfont=bf}

% -------------------------
% 本文
% -------------------------
\title{数理工学実験レポート\\[4pt]\large \ChapterTitle}
\author{学籍番号 1029366161 \quad 中塚一瑳}
\date{\ExperimentDate}

\begin{document}

\maketitle
\begin{abstract}
% 要旨をここに記述
\end{abstract}

\tableofcontents
\clearpage

% \section*{表記・略記}
% 本レポートにおける表記と略記を示す。必要に応じて編集すること。
% \begin{itemize}
%   \item $\R$:実数全体
%   \item $\|\cdot\|$:ユークリッドノルム(明示が必要な場合は添字を用いる)
%   \item 各種パラメータや記号は本文中で定義する
% \end{itemize}

% ===== 例:実験テンプレート =====
\section{はじめに}


\section{課題1 身の回りの現象の定式化}
\subsection{問題設定}
1次元拡散方程式
\begin{equation}
  \frac{\partial u}{\partial t} = \frac{\partial^2 u}{\partial x^2}
\end{equation}
を、区間 $[0,L]$($L=10$)上で数値的に解く。初期条件は
\begin{equation}
  u_0(x) = \frac{1}{\sqrt{2\pi}} \exp\left( -\frac{1}{2}(x-5)^2 \right)
\end{equation}
とし、Dirichlet境界条件($u(0,t)=u(L,t)=0$)およびNeumann境界条件($\partial u/\partial x |_{x=0} = \partial u/\partial x |_{x=L} = 0$)の両方について計算を行った。

\subsection{数値解法}
\subsubsection{オイラー陽解法}
時間微分を前進差分、空間2階微分を中心差分で近似すると、オイラー陽解法が得られる:
\begin{equation}
  u_j^{n+1} = u_j^n + c(u_{j-1}^n - 2u_j^n + u_{j+1}^n), \quad c = \frac{\Delta t}{(\Delta x)^2}
\end{equation}
この解法は $c \leq 1/2$ のとき安定である。

\subsubsection{クランク-ニコルソン法}
時間微分を中心差分、空間2階微分を $n$ ステップと $n+1$ ステップの平均で近似すると、クランク-ニコルソン法が得られる:
\begin{equation}
  -\frac{c}{2}u_{j-1}^{n+1} + (1+c)u_j^{n+1} - \frac{c}{2}u_{j+1}^{n+1} = \frac{c}{2}u_{j-1}^n + (1-c)u_j^n + \frac{c}{2}u_{j+1}^n
\end{equation}
この解法は無条件安定であり、各ステップで三重対角行列の連立方程式を解く必要がある。

\subsection{計算結果}
計算パラメータは $N=100$, $\Delta x = 0.1$, $\Delta t = 0.004$($c=0.4$)とした。

\subsubsection{Dirichlet境界条件}
図\ref{fig:dirichlet}にDirichlet境界条件における計算結果を示す。オイラー陽解法とクランク-ニコルソン法の結果はほぼ一致しており、初期のガウス分布が時間とともに拡散し、境界条件により最終的に消滅していく様子が観察できる。

\begin{figure}[H]
  \centering
  \begin{subfigure}{0.45\textwidth}
    \includegraphics[width=\textwidth]{graphs_dirichlet/comparison_t1.png}
    \caption{$t=1$}
  \end{subfigure}
  \begin{subfigure}{0.45\textwidth}
    \includegraphics[width=\textwidth]{graphs_dirichlet/comparison_t2.png}
    \caption{$t=2$}
  \end{subfigure}
  \begin{subfigure}{0.45\textwidth}
    \includegraphics[width=\textwidth]{graphs_dirichlet/comparison_t3.png}
    \caption{$t=3$}
  \end{subfigure}
  \begin{subfigure}{0.45\textwidth}
    \includegraphics[width=\textwidth]{graphs_dirichlet/comparison_t5.png}
    \caption{$t=5$}
  \end{subfigure}
  \caption{Dirichlet境界条件における拡散方程式の数値解}
  \label{fig:dirichlet}
\end{figure}

\subsubsection{Neumann境界条件}
図\ref{fig:neumann}にNeumann境界条件における計算結果を示す。Neumann境界条件では流束がゼロであるため、初期のガウス分布が境界で反射され、最終的には一様分布に近づいていく。

\begin{figure}[H]
  \centering
  \begin{subfigure}{0.45\textwidth}
    \includegraphics[width=\textwidth]{graphs_neumann/comparison_t1.png}
    \caption{$t=1$}
  \end{subfigure}
  \begin{subfigure}{0.45\textwidth}
    \includegraphics[width=\textwidth]{graphs_neumann/comparison_t2.png}
    \caption{$t=2$}
  \end{subfigure}
  \begin{subfigure}{0.45\textwidth}
    \includegraphics[width=\textwidth]{graphs_neumann/comparison_t3.png}
    \caption{$t=3$}
  \end{subfigure}
  \begin{subfigure}{0.45\textwidth}
    \includegraphics[width=\textwidth]{graphs_neumann/comparison_t5.png}
    \caption{$t=5$}
  \end{subfigure}
  \caption{Neumann境界条件における拡散方程式の数値解}
  \label{fig:neumann}
\end{figure}

\subsection{考察}
オイラー陽解法とクランク-ニコルソン法の結果を比較すると、同じパラメータでほぼ同一の解が得られた。これは、使用した $c=0.4 < 0.5$ がオイラー陽解法の安定条件を満たしており、十分小さな時間刻みを使用していることによる。クランク-ニコルソン法は無条件安定であるため、より大きな時間刻みを使用できるという利点がある。

境界条件の違いについては、Dirichlet境界条件では境界で$u=0$が課されるため、物質が境界から流出し最終的に消滅する。一方、Neumann境界条件では境界での流束がゼロであるため、物質は保存され、最終的に一様分布に近づく。

\section{課題2 4次のアダムス・バッシュフォース法とアダムス・ムルトン法の導出}
\subsection{問題設定}
Fisher方程式
\begin{equation}
  \frac{\partial u}{\partial t} = f(u) + \frac{\partial^2 u}{\partial x^2}, \quad f(u) = u(1-u)
\end{equation}
を、区間 $[0,L]$($L=40$)上で解く。境界条件は $u(0,t)=1$, $u(L,t)=0$ とし、初期条件は
\begin{equation}
  u_0(x) = \frac{1}{(1 + e^{bx-5})^2}
\end{equation}
とした。パラメータ $b$ を変化させて進行波解の速度を調べた。\cite{exp2025}

\subsection{数値解法}
反応項を含むオイラー陽解法を用いた:\cite{exp2025}
\begin{equation}
  u_j^{n+1} = u_j^n + \Delta t \cdot f(u_j^n) + c(u_{j-1}^n - 2u_j^n + u_{j+1}^n)
\end{equation}

\subsection{計算結果}
計算パラメータは $\Delta x = 0.1$, $\Delta t = 0.004$ とし、$b = 0.25, 0.5, 1.0$ の3ケースについて計算を行った。

図\ref{fig:fisher}に各ケースの計算結果を示す。いずれのケースでも、初期条件の階段状の分布が時間とともに右方向に進行する進行波解に収束していく様子が観察できる。

\begin{figure}[H]
  \centering
  \begin{subfigure}{0.32\textwidth}
    \includegraphics[width=\textwidth]{graphs_fisher/fisher_b0_25.png}
    \caption{$b=0.25$}
  \end{subfigure}
  \begin{subfigure}{0.32\textwidth}
    \includegraphics[width=\textwidth]{graphs_fisher/fisher_b0_5.png}
    \caption{$b=0.5$}
  \end{subfigure}
  \begin{subfigure}{0.32\textwidth}
    \includegraphics[width=\textwidth]{graphs_fisher/fisher_b1_0.png}
    \caption{$b=1.0$}
  \end{subfigure}
  \caption{Fisher方程式の数値解(異なる$b$での進行波)}
  \label{fig:fisher}
\end{figure}

図\ref{fig:fisher_time}に各時刻での計算結果を示す。時刻ごとに異なる$b$の値での解の分布を比較することで、$b$が大きいほど初期の遷移層が急峻になり、波の進行速度が遅くなることが観察できる。

\begin{figure}[H]
  \centering
  \begin{subfigure}{0.32\textwidth}
    \includegraphics[width=\textwidth]{graphs_fisher/fisher_t0.png}
    \caption{$t=0$}
  \end{subfigure}
  \begin{subfigure}{0.32\textwidth}
    \includegraphics[width=\textwidth]{graphs_fisher/fisher_t10.png}
    \caption{$t=10$}
  \end{subfigure}
  \begin{subfigure}{0.32\textwidth}
    \includegraphics[width=\textwidth]{graphs_fisher/fisher_t20.png}
    \caption{$t=20$}
  \end{subfigure}
  
  \begin{subfigure}{0.32\textwidth}
    \includegraphics[width=\textwidth]{graphs_fisher/fisher_t30.png}
    \caption{$t=30$}
  \end{subfigure}
  \begin{subfigure}{0.32\textwidth}
    \includegraphics[width=\textwidth]{graphs_fisher/fisher_t40.png}
    \caption{$t=40$}
  \end{subfigure}
  \caption{Fisher方程式の数値解(各時刻での異なる$b$の比較)}
  \label{fig:fisher_time}
\end{figure}

\subsection{考察}

\subsubsection{初期条件の数学的意味}

初期条件として用いた関数
\begin{equation}
  u_0(x) = \frac{1}{(1 + e^{bx-5})^2}
\end{equation}
が進行波解となる理由について考察する。

\subsubsection{拡散と反応のバランス}

Fisher方程式は2つの現象の競合を記述している:
\begin{enumerate}
  \item \textbf{拡散}($\partial^2 u/\partial x^2$):物質が濃いところから薄いところへ染み出し、波形をなだらかにしようとする。
  \item \textbf{反応}($f(u) = u(1-u)$):物質が増殖し、波形を急激に立ち上げようとする。
\end{enumerate}

以下では「なだらかにする力」と「立ち上げる力」が釣り合い、形を変えずに一定速度で進む進行波となる条件を満たす初期状態が存在するかどうかを考察する。

\subsubsection{数学的導出:進行波解の条件}

波が速度 $c$ で進むと仮定し、進行波座標 $\xi = x - ct$ を導入する。関数を $U(\xi)$ とすると、Fisher方程式は以下の常微分方程式になる:
\begin{equation}
  -cU' = U(1-U) + U''
\end{equation}

今回の初期条件である以下の関数を代入する:
\begin{equation}
  U(\xi) = \frac{1}{(1 + e^{b\xi})^2}
\end{equation}

計算を簡略化するため $w = e^{b\xi}$ と置くと、$U = (1+w)^{-2}$ と書ける。

\paragraph{導関数の計算}

1階微分:
\begin{equation}
  U' = \frac{dU}{d\xi} = -2(1+w)^{-3} \cdot bw = \frac{-2bw}{(1+w)^3}
\end{equation}

2階微分:
\begin{equation}
  U'' = \frac{-2b^2w(1-2w)}{(1+w)^4}
\end{equation}

\paragraph{方程式への代入}

元の方程式に代入し、分母を $(1+w)^4$ に揃えて整理すると、分子の合計は:
\begin{equation}
  2bcw(1+w) + (1+w)^2 w - w^2 (1+w)^2 - 2b^2 w + 4b^2 w^2
\end{equation}

$w$ の次数ごとに整理すると:
\begin{itemize}
  \item 定数項($w^0$):$0$(常に成立)
  \item $w^1$ の係数:$2bc + 1 - 2b^2 = 0$
  \item $w^2$ の係数:$2bc + 4b^2 - 1 = 0$
\end{itemize}

\paragraph{パラメータの決定}

連立方程式を解く。$w^1$ の式より:
\begin{equation}
  c = \frac{2b^2 - 1}{2b}
\end{equation}

$w^2$ の式に代入して整理すると:
\begin{equation}
  6b^2 = 1 \quad \Rightarrow \quad b = \frac{1}{\sqrt{6}}
\end{equation}

これを用いて速度を計算すると:
\begin{equation}
  c = \frac{2 \cdot \frac{1}{6} - 1}{2 \cdot \frac{1}{\sqrt{6}}} = \frac{-\frac{2}{3}}{\frac{2}{\sqrt{6}}} = \frac{-\sqrt{6}}{3} = -\frac{5}{\sqrt{6}}
\end{equation}

符号を考慮すると、波は速度 $|c| = 5/\sqrt{6} \approx 2.04$ で進行する。

\subsubsection{シミュレーション結果との対応}

以上より、$b = 1/\sqrt{6} \approx 0.408$ のときに理論的な進行波解となる。本シミュレーションでは $b = 0.25, 0.5, 1.0$ の3ケースを計算したが、$b = 0.5$ が理論値に最も近い。

シミュレーション結果(図\ref{fig:fisher})を見ると、$b = 0.5$ のケースのみ初期状態の波形を大きく崩さずに進行しており、確かに理論値に近い進行波解となっていることが確認できる。また、他の$b$のケースも$t=0$時点では波形が異なっていたものの、$t=40$時点においては$b = 0.5$と似た形状に収束していることがわかる。これは、Fisher方程式の進行波解が特定の形状(例えば$b = 1/\sqrt{6}$の進行波)に安定収束する性質を示していると考えられる。
\section{課題3 指数関数の微分方程式の数値解法}

\subsection{原理・方法}

本課題では,常微分方程式の初期値問題
\begin{align}
  u'(t) = u(t), \qquad u(0) = 1
  \label{eq:kadai3_ivp}
\end{align}
を区間 $t \in [0,1]$ で数値的に解き,いくつかの代表的な数値解法の
\emph{収束次数}を,誤差の振る舞いから確認する。

式 \eqref{eq:kadai3_ivp} の解析解は
\begin{align}
  u_{\mathrm{exact}}(t) = \exp(t)
\end{align}
であり,特に $t=1$ における真値は $u_{\mathrm{exact}}(1)=\mathrm{e}$ である。

\subsubsection*{離散化と誤差の定義}

時間刻み幅 $\Delta t$ を
\begin{align}
  \Delta t = \frac{1}{2^i}, \qquad i = 1,2,\dots,i_{\max}
\end{align}
とし,ステップ数を $N = 2^i$,離散化した時刻を
\begin{align}
  t_n = n \Delta t, \qquad n = 0,1,\dots,N
\end{align}
とおく。数値解を $u_n \approx u(t_n)$ と表す。

各 $i$ に対し,$t=1$ に対応する $n=N$ での数値解 $u_N^{(i)}$ を求め,  
絶対誤差
\begin{align}
  E^{(i)} = \left| u_N^{(i)} - u_{\mathrm{exact}}(1) \right|
  = \left| u_N^{(i)} - \mathrm{e} \right|
\end{align}
を定義する。

さらに,連続する刻み幅に対する誤差比
\begin{align}
  E_r^{(i)} = \frac{E^{(i)}}{E^{(i-1)}} \qquad (i \geq 2)
\end{align}
を考える。$p$ 次精度の手法では,$\Delta t$ を半分にすると誤差が理論的に
\begin{align}
  E^{(i)} \approx C (\Delta t)^p = C 2^{-ip}
\end{align}
と振る舞うので,
\begin{align}
  E_r^{(i)} \approx \frac{C 2^{-ip}}{C 2^{-(i-1)p}} = 2^{-p}
\end{align}
が期待される。したがって,数値的に $E_r^{(i)}$ が $2^{-p}$ に近づくかどうかを調べることで,
各手法の収束次数 $p$ を検証できる。

\subsubsection*{対象とする数値解法}

本課題では,次の 5 種類の手法を比較する。

\paragraph{(1) 前進オイラー法 (Forward Euler, FE)}

式 \eqref{eq:kadai3_ivp} に対する前進オイラー法は
\begin{align}
  u_{n+1}
  &= u_n + \Delta t\, f(t_n,u_n), \\
  f(t,u) &= u
\end{align}
で与えられる。本問題では $f(t_n,u_n)=u_n$ であるから,
\begin{align}
  u_{n+1} = (1+\Delta t)\, u_n
\end{align}
となる。これは 1 次精度の陽的一段法である。

\paragraph{(2) 2 次アダムス・バッシュフォース法 (AB2)}

2 次の Adams–Bashforth 法は陽的二段法であり,
\begin{align}
  u_{n+1}
  = u_n + \Delta t
  \left(
    \frac{3}{2} f_n - \frac{1}{2} f_{n-1}
  \right),
  \qquad f_n = f(t_n,u_n)
\end{align}
と表される。本問題では $f_n = u_n$ である。

多段法の性質上,$u_0,u_1$ の 2 点が初期値として必要になる。
本レポートでは,$u_0$ に加えて $u_1 = u_{\mathrm{exact}}(t_1)$ を
解析解から与えている。

\paragraph{(3) 3 次アダムス・バッシュフォース法 (AB3)}

3 次の Adams–Bashforth 法は陽的三段法であり,
\begin{align}
  u_{n+1}
  = u_n + \Delta t
  \left(
    \frac{23}{12} f_n
    - \frac{16}{12} f_{n-1}
    + \frac{5}{12}  f_{n-2}
  \right)
\end{align}
で与えられる。本問題では $f_n=u_n$ である。

三段法のため,$u_0,u_1,u_2$ の 3 点が初期値として必要になる。
本レポートでは,$u_1,u_2$ を
\begin{align}
  u_1 = u_{\mathrm{exact}}(t_1),\qquad
  u_2 = u_{\mathrm{exact}}(t_2)
\end{align}
と解析解から与えている。

\paragraph{(4) ホイン法 (Heun 法, 2 次ルンゲ・クッタ法)}

ホイン法は 2 次精度の陽的一段法であり,予測子・修正子の形で
\begin{align}
  &\text{予測:} && u^{\ast}_{n+1} = u_n + \Delta t\, f(t_n,u_n), \\
  &\text{修正:} && u_{n+1}
    = u_n + \frac{\Delta t}{2}
      \bigl( f(t_n,u_n) + f(t_{n+1},u^{\ast}_{n+1}) \bigr)
\end{align}
と書ける。本問題では $f(t,u)=u$ である。

\paragraph{(5) 4 次ルンゲ・クッタ法 (RK4)}

4 次のルンゲ・クッタ法は
\begin{align}
  k_1 &= f(t_n, u_n), \\
  k_2 &= f\left(t_n + \frac{\Delta t}{2},\, u_n + \frac{\Delta t}{2} k_1\right), \\
  k_3 &= f\left(t_n + \frac{\Delta t}{2},\, u_n + \frac{\Delta t}{2} k_2\right), \\
  k_4 &= f\left(t_n + \Delta t,\, u_n + \Delta t\, k_3\right),
\end{align}
\begin{align}
  u_{n+1}
  = u_n + \frac{\Delta t}{6}
    \left( k_1 + 2k_2 + 2k_3 + k_4 \right)
\end{align}
と定義される。本問題では $f(t,u)=u$ であり,4 次精度の陽的一段法となる。

\subsubsection*{数値実験の手順}

式
\[
  u'(t) = u(t),\quad u(0)=1
\]
を区間 $[0,1]$ で数値的に解き,$t=1$ における解の誤差を比較する。

時間刻みは
\[
  n = 2^i,\quad i = 2,3,\dots,9,\qquad
  \Delta t = \frac{1}{n}
\]
とし,$t_0=0,\,u_0=1$ から $n$ ステップ進めて $t_n=1$ の値 $u_n$ を求める。

実装では,以下の 5 種類の手法を用いた。

\begin{itemize}
  \item 前進オイラー法(Forward Euler)\\
        更新式
        \[
          u_{k+1} = u_k + \Delta t\, f(t_k,u_k),\quad f(t,u)=u
        \]
        を $k=0,\dots,n-1$ について適用する関数
        \texttt{foward\_euler} を用いた。

  \item 2次アダムス・バッシュフォース法(AB2)\\
        更新式
        \[
          u_{k+1}
          = u_k + \Delta t\left(
            \frac{3}{2} f_k - \frac{1}{2} f_{k-1}
          \right),\quad f_k=f(t_k,u_k)
        \]
        を用いる関数 \texttt{adam\_bashforth2} を実装した。
        多段法の初期化のため,$u_0=1$ に加えて
        \[
          u_1 = u_{\mathrm{exact}}(t_1) = \exp(\Delta t)
        \]
        を解析解から与え,$k=1,\dots,n-1$ で上式を適用した。

  \item 3次アダムス・バッシュフォース法(AB3)\\
        更新式
        \[
          u_{k+1}
          = u_k + \Delta t\left(
            \frac{23}{12} f_k
            - \frac{16}{12} f_{k-1}
            + \frac{5}{12}  f_{k-2}
          \right)
        \]
        を用いる関数 \texttt{adam\_bashforth3} を実装した。
        初期化のため $u_0=1$ に加え
        \[
          u_1 = u_{\mathrm{exact}}(t_1),\quad
          u_2 = u_{\mathrm{exact}}(t_2)
        \]
        を解析解から与え,$k=2,\dots,n-1$ で上式を適用した。

  \item ホイン法(Heun 法,2次ルンゲ・クッタ法)\\
        関数 \texttt{heun} で
        \begin{align}
          &\text{予測: } &&
            u^{\ast}_{k+1} = u_k + \Delta t\, f(t_k,u_k),\\
          &\text{修正: } &&
            u_{k+1}
              = u_k + \frac{\Delta t}{2}
              \bigl( f(t_k,u_k) + f(t_{k+1},u^{\ast}_{k+1}) \bigr)
        \end{align}
        を $k=0,\dots,n-1$ について適用した。

  \item 4次ルンゲ・クッタ法(RK4)\\
        関数 \texttt{runge\_kutta4} で
        \begin{align}
          k_1 &= f(t_k, u_k),\\
          k_2 &= f\!\left(t_k + \frac{\Delta t}{2},\, u_k + \frac{\Delta t}{2} k_1\right),\\
          k_3 &= f\!\left(t_k + \frac{\Delta t}{2},\, u_k + \frac{\Delta t}{2} k_2\right),\\
          k_4 &= f\!\left(t_k + \Delta t,\, u_k + \Delta t\, k_3\right),
        \end{align}
        \[
          u_{k+1}
            = u_k + \frac{\Delta t}{6}
              (k_1 + 2k_2 + 2k_3 + k_4)
        \]
        を $k=0,\dots,n-1$ について適用した。
\end{itemize}

各 $n$ に対して,$t=1$ における真値 $u_{\mathrm{exact}}(1)=\mathrm{e}$ との差
\[
  E(n) = \bigl|u_n - \mathrm{e}\bigr|
\]
を計算し,誤差の大きさと減少の仕方を比較した。
さらに,誤差比
\[
  \frac{E(n)}{E(n/2)}
\]
を用いて,各手法の収束次数を推定した。

\subsection{結果}

$n=4,8,\dots,512$ に対して得られた $t=1$ における誤差 $E(n)$ を
表\ref{tab:kadai3_error} に示す(有効数字 4 桁)。

\begin{table}[htbp]
  \centering
  \caption{$t=1$ における絶対誤差 $E(n)=|u_n-\mathrm{e}|$}
  \begin{tabular}{c|ccccc}
    \hline
    $n$
      & 前進オイラー
      & AB2
      & AB3
      & ホイン
      & RK4 \\
    \hline
     4  & 0.2769 & 0.04240        & 0.005826        & 0.02343        & $7.189\times 10^{-5}$ \\
     8  & 0.1525 & 0.01407        & 0.001300        & 0.006441       & $4.984\times 10^{-6}$ \\
    16  & 0.08035& 0.003973       & $2.035\times 10^{-4}$ & 0.001688 & $3.281\times 10^{-7}$ \\
    32  & 0.04129& 0.001050       & $2.820\times 10^{-5}$ & $4.322\times 10^{-4}$ & $2.105\times 10^{-8}$ \\
    64  & 0.02094& $2.696\times 10^{-4}$ & $3.704\times 10^{-6}$ & $1.093\times 10^{-4}$ & $1.333\times 10^{-9}$ \\
   128  & 0.01054& $6.826\times 10^{-5}$ & $4.745\times 10^{-7}$ & $2.749\times 10^{-5}$ & $8.384\times 10^{-11}$ \\
   256  & 0.005290& $1.717\times 10^{-5}$& $6.003\times 10^{-8}$ & $6.893\times 10^{-6}$ & $5.261\times 10^{-12}$ \\
   512  & 0.002650& $4.307\times 10^{-6}$& $7.549\times 10^{-9}$ & $1.726\times 10^{-6}$ & $3.286\times 10^{-13}$ \\
    \hline
  \end{tabular}
  \label{tab:kadai3_error}
\end{table}

また,誤差比 $E(n)/E(n/2)$ の典型的な値として,
$n=256$ と $512$ の組に対しては
\[
\begin{array}{ll}
  \text{前進オイラー} & E(512)/E(256) \approx 0.5009,\\
  \text{AB2}           & E(512)/E(256) \approx 0.2508,\\
  \text{AB3}           & E(512)/E(256) \approx 0.1258,\\
  \text{ホイン}        & E(512)/E(256) \approx 0.2504,\\
  \text{RK4}           & E(512)/E(256) \approx 0.0625
\end{array}
\]
が得られた。

\subsection{考察}

表\ref{tab:kadai3_error} から,刻み数 $n$ を 2 倍にすると各手法の誤差は
ほぼ一定の比率で減少していることがわかる。
誤差比 $E(n)/E(n/2)$ の極限値は
\[
  2^{-p}
\]
となることから,$p$ 次精度の手法では
\[
  E(n)/E(n/2) \to 2^{-p}
\]
が期待される。本数値結果では,$n$ が大きい領域で

\begin{itemize}
  \item 前進オイラー法:$E(n)/E(n/2)\to 0.5 \approx 2^{-1}$,
  \item AB2,ホイン法:$E(n)/E(n/2)\to 0.25 \approx 2^{-2}$,
  \item AB3:$E(n)/E(n/2)\to 0.125 \approx 2^{-3}$,
  \item RK4:$E(n)/E(n/2)\to 0.0625 = 2^{-4}$
\end{itemize}
となっており,それぞれ 1 次,2 次,3 次,4 次精度であるという
理論的な収束次数と整合的な結果になっている。

$n=512$ での誤差の大きさを比較すると,
前進オイラー法は $E \approx 2.65\times 10^{-3}$ であるのに対し,
AB2,ホイン法はそれぞれ $10^{-6}$ オーダー,
AB3 は $10^{-9}$ オーダー,
RK4 は $10^{-13}$ オーダーまで誤差が減少している。
同じ刻み幅でも,高次の手法ほど誤差が急速に小さくなることが確認できる。

また,多段法である AB2,AB3 は開始ステップで解析解を用いて初期値を与えているため,
この問題設定では一段法よりも有利な初期条件になっている。
それにもかかわらず,AB2 とホイン法(ともに 2 次精度)の誤差比は
ほぼ同じ値に収束しており,「次数が同じであれば刻み幅に対する誤差の減少率も同程度」
であることが数値的に確かめられる。
RK4 は計算あたりの評価回数が多い一方で,
同じ $n$ に対して最も小さい誤差を与えており,
高次の一段法を用いることにより,
粗い刻みでも高精度な解が得られることがわかる。



\section{課題4 各手法の安定性と安定性な領域の確認}
\subsection{原理と方法}

本節では,二次元トーラス上の強磁性イジング模型
\[
H(s)= -J \sum_{\langle i,j\rangle} s_i s_j , \qquad s_i\in\{+1,-1\},\ J=1
\]
に対し,マルコフ連鎖モンテカルロ法(MCMC)を用いて平衡状態を生成する手法について述べる\cite{exp2025}.
格子は $L\times L$ の正方格子とし,周期境界条件(トーラス)を課す.
本問では特に,熱浴法(Heat Bath 法)とメトロポリス法(Metropolis 法)を比較し,
初期状態依存性が消失する様子を磁化の時間発展から観察する。

%%%%%%%%%%%%%%%%%%%%%%%%%%%%%%%%%%%%%%%%%%%%%%%%%%%%%%%%%%%%%%
\subsubsection{イジングモデルの熱浴法(小問1)}

熱浴法は,選んだ格子点 $i$ のスピン $s_i$ を,
その条件付き分布 $P(s_i\mid s_{\setminus i})$ から
\emph{直接サンプリング}する更新法である\cite{exp2025}。
スピン $i$ の近傍スピン(4 近傍)を $s_j$ とし,
その和を
\[
h_i = \sum_{j\in\mathrm{nn}(i)} s_j
\]
と書くと,
ハミルトニアンからエネルギーは
\[
E(s_i) = - J s_i h_i
\]
であるから,ボルツマン分布より
\[
P(s_i=+1)
\propto e^{\beta h_i},\qquad
P(s_i=-1)\propto e^{-\beta h_i}.
\]
よって正規化すれば
\begin{align}
P(s_i = +1 \mid s_{\setminus i})
= \frac{1}{1+e^{-2\beta h_i}}, \qquad
P(s_i=-1)=1-P(s_i=+1). 
\end{align}

以上より,熱浴法の 1 ステップは以下のように与えられる。

\begin{enumerate}
  \item 格子点 $i$ をランダムに 1 つ選ぶ。
  \item 近傍和 $h_i=\sum_{j}s_j$ を計算する。
  \item $P(s_i=+1)=1/(1+e^{-2\beta h_i})$ を求める。
  \item 一様乱数 $r\in[0,1]$ を生成し,
    \[
    r < P(s_i=+1)\ \Rightarrow\ s_i\leftarrow +1,
    \quad
    r\ge P(s_i=+1)\ \Rightarrow\ s_i\leftarrow -1.
    \]
\end{enumerate}

条件付き分布からの厳密サンプリングに基づくため,自動的に平衡分布を保つマルコフ連鎖が得られる。

%%%%%%%%%%%%%%%%%%%%%%%%%%%%%%%%%%%%%%%%%%%%%%%%%%%%%%%%%%%%%%
\subsubsection{メトロポリス法}

メトロポリス法では,スピン反転 $s_i\to -s_i$ を更新案とし,
対応するエネルギー変化
\[
\Delta E = 2 s_i h_i
\]
を用いて採択判定を行う\cite{exp2025}。
採択確率は
\[
P_{\mathrm{acc}}
= \min\{1, e^{-\beta\Delta E}\}
\]
であり,$\Delta E\le 0$(エネルギーが下がる場合)は必ず採択され,
$\Delta E>0$ の場合には確率的に採択される。

%%%%%%%%%%%%%%%%%%%%%%%%%%%%%%%%%%%%%%%%%%%%%%%%%%%%%%%%%%%%%%
\subsection{実装上の工夫}

周期境界条件の最近接スピンは NumPy の {\tt np.roll} を用いて
\[
\text{上, 下, 左, 右}
\]
の方向に周期的に配列をシフトすることで効率的に求めることができる。
また,格子をチェッカーボード状に黒白に分割し,
黒マス→白マスの順に更新することで,
隣接点を同時更新することによる干渉を防ぎつつ,
ベクトル化された高速な一括更新が可能となる。

\subsubsection{Pythonによる実装例}

以下では NumPy による簡潔な実装例を示す。

\paragraph{熱浴法:}
\begin{verbatim}
def sweep_heatbath(spins, beta, rng, itrations=1500):
    ms = []
    for _ in range(itrations):
        for mask in (mask_black, mask_white):

            n  = np.roll(spins,  1, axis=0)
            n += np.roll(spins, -1, axis=0)
            n += np.roll(spins,  1, axis=1)
            n += np.roll(spins, -1, axis=1)

            p = 1.0 / (1.0 + np.exp(-2 * beta * n))
            r = rng.random((L, L))

            spins[mask] = np.where(r[mask] < p[mask], 1, -1)

        ms.append(spins.mean())
    return np.array(ms)
\end{verbatim}

\paragraph{メトロポリス法:}
\begin{verbatim}
def sweep_metropolis(spins, beta, rng, itrations=1500):
    ms = []
    for _ in range(itrations):
        for mask in (mask_black, mask_white):

            n  = np.roll(spins,  1, axis=0)
            n += np.roll(spins, -1, axis=0)
            n += np.roll(spins,  1, axis=1)
            n += np.roll(spins, -1, axis=1)

            s_sub = spins[mask]
            n_sub = n[mask]
            deltaE = 2.0 * s_sub * n_sub

            r = rng.random(s_sub.shape)
            accept = (deltaE <= 0) | (r < np.exp(-beta * deltaE))

            s_sub[accept] *= -1
            spins[mask] = s_sub

        ms.append(spins.mean())
    return np.array(ms)
\end{verbatim}


%%%%%%%%%%%%%%%%%%%%%%%%%%%%%%%%%%%%%%%%%%%%%%%%%%%%%%%%%%%%%%
\subsection{メトロポリス法、熱浴法によるイジングモデルのシミュレーション(小問2)}
\subsubsection{シミュレーション条件}
本問では,問題文の指定に従い以下の条件でシミュレーションを行う。

\begin{itemize}
  \item 系サイズ: $L = 64$。
  \item 境界条件:トーラス境界。
  \item 初期状態:全スピン $s_i(0)=+1$。
  \item 観測量:逐次ステップにおける磁化
  \[
  m(t)=\frac{1}{L^2}\sum_{i} s_i(t).
  \]
  \item 温度:$z=\exp(2/T)-1=\sqrt{2}$ を満たす温度 $T$。
\end{itemize}


\subsubsection{磁化の時間発展(小問2)}

初期状態 $s_i(0)=+1$ からスタートし,
熱浴法およびメトロポリス法で $10000$ ステップの更新を行った。
結果として得られた磁化 $m(t)$ の時間発展を比較したところ、以下の図のようになった。

\begin{figure}[H]
  \centering
  \includegraphics[width=0.9\textwidth]{task4_2_heat.png}
  \caption{熱よく法による磁化の時間発展}
    \label{fig:ising-magnetization}
\end{figure}

\begin{figure}[H]
  \centering
  \includegraphics[width=0.9\textwidth]{task4_2_metro.png}
  \caption{メトロポリス法による磁化の時間発展}
    \label{fig:ising-magnetization-metropolis}
\end{figure}


%%%%%%%%%%%%%%%%%%%%%%%%%%%%%%%%%%%%%%%%%%%%%%%%%%%%%%%%%%%%%%
\subsubsection{考察}

図\ref{fig:ising-magnetization},図\ref{fig:ising-magnetization-metropolis}から、いずれの方法でも初期の$+1$状態から出発して磁化の逆転を繰り返しながら平衡状態を中心に振動する様子が観察された。
熱浴法とメトロポリス法の比較では、メトロポリス方法の方が磁化の振動が大きく、不安定であることがわかる。いずれの方法でも、500ステップたてば十分に初期の$+1$状態からの影響が消失しており、小問3で用いるburn-in期間として適切であると考えられる。

%%%%%%%%%%%%%%%%%%%%%%%%%%%%%%%%%%%%%%%%%%%%%%%%%%%%%%%%%%%%%%
\subsection{磁化の期待値と系サイズ・温度依存性(小問3)}

本問では,磁化の期待値
\[
\langle m\rangle(T,L)
= \Bigl\langle \frac{1}{L^2}\sum_i s_i \Bigr\rangle
\]
の温度 $T$ および系サイズ $L$ への依存性を,
モンテカルロ法により求める.
ここで $\langle\cdot\rangle$ は平衡分布に関する期待値を表す。
問題文の条件に従い,温度は
\[
z = \exp(2/T)-1
\]
が区間 $z\in(1.2,1.5]$ を等間隔に分割する $12$ 点以上を用いた。
この区間には二次元イジング模型の臨界温度
\[
T_c = \frac{2}{\ln(1+\sqrt{2})}
\quad\Longleftrightarrow\quad
z_c = \sqrt{2}
\]
が含まれており,秩序–無秩序転移の挙動を観察できる。

系サイズは
\[
L = 12,\,16,\,24,\,32,\,48,\,64
\]
の 6 通りについて計算した。
各 $(L,z)$ ごとに,メトロポリス法と熱浴法のいずれかで
マルコフ連鎖を構成し,十分に長いモンテカルロステップ (MCS) を回した。

%%%%%%%%%%%%%%%%%%%%%%%%%%%%%%%%%%%%%%%%%%%%%%%%%%%%%%%%%%%%%%
\subsubsection{サンプリング方法}

問題文の指示に従い,各条件 $(L,z)$ について
総サンプリング数が $10^6$ MCS 以上となるように設定した。
具体的には,以下のような方法で実装した。

\begin{itemize}
  \item それぞれの $(L,z)$ について独立なイジング系を $N=1000$ 個用意する。
  \item 各系を同じ温度 $T$ のもとで $1500$ MCS だけ時間発展させる。
  \item 初期値依存性を除くため,問題(2)の結果を参考に,
        最初の $500$ ステップを burn-in として捨て,
        残り $1000$ ステップのみを平均に用いる。
\end{itemize}

このとき,有効サンプル数は
\[
N_{\mathrm{sample}}
= N \times (1500 - 500)
= 1000 \times 1000
= 10^6
\]
となり,問題文の条件を満たす。

各ステップ $t$ における $k$ 番目の系の磁化を $m_k(t)$ と書くと,
磁化の期待値の推定量は
\[
\langle m\rangle(T,L)
\simeq
\frac{1}{N_{\mathrm{sample}}}
\sum_{k=1}^{N}\sum_{t=t_{\mathrm{burn}}+1}^{t_{\mathrm{max}}}
m_k(t),
\qquad
t_{\mathrm{burn}}=500,\quad t_{\mathrm{max}}=1500.
\]
実装上は,まず各ステップごとに格子平均
$m_k(t)=L^{-2}\sum_i s_i^{(k)}(t)$ を計算し,
その後,$k$ と $t$ の二重和を取ることで $\langle m\rangle$ を求めた。

\subsubsection{実装上の工夫}

本問のシミュレーションでは,$L=64$ に対して $10^6$ サンプル以上を要求されるため,
単純な Python ループに基づく逐次更新では計算量が膨大となり,
実行時間が現実的でなくなる。
そこで本研究では次の 3 点の最適化を施し,
計算速度を大幅に改善した。

\paragraph{(1) 系の並列化:$N=1000$ 系を同時に更新}

問題文にある「$10^6$ MCS 以上」という条件を効率的に満たすために,
1 つの温度に対して $N=1000$ 個の独立なイジング系を同時に生成し,
これらを同一の温度で並列に時間発展させた。
これにより,1 系あたり $1000$ ステップ進めるだけで
\[
N \times (1500 - 500)
= 1000 \times 1000
= 10^6
\]
という必要サンプル数を一度に確保でき,
ループ回数を従来の $1000$ 分の 1 以下に削減できた。

\paragraph{(2) PyTorch による GPU 並列計算}

NumPy による CPU ベースの更新では,
$N=1000$ 系を同時に扱うと計算量が急激に増大する。
そこで本研究では PyTorch を用いて,
スピン配列 $(K, N, L, L)$ を GPU メモリ上に展開し,
以下のすべての処理を GPU 上で実行した。

\begin{itemize}
  \item 近傍和の計算({\tt torch.roll} によるベクトル化)
  \item 熱浴法の確率計算({\tt torch.sigmoid} による高速化)
  \item メトロポリス法の採択判定
  \item 磁化の計算(空間平均の並列化)
\end{itemize}

これにより,CPU 実装では数十分を要する計算が,
GPU ではわずか数十秒程度で完了するようになった。

\paragraph{(3) 温度点 $z$ の 12 個をひとまとめにして並列更新}

問題文にある ``$z\in(1.2,1.5]$ の範囲で 12 点以上'' という条件に対し,
通常であれば温度を 1 つずつ変えて 12 回のシミュレーションを行う必要がある。
しかし本研究では温度の逆温度 $\beta$ を 12 個まとめて
\[
\beta_k \quad (k=1,\dots,12)
\]
としてベクトル化し,
スピン配列を
\[
\text{spins} \in \mathbb{R}^{K \times N \times L \times L}
\]
として $K=12$ 個の温度点を同時並列に扱った。
これにより,12 回の独立な実行を 1 回にまとめることができ,
計算時間をさらに 1/12 に短縮できた。

\paragraph{(4) 実行速度の改善と $L=128$ への拡張}

以上の最適化により,$L=64$ および 12 個の温度点に対する
$10^6$ サンプルの磁化期待値計算が,
\[
\text{総実行時間} \approx 2\ \text{分}
\]
という高速な実行を達成した。

計算速度に十分余裕が生まれたため,
本問の指定にはないものの,
\[
L = 128
\]
についても同一コードで追加計算を行ったところ,
GPU メモリの制約にも抵触せず,
問題なく同様のシミュレーションを実行できた。

\subsubsection{結果}

前節で述べた方法により,
$z=\exp(2/T)-1$ を $z\in(1.2,1.5]$ の範囲で 12 点とり,
それぞれに対応する温度 $T$ において
磁化の期待値 $\langle m\rangle(T,L)$ を計算した。
図\ref{fig:ising-m-vs-T} に,横軸を温度 $T$,縦軸を磁化の期待値としたプロットを示す。
各曲線は $L=12,16,24,32,48,64,128$ に対応しており,
系サイズが大きくなるにつれて曲線がより急峻になる様子が見てとれる。

\begin{figure}[H]
  \centering
  \includegraphics[width=0.9\textwidth]{task4_3_m_vs_T.png}
  \caption{各系サイズ $L$ における磁化の期待値 $\langle m\rangle$ の
    温度 $T$ 依存性.}
  \label{fig:ising-m-vs-T}
\end{figure}

\subsubsection{考察}

低温側($T\lesssim 2.2$)では,大きな系サイズ $L\ge 24$ について
$\langle m\rangle\approx 0.75$--$0.8$ の飽和した値をとり,
強磁性秩序状態にあることが確認できる。
温度を上げていくと,$T\approx 2.25$ 付近から
$\langle m\rangle$ が急激に減少する領域が現れ,
$T\approx 2.35$ 付近ではほとんどの系で
$\langle m\rangle\approx 0$ となる。
高温側($T\gtrsim 2.4$)では,いずれの系サイズでも
$\langle m\rangle$ は 0 近傍に張り付いており,
常磁性状態にあることが分かる。

また,系サイズ $L$ を変化させたときの詳細を見ると,
小さい系サイズ ($L=12,16$) では
$\langle m\rangle$ の減少がなだらかであるのに対し,
$L=48,64,128$ では,
$T\approx 2.25$~$2.3$ の狭い温度範囲で急峻に落ち込んでいる。
これは有限サイズ系における相転移の丸め込みの典型的な挙動と一致する。




図\ref{fig:ising-m-vs-T} から,
二次元イジング模型の強磁性–常磁性転移の特徴的な振る舞いが
数値的に再現されていることが分かる。
低温側では系サイズに依らず大きな自発磁化が観測され,
高温側では $\langle m\rangle\simeq 0$ に収束している。
臨界温度近傍では,系サイズ $L$ が大きくなるほど
$\langle m\rangle(T,L)$ の曲線がより急峻となり,
転移がシャープになっていく。

また,$L$ を大きくすると低温側および高温側の曲線が
ほぼ重なってくる一方で,
臨界付近のみ系サイズ依存性が強くなることも確認できる。
$L < 64$においては$L$によってグラフが大きく変化していたのに対して $L = 64,128$においてはグラフがほぼ重なっていることから、$L=64$あたりで十分に大きな系サイズに達しており、臨界挙動の解析に適していると考えられる。


\section{課題5 不安定性の確認}
\section{課題5}

\subsection{問題設定}

常微分方程式
\begin{align}
  u'(t) = -2u(t) + 1, \qquad u(0)=1
  \label{eq:kadai5_ode}
\end{align}
を考える。
本課題では,ステップ幅 $\Delta t>0$ を用いて
\[
  t_n = n\Delta t,\qquad u_n \approx u(t_n)
\]
とし,2 段法
\begin{align}
  u_n = u_{n-2} + 2\Delta t\, f(t_{n-1},u_{n-1}),
  \qquad f(t,u)=-2u+1
  \label{eq:kadai5_scheme}
\end{align}
による数値解の安定性を解析する。

式 \eqref{eq:kadai5_scheme} は,区間 $[t_{n-2},t_n]$ で
中点 $t_{n-1}$ における微分係数 $u'(t_{n-1})$ を
$2\Delta t$ だけ積分した形になっており,
いわゆる中点公式(明示 2 段法)の一種である。

\subsection{差分スキームの形}

\eqref{eq:kadai5_ode} を \eqref{eq:kadai5_scheme} に代入すると
\begin{align}
  u_n
  &= u_{n-2} + 2\Delta t\,(-2u_{n-1}+1) \nonumber\\
  &= u_{n-2} - 4\Delta t\,u_{n-1} + 2\Delta t
  \label{eq:kadai5_un_recur}
\end{align}
を得る。

この差分方程式の定常解を考えると,
連続系の平衡解 $u_\ast$ は
\[
  -2u_\ast + 1 = 0 \quad\Rightarrow\quad u_\ast = \frac{1}{2}
\]
であるから,数値解についても
\[
  u_n \equiv \frac{1}{2}
\]
が定常解になっていることが分かる。

数値解の安定性を見るため,
平衡解からの偏差
\[
  v_n = u_n - u_\ast = u_n - \frac{1}{2}
\]
を導入する。\eqref{eq:kadai5_un_recur} に
$u_n = v_n + \tfrac{1}{2}$ を代入すると
\begin{align}
  v_n + \frac{1}{2}
  &= v_{n-2} + \frac{1}{2}
   - 4\Delta t\left(v_{n-1} + \frac{1}{2}\right) + 2\Delta t.
\end{align}
両辺から $\tfrac{1}{2}$ を消去すると
\begin{align}
  v_n
  &= v_{n-2} - 4\Delta t\,v_{n-1}
\end{align}
を得る。したがって誤差 $v_n$ は線形 2 階の斉次漸化式
\begin{align}
  v_n + 4\Delta t\,v_{n-1} - v_{n-2} = 0
  \label{eq:kadai5_v_recur}
\end{align}
に従う。

\subsection{特性方程式による安定性解析}

漸化式 \eqref{eq:kadai5_v_recur} の特性方程式は
\begin{align}
  D(\lambda) = \lambda^2 + 4\Delta t\,\lambda - 1 = 0
  \label{eq:kadai5_charpoly}
\end{align}
である。\eqref{eq:kadai5_charpoly} の 2 つの根を
$\lambda_1,\lambda_2$ とすると,
係数比較から
\begin{align}
  \lambda_1 + \lambda_2 &= -4\Delta t, \\
  \lambda_1 \lambda_2 &= -1
  \label{eq:kadai5_root_product}
\end{align}
が成り立つ。

特に,\eqref{eq:kadai5_root_product} より
\[
  |\lambda_1 \lambda_2| = 1
\]
であるから,もし一方の根が $|\lambda_1|<1$ を満たすならば,
もう一方は
\[
  |\lambda_2| = \frac{1}{|\lambda_1|} > 1
\]
を満たさざるを得ない。
すなわち,2 つの根の少なくとも一方は必ず $|\lambda|>1$ となる。

根の位置をもう少し具体的に確認するため,
$D(\lambda)$ の符号を調べる。
$\Delta t>0$ とすると
\begin{align}
  D(-1) &= (-1)^2 + 4\Delta t(-1) - 1
        = -4\Delta t < 0, \\
  D(0)  &= -1 < 0, \\
  D(1)  &= 1 + 4\Delta t - 1
        = 4\Delta t > 0.
\end{align}
したがって,連続性より
\[
  0 < \lambda_1 < 1,\qquad \lambda_2 < -1
\]
となる 2 つの実根が存在する。
このうち $\lambda_2$ は必ず $|\lambda_2|>1$ を満たす。

漸化式 \eqref{eq:kadai5_v_recur} の一般解は
\begin{align}
  v_n = C_1 \lambda_1^n + C_2 \lambda_2^n
\end{align}
と書けるので,
初期値 $(v_0,v_1)$ が特別に選ばれて
$C_2=0$ となる場合を除き,
$\lambda_2^n$ の項が支配的になり
\[
  |v_n| \sim |C_2|\,|\lambda_2|^n \to \infty
  \qquad (n\to\infty)
\]
となる。

\subsection{まとめ}

以上から,差分スキーム \eqref{eq:kadai5_scheme} を
\eqref{eq:kadai5_ode} に適用すると,
平衡解からの偏差 $v_n$ は必ず
\[
  |\lambda_2|>1
\]
を持つ固有値成分を含むため,
一般の初期値に対して $n\to\infty$ で発散する。

すなわち,任意の $\Delta t>0$ に対して,
この 2 段法は式 \eqref{eq:kadai5_ode} に対して
\emph{安定ではなく},$u_n$ は必ず発散する。

\subsection{数値解析による結果}

常微分方程式
\begin{align}
  u'(t) = -2u(t) + 1, \qquad u(0)=1
\end{align}
に対して,2 段法
\begin{align}
  u_n = u_{n-2} + 2\Delta t\,(-2u_{n-1}+1)
  \label{eq:kadai5_scheme_num}
\end{align}
を用いて数値解の時間発展を調べた。

時間区間は $[0,10]$ とし,刻み幅
\[
  \Delta t = 0.01,\ 0.05,\ 0.10
\]
の 3 通りについて計算を行った。  
2 段法であるため,$u_0=1$ に加え,$u_1$ は解析解
\[
  u_{\mathrm{exact}}(t)
  = \frac{1}{2} + \frac{1}{2} e^{-2t}
\]
を用いて
\[
  u_1 = u_{\mathrm{exact}}(\Delta t)
\]
と与えた。その後は式 \eqref{eq:kadai5_scheme_num} に従って
$n=2,3,\dots,N$ まで時間発展させた($N\Delta t = 10$)。

図\ref{fig:task5_twostep} に,真値と 3 種類の刻み幅に対する数値解を示す。
縦軸は $u\in[-200,200]$ に固定して描画した。
いずれの刻み幅でも,初期のうちは真値と同様に減衰しているように見えるが,
時間が進むにつれて振幅が増大し,$t\approx 10$ では
$u$ が $\pm 10^2$~$10^5$ のオーダで発散していることが分かる。
特に $\Delta t$ が大きいほど発散が急激であり,
$\Delta t=0.10$ の場合には数値解が短時間で縦軸の表示範囲を大きく超えている。

\begin{figure}[H]
  \centering
  \includegraphics[width=0.7\linewidth]{task5_two_step.png}
  \caption{課題5の2段法
           $u_n = u_{n-2} + 2\Delta t(-2u_{n-1}+1)$
           による式(4.29)の数値解
           ($\Delta t = 0.01,\,0.05,\,0.10$)。}
  \label{fig:task5_twostep}
\end{figure}


$t=10$ における数値解 $u_N$ と真値 $u_{\mathrm{exact}}(10)$ の絶対誤差
\[
  E = \left|u_N - u_{\mathrm{exact}}(10)\right|
\]
を表\ref{tab:kadai5_error} に示す(有効数字 4 桁)。

\begin{table}[H]
  \centering
  \caption{課題5 の 2 段法における $t=10$ での数値解と誤差}
  \begin{tabular}{c|c|c}
    \hline
    $\Delta t$ & $u_N$ & $E = |u_N-u_{\mathrm{exact}}(10)|$ \\
    \hline
    $0.01$ & $1.588\times 10^{2}$ & $1.583\times 10^{2}$ \\
    $0.05$ & $1.753\times 10^{4}$ & $1.753\times 10^{4}$ \\
    $0.10$ & $1.120\times 10^{5}$ & $1.120\times 10^{5}$ \\
    \hline
  \end{tabular}
  \label{tab:kadai5_error}
\end{table}

真値 $u_{\mathrm{exact}}(10)\approx 0.5$ が有界であるのに対し,
どの刻み幅でも $u_N$ は時間とともに急速に増大しており,
解析で得られた「本 2 段法の特性根の一つは常に $|\lambda|>1$ となるため,
一般の初期値に対して数値解は必ず発散する」という結果と
定性的にも定量的にも一致している。


\subsection*{課題6 爆発解の数値計算と弧長パラメータによる変数変換}

\subsection{原理と方法}

生物個体群の増減を表す簡単なアリー効果モデルとして,次の常微分方程式を考える:
\begin{equation}
  u'(t) = (u(t)-1)\,u(t), \qquad u(0) = u_0>0.
  \label{eq:allee-ode}
\end{equation}
右辺は個体数 $u$ が多いほど増加率が大きくなること,さらに $u<1$ のときは右辺が負となり,個体群が減少に向かうことを表している。

式\eqref{eq:allee-ode} の解は変数分離により解析的に求めることができる。$u'(t)=u(u-1)$ より
\[
  \frac{1}{u(u-1)}\,\mathrm{d}u = \mathrm{d}t
\]
を積分すると
\[
  \ln\left|\frac{u-1}{u}\right| = t + C
\]
を得る。初期条件 $u(0)=u_0$ を代入して $C$ を消去すると,解析解は
\begin{equation}
  u(t) =
  \begin{cases}
    0, & u_0 = 0,\\[3pt]
    1, & u_0 = 1,\\[3pt]
    \displaystyle \frac{u_0}{u_0 + (1-u_0)e^{t}}, & \text{それ以外}
  \end{cases}
  \label{eq:allee-analytic}
\end{equation}
と書ける。$0<u_0<1$ では $t\to\infty$ で $u(t)\to 0$,$u_0=1$ では $u(t)\equiv 1$,$u_0>1$ では有限時刻で分母が 0 となり,解は有限時間で爆発する。特に $u_0>1$ の場合の爆発時刻 $T_{\mathrm{b}}$ は
\begin{equation}
  u_0 + (1-u_0)e^{T_{\mathrm{b}}} = 0
  \quad\Longrightarrow\quad
  T_{\mathrm{b}} = \ln\frac{u_0}{u_0-1}
  \label{eq:blowup-time-exact}
\end{equation}
と求まる。

爆発解の数値計算では,$t$ のごく短い変化に対して $u$ が急激に増加するため,固定ステップ幅の手法では安定に追跡することが困難になることが多い。本課題では,解軌道 $(t,u(t))$ に沿った弧長
\[
  s(t) = \int_0^t \sqrt{1 + (u'(\tau))^2}\,\mathrm{d}\tau
\]
を新たな独立変数として用い,弧長方向に一様ステップで積分することで爆発近傍の解を追跡する\cite{exp2025}。

\subsection{数値解法の実装}

教科書の弧長パラメータへの変数変換に従えば,$f(t,u)=(u-1)u$ に対して
\begin{align}
  \frac{\mathrm{d}u}{\mathrm{d}s}
  &= \frac{f(t,u)}{\sqrt{1+f(t,u)^2}}
   = \frac{(u-1)u}{\sqrt{1 + (u-1)^2u^2}},\\
  \frac{\mathrm{d}t}{\mathrm{d}s}
  &= \frac{1}{\sqrt{1+f(t,u)^2}}
   = \frac{1}{\sqrt{1 + (u-1)^2u^2}},
\end{align}
となり,$s$ を独立変数とする 2次元の常微分方程式系が得られる。

実装では,この系をそのまま解く代わりに,ステップ幅 $\Delta t$ を
\begin{equation}
  \Delta s \approx \sqrt{1 + f(t,u)^2}\,\Delta t = \text{const.}
\end{equation}
とみなして更新する。具体的には,まず等間隔の基準幅
\[
  \Delta t_{\mathrm{default}} = \frac{t_{\mathrm{end}} - t_{\mathrm{start}}}{n}
\]
を定め,各ステップで
\begin{equation}
  \Delta t = \frac{\Delta t_{\mathrm{default}}}{\sqrt{1+f(t,u)^2}}
\end{equation}
と更新することで,解軌道に沿った弧長ステップ $\Delta s$ が概ね一定になるように制御した。

従属変数の更新には 4 次のルンゲ・クッタ法を用いた。数値解が発散したかの判定には $|u|>10^5$ を閾値とし,この閾値を超えた時点で $u=\infty$ とみなして計算を打ち切った。計算は $t\in[0,5]$ とし,初期値 $u_0$ を
\[
  u_0 = 0.0, 0.2, 0.4, \dots, 2.0
\]
の 11 通りについて実行した。

爆発解の評価では,$u$ 自体の誤差は爆発近傍で非常に大きくなり比較が困難であるため,値を圧縮するための変換として
\[
  v(t) = \arctan(u(t))
\]
を導入した。これにより $u\to\infty$ のときでも $v\to\pi/2$ で抑えられ,数値解と解析解の差
\[
  e_{\mathrm{atan}}(t) = \bigl|\arctan u_{\mathrm{num}}(t) - \arctan u_{\mathrm{exact}}(t)\bigr|
\]
を全時間区間で評価できる。

さらに $u_0>1$ の場合については,最後の有限な 2 点
\[
  \bigl(t_{n-1},\arctan u_{n-1}\bigr),\quad
  \bigl(t_n,\arctan u_n\bigr)
\]
を通る直線で $\arctan u$ を近似し,この直線が $y=\pi/2$ と交わる $t$ を爆発時刻の数値的推定値 $\hat{T}_{\mathrm{b}}$ として求め,解析的な $T_{\mathrm{b}}$(式\eqref{eq:blowup-time-exact})との比較を行った。

\subsection{結果}

図\ref{fig:task5-trajectory} に $u(t)$ と $\arctan(u(t))$ の軌道を示す。上段は $u(t)$ を,下段は $\arctan(u(t))$ を時間 $t$ の関数として描いたものである。

\begin{figure}[htbp]
  \centering
  \includegraphics[width=0.9\linewidth]{task6_trajectory.png}
  \caption{式\eqref{eq:allee-ode} の数値解(上:$u(t)$,下:$\arctan u(t)$)。
  初期値 $u_0=0.0,0.2,\dots,2.0$ の結果を重ねて表示した。}
  \label{fig:task5-trajectory}
\end{figure}

図より,$0<u_0<1$ のとき解は単調減少し $t\to\infty$ で $u(t)\to 0$ となること,$u_0=1$ では $u(t)\equiv 1$ の平衡解であること,$u_0>1$ では有限時間で爆発することが確認できる。$\arctan(u(t))$ をプロットした下段の図では,$u_0>1$ の場合に $\arctan(u(t))$ が $\pi/2$ に向かって滑らかに増加し,爆発時刻近傍の挙動を有限スケールで観察できている。

数値解の精度評価として,$\arctan(u)$ 空間での平均絶対誤差 $\mathrm{mean}|e_{\mathrm{atan}}|$ と最大絶対誤差 $\max|e_{\mathrm{atan}}|$ を求めた結果の一部を表\ref{tab:task5-atan-error} に示す。$u_0\leq 1$ の範囲では
\[
  \mathrm{mean}|e_{\mathrm{atan}}| \lesssim 1.3\times 10^{-6},\qquad
  \max|e_{\mathrm{atan}}| \lesssim 3.1\times 10^{-6}
\]
となっており,弧長パラメータを用いたルンゲ・クッタ法が高い精度で解析解を再現している。

\begin{table}[htbp]
  \centering
  \caption{$\arctan(u)$ 空間での絶対誤差と爆発時刻推定値(抜粋)。}
  \label{tab:task5-atan-error}
  \begin{tabular}{cccccc}
    \hline
    $u_0$ & $\mathrm{mean}|e_{\mathrm{atan}}|$ & $\max|e_{\mathrm{atan}}|$
          & $T_{\mathrm{b}}$ & $\hat{T}_{\mathrm{b}}$ & $\hat{T}_{\mathrm{b}}-T_{\mathrm{b}}$\\
    \hline
    0.2 & $9.38\times 10^{-8}$ & $9.65\times 10^{-7}$ & --- & --- & ---\\
    0.4 & $4.80\times 10^{-7}$ & $2.86\times 10^{-6}$ & --- & --- & ---\\
    0.6 & $9.52\times 10^{-7}$ & $3.03\times 10^{-6}$ & --- & --- & ---\\
    0.8 & $1.23\times 10^{-6}$ & $3.03\times 10^{-6}$ & --- & --- & ---\\
    1.0 & $0$                 & $0$                  & --- & --- & ---\\
    \hline
    1.2 & $4.92\times 10^{-4}$ & $4.99\times 10^{-4}$ & $1.791759$ & $1.791261$ & $-4.98\times 10^{-4}$\\
    1.4 & $4.93\times 10^{-4}$ & $4.99\times 10^{-4}$ & $1.252763$ & $1.252265$ & $-4.98\times 10^{-4}$\\
    1.6 & $4.93\times 10^{-4}$ & $4.99\times 10^{-4}$ & $0.980829$ & $0.980331$ & $-4.98\times 10^{-4}$\\
    1.8 & $4.93\times 10^{-4}$ & $4.99\times 10^{-4}$ & $0.810930$ & $0.810432$ & $-4.98\times 10^{-4}$\\
    2.0 & $4.94\times 10^{-4}$ & $4.99\times 10^{-4}$ & $0.693147$ & $0.692649$ & $-4.98\times 10^{-4}$\\
    \hline
  \end{tabular}
\end{table}

$u_0>1$ の場合,$\arctan(u)$ 空間での平均・最大誤差はいずれも約 $5\times 10^{-4}$ 程度であり,角度に換算するとおよそ $0.03^\circ$ に相当する。また,全ての $u_0>1$ について,爆発時刻の数値的推定値 $\hat{T}_{\mathrm{b}}$ は解析的な $T_{\mathrm{b}}$ より約 $4.98\times 10^{-4}$ だけ早い(負の差)というほぼ一定のバイアスを示し,相対誤差は最大でも $|\hat{T}_{\mathrm{b}}-T_{\mathrm{b}}|/T_{\mathrm{b}}\lesssim 7.2\times 10^{-4}$ に抑えられている。

\subsubsection*{考察}

式\eqref{eq:allee-ode} の右辺 $(u-1)u$ は $u=0$ および $u=1$ を平衡点として持ち,$0<u<1$ のとき負,$u>1$ のとき正である。このため,$0<u_0<1$ の解は単調減少して $u=0$ に収束し,$u_0=1$ は不動点として保たれる。一方 $u_0>1$ では $u(t)$ が単調増加し,有限時間で爆発することが解析的にも数値的にも確認された。

固定ステップ幅の方法で爆発解を扱うと,$u'(t)$ が非常に大きくなる領域で解の変化を捉えるために極端に小さな $\Delta t$ が必要になる。本課題で用いた弧長に基づくステップ幅制御では,$|u'|$ が大きくなるにつれて $\Delta t$ が自動的に小さくなり,弧長方向にはほぼ一定の刻み幅で積分が進む。その結果,$u_0>1$ の場合にも爆発直前まで解の軌道を比較的少ないステップ数で追跡することができた。

また,今回の変換の仕様上,値が無限に発散する爆発解であっても,1ステップで縦幅にて高々Δtの増分で更新されるため,数値解が十分大きくなるのに途方もない時間がかかるので,ループの回数の上限を設定し,閾値を超えた時点で計算を打ち切るようにした。このため,爆発直前の非常に大きな $u$ の値は数値的に得られなかったが,$\arctan(u)$ 空間に圧縮することにより,爆発近傍の挙動が有限スケールで観察でき,これによって関数uが無限大に相当する$v=\pi/2$に到達する時刻を推定できた。



\section{結論}


\appendix
\section{使用コード一覧}

\begin{itemize}

  \item レポート作成、Pythonの実装の一部にGitHub Copilotの補完を使用
\end{itemize}


% ===== 参考文献 =====
\begin{thebibliography}{9}
\bibitem{exp2025} 数理工学実験(2025年度配布資料).
\end{thebibliography}

\end{document}
