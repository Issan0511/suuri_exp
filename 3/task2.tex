Adams 型多段法の一般的構成に基づき,
4次のアダムス・バッシュフォース法(Adams–Bashforth, AB4)と
4次のアダムス・ムルトン法(Adams–Moulton, AM4)を
\emph{Lagrange 補間の積分を明示的に行うことで}導出する。
さらに,これらを組み合わせて 4次の予測子・修正子法を構成する。

対象とする常微分方程式は
\begin{align}
  u' = f(t,u), \qquad u(t_n) = u_n
\end{align}
とし,時間刻みを一定 $\Delta t$ とおく。
資料と同様に
\[
  f_n = f(t_n,u_n)
\]
と略記する。

\subsection{4次アダムス・バッシュフォース法の導出}

まず,
\[
  u_n
  = u_{n-1} + \int_{t_{n-1}}^{t_n} f(\tau,u(\tau))\, d\tau
\]
を数値積分で近似することを考える。
4次 AB 法では,区間 $[t_{n-1},t_n]$ 上の $f$ を
\[
  (t_{n-1}, f_{n-1}),\ (t_{n-2}, f_{n-2}),\ (t_{n-3}, f_{n-3}),\ (t_{n-4}, f_{n-4})
\]
の 4 点を通る 3 次 Lagrange 多項式で近似し,その積分をとる。

\subsubsection*{変数変換}
等間隔刻み $\Delta t$ を用いて
\begin{align}
  \tau = t_{n-1} + \theta \Delta t, \qquad \theta \in [0,1]
\end{align}
とおくと,
\[
  d\tau = \Delta t\, d\theta
\]
より
\begin{align}
  \int_{t_{n-1}}^{t_n} f(\tau,u(\tau))\, d\tau
  &= \Delta t \int_0^1 f(t_{n-1}+\theta\Delta t, u(t_{n-1}+\theta\Delta t))\, d\theta \\
  &= \Delta t \int_0^1 g(\theta)\, d\theta,
\end{align}
ただし
\[
  g(\theta) = f(t_{n-1}+\theta\Delta t, u(t_{n-1}+\theta\Delta t))
\]
とおいた。

$\theta$ の節点は
\[
  t_{n-1} \leftrightarrow \theta = 0,\quad
  t_{n-2} \leftrightarrow \theta = -1,\quad
  t_{n-3} \leftrightarrow \theta = -2,\quad
  t_{n-4} \leftrightarrow \theta = -3
\]
となる。

\subsubsection*{Lagrange 多項式の構成}
節点
\[
  \theta_0 = 0,\quad \theta_1 = -1,\quad \theta_2 = -2,\quad \theta_3 = -3
\]
に対し,
\[
  g(\theta_k) = f_{n-1-k}\quad (k=0,1,2,3)
\]
となる。Lagrange 基底多項式を
\[
  \ell_k(\theta)
  = \prod_{\substack{j=0\\ j\neq k}}^3
    \frac{\theta-\theta_j}{\theta_k-\theta_j}
\]
とおくと,
\[
  g(\theta) \approx
  \ell_0(\theta) f_{n-1}
  + \ell_1(\theta) f_{n-2}
  + \ell_2(\theta) f_{n-3}
  + \ell_3(\theta) f_{n-4}
\]
と表される。

各 $\ell_k(\theta)$ を具体的に書くと
\begin{align}
  \ell_0(\theta)
  &= \frac{(\theta-\theta_1)(\theta-\theta_2)(\theta-\theta_3)}
          {(\theta_0-\theta_1)(\theta_0-\theta_2)(\theta_0-\theta_3)} \\
  &= \frac{(\theta+1)(\theta+2)(\theta+3)}{(0+1)(0+2)(0+3)}
   = \frac{(\theta+1)(\theta+2)(\theta+3)}{6},
\end{align}
\begin{align}
  \ell_1(\theta)
  &= \frac{(\theta-\theta_0)(\theta-\theta_2)(\theta-\theta_3)}
          {(\theta_1-\theta_0)(\theta_1-\theta_2)(\theta_1-\theta_3)} \\
  &= \frac{\theta(\theta+2)(\theta+3)}
          {(-1-0)(-1+2)(-1+3)}
   = -\frac{\theta(\theta+2)(\theta+3)}{2},
\end{align}
\begin{align}
  \ell_2(\theta)
  &= \frac{(\theta-\theta_0)(\theta-\theta_1)(\theta-\theta_3)}
          {(\theta_2-\theta_0)(\theta_2-\theta_1)(\theta_2-\theta_3)} \\
  &= \frac{\theta(\theta+1)(\theta+3)}
          {(-2-0)(-2+1)(-2+3)}
   = \frac{\theta(\theta+1)(\theta+3)}{2},
\end{align}
\begin{align}
  \ell_3(\theta)
  &= \frac{(\theta-\theta_0)(\theta-\theta_1)(\theta-\theta_2)}
          {(\theta_3-\theta_0)(\theta_3-\theta_1)(\theta_3-\theta_2)} \\
  &= \frac{\theta(\theta+1)(\theta+2)}
          {(-3-0)(-3+1)(-3+2)}
   = -\frac{\theta(\theta+1)(\theta+2)}{6}.
\end{align}

\subsubsection*{係数の積分計算}
\[
  \int_0^1 g(\theta)\, d\theta
  \approx
  \sum_{k=0}^3
  \Bigl(\int_0^1 \ell_k(\theta)\, d\theta\Bigr) f_{n-1-k}
\]
であるから,各
\[
  \alpha_k = \int_0^1 \ell_k(\theta)\, d\theta
\]
を計算すればよい。

\paragraph{$\ell_0$ の積分}
\begin{align}
  \ell_0(\theta)
    &= \frac{(\theta+1)(\theta+2)(\theta+3)}{6} \\
    &= \frac{(\theta^2+3\theta+2)(\theta+3)}{6} \\
    &= \frac{\theta^3+6\theta^2+11\theta+6}{6}.
\end{align}
したがって
\begin{align}
  \alpha_0
  &= \int_0^1 \ell_0(\theta)\, d\theta \\
  &= \frac{1}{6}
     \int_0^1
     (\theta^3+6\theta^2+11\theta+6)\, d\theta \\
  &= \frac{1}{6}
     \Bigl[
       \frac{\theta^4}{4}
       + 6\frac{\theta^3}{3}
       + 11\frac{\theta^2}{2}
       + 6\theta
     \Bigr]_{0}^{1} \\
  &= \frac{1}{6}
     \Bigl(
       \frac14 + 2 + \frac{11}{2} + 6
     \Bigr)
   = \frac{55}{24}.
\end{align}

\paragraph{$\ell_1$ の積分}
\begin{align}
  \ell_1(\theta)
  &= -\frac{\theta(\theta+2)(\theta+3)}{2} \\
  &= -\frac{\theta(\theta^2+5\theta+6)}{2}
   = -\frac{\theta^3+5\theta^2+6\theta}{2}.
\end{align}
よって
\begin{align}
  \alpha_1
  &= \int_0^1 \ell_1(\theta)\, d\theta \\
  &= -\frac12
     \int_0^1 (\theta^3+5\theta^2+6\theta)\, d\theta \\
  &= -\frac12
     \Bigl[
       \frac{\theta^4}{4}
       + 5\frac{\theta^3}{3}
       + 6\frac{\theta^2}{2}
     \Bigr]_{0}^{1} \\
  &= -\frac12
     \Bigl(
       \frac14 + \frac{5}{3} + 3
     \Bigr)
   = -\frac{59}{24}.
\end{align}

\paragraph{$\ell_2$ の積分}
\begin{align}
  \ell_2(\theta)
  &= \frac{\theta(\theta+1)(\theta+3)}{2} \\
  &= \frac{\theta(\theta^2+4\theta+3)}{2}
   = \frac{\theta^3+4\theta^2+3\theta}{2}.
\end{align}
したがって
\begin{align}
  \alpha_2
  &= \int_0^1 \ell_2(\theta)\, d\theta \\
  &= \frac12
     \int_0^1 (\theta^3+4\theta^2+3\theta)\, d\theta \\
  &= \frac12
     \Bigl[
       \frac{\theta^4}{4}
       + 4\frac{\theta^3}{3}
       + 3\frac{\theta^2}{2}
     \Bigr]_{0}^{1} \\
  &= \frac12
     \Bigl(
       \frac14 + \frac{4}{3} + \frac32
     \Bigr)
   = \frac{37}{24}.
\end{align}

\paragraph{$\ell_3$ の積分}
\begin{align}
  \ell_3(\theta)
  &= -\frac{\theta(\theta+1)(\theta+2)}{6} \\
  &= -\frac{\theta(\theta^2+3\theta+2)}{6}
   = -\frac{\theta^3+3\theta^2+2\theta}{6}.
\end{align}
したがって
\begin{align}
  \alpha_3
  &= \int_0^1 \ell_3(\theta)\, d\theta \\
  &= -\frac16
     \int_0^1 (\theta^3+3\theta^2+2\theta)\, d\theta \\
  &= -\frac16
     \Bigl[
       \frac{\theta^4}{4}
       + 3\frac{\theta^3}{3}
       + 2\frac{\theta^2}{2}
     \Bigr]_{0}^{1} \\
  &= -\frac16
     \Bigl(
       \frac14 + 1 + 1
     \Bigr)
   = -\frac{3}{8}
   = -\frac{9}{24}.
\end{align}

\paragraph{AB4 の最終形}
以上より
\[
  \int_{t_{n-1}}^{t_n} f(\tau,u(\tau))\, d\tau
  \approx
  \Delta t \Bigl(
    \alpha_0 f_{n-1}
  + \alpha_1 f_{n-2}
  + \alpha_2 f_{n-3}
  + \alpha_3 f_{n-4}
  \Bigr)
\]
であり,
\begin{align}
  u_n
  &= u_{n-1}
   + \Delta t \Bigl(
    \frac{55}{24} f_{n-1}
   - \frac{59}{24} f_{n-2}
   + \frac{37}{24} f_{n-3}
   - \frac{9}{24}  f_{n-4}
   \Bigr) \\
  &= u_{n-1}
   + \frac{\Delta t}{24}
      \Bigl(
        55 f_{n-1}
      - 59 f_{n-2}
      + 37 f_{n-3}
      - 9  f_{n-4}
      \Bigr).
\end{align}
これが 4次アダムス・バッシュフォース法である。

\subsection{4次アダムス・ムルトン法の導出}

同様にして AM4 を導出する。
今度は節点として
\[
  (t_n,f_n),\ (t_{n-1},f_{n-1}),\ (t_{n-2},f_{n-2}),\ (t_{n-3},f_{n-3})
\]
の 4 点を用いる。先ほどと同じ変数変換
\[
  \tau = t_{n-1} + \theta \Delta t,\quad \theta\in[0,1]
\]
を用いると
\[
  t_n \leftrightarrow \theta=1,\quad
  t_{n-1} \leftrightarrow \theta=0,\quad
  t_{n-2} \leftrightarrow \theta=-1,\quad
  t_{n-3} \leftrightarrow \theta=-2
\]
となる。

節点
\[
  \theta_0=1,\quad \theta_1=0,\quad \theta_2=-1,\quad \theta_3=-2
\]
に対し
\[
  g(\theta_k) =
  \begin{cases}
    f_n     & (k=0), \\
    f_{n-1} & (k=1), \\
    f_{n-2} & (k=2), \\
    f_{n-3} & (k=3)
  \end{cases}
\]
である。Lagrange 基底は
\[
  g(\theta) \approx
  \tilde\ell_0(\theta) f_n
  + \tilde\ell_1(\theta) f_{n-1}
  + \tilde\ell_2(\theta) f_{n-2}
  + \tilde\ell_3(\theta) f_{n-3}
\]
と書ける。

同様に
\begin{align}
  \tilde\ell_0(\theta)
  &= \frac{(\theta-\theta_1)(\theta-\theta_2)(\theta-\theta_3)}
          {(\theta_0-\theta_1)(\theta_0-\theta_2)(\theta_0-\theta_3)} \\
  &= \frac{\theta(\theta+1)(\theta+2)}{(1-0)(1+1)(1+2)}
   = \frac{\theta(\theta+1)(\theta+2)}{6},
\end{align}
\begin{align}
  \tilde\ell_1(\theta)
  &= \frac{(\theta-\theta_0)(\theta-\theta_2)(\theta-\theta_3)}
          {(\theta_1-\theta_0)(\theta_1-\theta_2)(\theta_1-\theta_3)} \\
  &= -\frac{(\theta-1)(\theta+1)(\theta+2)}{2},
\end{align}
\begin{align}
  \tilde\ell_2(\theta)
  &= \frac{(\theta-\theta_0)(\theta-\theta_1)(\theta-\theta_3)}
          {(\theta_2-\theta_0)(\theta_2-\theta_1)(\theta_2-\theta_3)} \\
  &= \frac{\theta(\theta-1)(\theta+2)}{2},
\end{align}
\begin{align}
  \tilde\ell_3(\theta)
  &= \frac{(\theta-\theta_0)(\theta-\theta_1)(\theta-\theta_2)}
          {(\theta_3-\theta_0)(\theta_3-\theta_1)(\theta_3-\theta_2)} \\
  &= \frac{\theta(1-\theta^2)}{6}.
\end{align}

同様に
\[
  \tilde\alpha_k = \int_0^1 \tilde\ell_k(\theta)\, d\theta
\]
を計算すると
\begin{align}
  \tilde\alpha_0 &= \frac{3}{8} = \frac{9}{24}, \\
  \tilde\alpha_1 &= \frac{19}{24}, \\
  \tilde\alpha_2 &= -\frac{5}{24}, \\
  \tilde\alpha_3 &= \frac{1}{24}
\end{align}
が得られる(展開・積分は AB4 の場合と同様に多項式を展開して項別に行う)。

したがって
\begin{align}
  u_n
  &= u_{n-1}
   + \Delta t
      \Bigl(
        \tilde\alpha_0 f_n
      + \tilde\alpha_1 f_{n-1}
      + \tilde\alpha_2 f_{n-2}
      + \tilde\alpha_3 f_{n-3}
      \Bigr) \\
  &= u_{n-1}
   + \frac{\Delta t}{24}
      \Bigl(
         9 f_n
      + 19 f_{n-1}
      -  5 f_{n-2}
      +    f_{n-3}
      \Bigr).
\end{align}
これが 4次アダムス・ムルトン法である。

\subsection{4次予測子・修正子法}

AM4 は $f_n = f(t_n,u_n)$ を含む陰的法である。
そこで,まず AB4 を用いて $u_n$ の予測値 $\tilde u_n$ を計算し,
AM4 の中の $f_n$ を $f(t_n,\tilde u_n)$ に置き換えることで
完全に陽的な 4次予測子・修正子法を得る。

\paragraph{予測 (Predictor: AB4)}
\begin{align}
  \tilde{u}_n
  = u_{n-1}
  + \frac{\Delta t}{24}
    \Bigl(
      55 f_{n-1}
      - 59 f_{n-2}
      + 37 f_{n-3}
      -  9 f_{n-4}
    \Bigr).
\end{align}

\paragraph{修正 (Corrector: AM4)}
\begin{align}
  u_n
  = u_{n-1}
  + \frac{\Delta t}{24}
    \Bigl(
       9 f(t_n,\tilde u_n)
     + 19 f_{n-1}
     -  5 f_{n-2}
     +    f_{n-3}
    \Bigr).
\end{align}

必要に応じて修正ステップを繰り返せば,
AM4 の陰的方程式に対する反復解法として解釈することもできる。


