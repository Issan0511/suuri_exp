
\subsection{原理・方法}

本課題では,常微分方程式の初期値問題
\begin{align}
  u'(t) = u(t), \qquad u(0) = 1
  \label{eq:kadai3_ivp}
\end{align}
を区間 $t \in [0,1]$ で数値的に解き,いくつかの代表的な数値解法の
\emph{収束次数}を,誤差の振る舞いから確認する。

式 \eqref{eq:kadai3_ivp} の解析解は
\begin{align}
  u_{\mathrm{exact}}(t) = \exp(t)
\end{align}
であり,特に $t=1$ における真値は $u_{\mathrm{exact}}(1)=\mathrm{e}$ である。

\subsubsection*{離散化と誤差の定義}

時間刻み幅 $\Delta t$ を
\begin{align}
  \Delta t = \frac{1}{2^i}, \qquad i = 1,2,\dots,i_{\max}
\end{align}
とし,ステップ数を $N = 2^i$,離散化した時刻を
\begin{align}
  t_n = n \Delta t, \qquad n = 0,1,\dots,N
\end{align}
とおく。数値解を $u_n \approx u(t_n)$ と表す。

各 $i$ に対し,$t=1$ に対応する $n=N$ での数値解 $u_N^{(i)}$ を求め,  
絶対誤差
\begin{align}
  E^{(i)} = \left| u_N^{(i)} - u_{\mathrm{exact}}(1) \right|
  = \left| u_N^{(i)} - \mathrm{e} \right|
\end{align}
を定義する。

さらに,連続する刻み幅に対する誤差比
\begin{align}
  E_r^{(i)} = \frac{E^{(i)}}{E^{(i-1)}} \qquad (i \geq 2)
\end{align}
を考える。$p$ 次精度の手法では,$\Delta t$ を半分にすると誤差が理論的に
\begin{align}
  E^{(i)} \approx C (\Delta t)^p = C 2^{-ip}
\end{align}
と振る舞うので,
\begin{align}
  E_r^{(i)} \approx \frac{C 2^{-ip}}{C 2^{-(i-1)p}} = 2^{-p}
\end{align}
が期待される。したがって,数値的に $E_r^{(i)}$ が $2^{-p}$ に近づくかどうかを調べることで,
各手法の収束次数 $p$ を検証できる。

\subsubsection*{対象とする数値解法}

本課題では,次の 5 種類の手法を比較する。

\paragraph{(1) 前進オイラー法 (Forward Euler, FE)}

式 \eqref{eq:kadai3_ivp} に対する前進オイラー法は
\begin{align}
  u_{n+1}
  &= u_n + \Delta t\, f(t_n,u_n), \\
  f(t,u) &= u
\end{align}
で与えられる。本問題では $f(t_n,u_n)=u_n$ であるから,
\begin{align}
  u_{n+1} = (1+\Delta t)\, u_n
\end{align}
となる。これは 1 次精度の陽的一段法である。

\paragraph{(2) 2 次アダムス・バッシュフォース法 (AB2)}

2 次の Adams–Bashforth 法は陽的二段法であり,
\begin{align}
  u_{n+1}
  = u_n + \Delta t
  \left(
    \frac{3}{2} f_n - \frac{1}{2} f_{n-1}
  \right),
  \qquad f_n = f(t_n,u_n)
\end{align}
と表される。本問題では $f_n = u_n$ である。

多段法の性質上,$u_0,u_1$ の 2 点が初期値として必要になる。
本レポートでは,$u_0$ に加えて $u_1 = u_{\mathrm{exact}}(t_1)$ を
解析解から与えている。

\paragraph{(3) 3 次アダムス・バッシュフォース法 (AB3)}

3 次の Adams–Bashforth 法は陽的三段法であり,
\begin{align}
  u_{n+1}
  = u_n + \Delta t
  \left(
    \frac{23}{12} f_n
    - \frac{16}{12} f_{n-1}
    + \frac{5}{12}  f_{n-2}
  \right)
\end{align}
で与えられる。本問題では $f_n=u_n$ である。

三段法のため,$u_0,u_1,u_2$ の 3 点が初期値として必要になる。
本レポートでは,$u_1,u_2$ を
\begin{align}
  u_1 = u_{\mathrm{exact}}(t_1),\qquad
  u_2 = u_{\mathrm{exact}}(t_2)
\end{align}
と解析解から与えている。

\paragraph{(4) ホイン法 (Heun 法, 2 次ルンゲ・クッタ法)}

ホイン法は 2 次精度の陽的一段法であり,予測子・修正子の形で
\begin{align}
  &\text{予測:} && u^{\ast}_{n+1} = u_n + \Delta t\, f(t_n,u_n), \\
  &\text{修正:} && u_{n+1}
    = u_n + \frac{\Delta t}{2}
      \bigl( f(t_n,u_n) + f(t_{n+1},u^{\ast}_{n+1}) \bigr)
\end{align}
と書ける。本問題では $f(t,u)=u$ である。

\paragraph{(5) 4 次ルンゲ・クッタ法 (RK4)}

4 次のルンゲ・クッタ法は
\begin{align}
  k_1 &= f(t_n, u_n), \\
  k_2 &= f\left(t_n + \frac{\Delta t}{2},\, u_n + \frac{\Delta t}{2} k_1\right), \\
  k_3 &= f\left(t_n + \frac{\Delta t}{2},\, u_n + \frac{\Delta t}{2} k_2\right), \\
  k_4 &= f\left(t_n + \Delta t,\, u_n + \Delta t\, k_3\right),
\end{align}
\begin{align}
  u_{n+1}
  = u_n + \frac{\Delta t}{6}
    \left( k_1 + 2k_2 + 2k_3 + k_4 \right)
\end{align}
と定義される。本問題では $f(t,u)=u$ であり,4 次精度の陽的一段法となる。

\subsubsection*{数値実験の手順}

式
\[
  u'(t) = u(t),\quad u(0)=1
\]
を区間 $[0,1]$ で数値的に解き,$t=1$ における解の誤差を比較する。

時間刻みは
\[
  n = 2^i,\quad i = 2,3,\dots,9,\qquad
  \Delta t = \frac{1}{n}
\]
とし,$t_0=0,\,u_0=1$ から $n$ ステップ進めて $t_n=1$ の値 $u_n$ を求める。

実装では,以下の 5 種類の手法を用いた。

\begin{itemize}
  \item 前進オイラー法(Forward Euler)\\
        更新式
        \[
          u_{k+1} = u_k + \Delta t\, f(t_k,u_k),\quad f(t,u)=u
        \]
        を $k=0,\dots,n-1$ について適用する関数
        \texttt{foward\_euler} を用いた。

  \item 2次アダムス・バッシュフォース法(AB2)\\
        更新式
        \[
          u_{k+1}
          = u_k + \Delta t\left(
            \frac{3}{2} f_k - \frac{1}{2} f_{k-1}
          \right),\quad f_k=f(t_k,u_k)
        \]
        を用いる関数 \texttt{adam\_bashforth2} を実装した。
        多段法の初期化のため,$u_0=1$ に加えて
        \[
          u_1 = u_{\mathrm{exact}}(t_1) = \exp(\Delta t)
        \]
        を解析解から与え,$k=1,\dots,n-1$ で上式を適用した。

  \item 3次アダムス・バッシュフォース法(AB3)\\
        更新式
        \[
          u_{k+1}
          = u_k + \Delta t\left(
            \frac{23}{12} f_k
            - \frac{16}{12} f_{k-1}
            + \frac{5}{12}  f_{k-2}
          \right)
        \]
        を用いる関数 \texttt{adam\_bashforth3} を実装した。
        初期化のため $u_0=1$ に加え
        \[
          u_1 = u_{\mathrm{exact}}(t_1),\quad
          u_2 = u_{\mathrm{exact}}(t_2)
        \]
        を解析解から与え,$k=2,\dots,n-1$ で上式を適用した。

  \item ホイン法(Heun 法,2次ルンゲ・クッタ法)\\
        関数 \texttt{heun} で
        \begin{align}
          &\text{予測: } &&
            u^{\ast}_{k+1} = u_k + \Delta t\, f(t_k,u_k),\\
          &\text{修正: } &&
            u_{k+1}
              = u_k + \frac{\Delta t}{2}
              \bigl( f(t_k,u_k) + f(t_{k+1},u^{\ast}_{k+1}) \bigr)
        \end{align}
        を $k=0,\dots,n-1$ について適用した。

  \item 4次ルンゲ・クッタ法(RK4)\\
        関数 \texttt{runge\_kutta4} で
        \begin{align}
          k_1 &= f(t_k, u_k),\\
          k_2 &= f\!\left(t_k + \frac{\Delta t}{2},\, u_k + \frac{\Delta t}{2} k_1\right),\\
          k_3 &= f\!\left(t_k + \frac{\Delta t}{2},\, u_k + \frac{\Delta t}{2} k_2\right),\\
          k_4 &= f\!\left(t_k + \Delta t,\, u_k + \Delta t\, k_3\right),
        \end{align}
        \[
          u_{k+1}
            = u_k + \frac{\Delta t}{6}
              (k_1 + 2k_2 + 2k_3 + k_4)
        \]
        を $k=0,\dots,n-1$ について適用した。
\end{itemize}

各 $n$ に対して,$t=1$ における真値 $u_{\mathrm{exact}}(1)=\mathrm{e}$ との差
\[
  E(n) = \bigl|u_n - \mathrm{e}\bigr|
\]
を計算し,誤差の大きさと減少の仕方を比較した。
さらに,誤差比
\[
  \frac{E(n)}{E(n/2)}
\]
を用いて,各手法の収束次数を推定した。

\subsection{結果}

$n=4,8,\dots,512$ に対して得られた $t=1$ における誤差 $E(n)$ を
表\ref{tab:kadai3_error} に示す(有効数字 4 桁)。

\begin{table}[htbp]
  \centering
  \caption{$t=1$ における絶対誤差 $E(n)=|u_n-\mathrm{e}|$}
  \begin{tabular}{c|ccccc}
    \hline
    $n$
      & 前進オイラー
      & AB2
      & AB3
      & ホイン
      & RK4 \\
    \hline
     4  & 0.2769 & 0.04240        & 0.005826        & 0.02343        & $7.189\times 10^{-5}$ \\
     8  & 0.1525 & 0.01407        & 0.001300        & 0.006441       & $4.984\times 10^{-6}$ \\
    16  & 0.08035& 0.003973       & $2.035\times 10^{-4}$ & 0.001688 & $3.281\times 10^{-7}$ \\
    32  & 0.04129& 0.001050       & $2.820\times 10^{-5}$ & $4.322\times 10^{-4}$ & $2.105\times 10^{-8}$ \\
    64  & 0.02094& $2.696\times 10^{-4}$ & $3.704\times 10^{-6}$ & $1.093\times 10^{-4}$ & $1.333\times 10^{-9}$ \\
   128  & 0.01054& $6.826\times 10^{-5}$ & $4.745\times 10^{-7}$ & $2.749\times 10^{-5}$ & $8.384\times 10^{-11}$ \\
   256  & 0.005290& $1.717\times 10^{-5}$& $6.003\times 10^{-8}$ & $6.893\times 10^{-6}$ & $5.261\times 10^{-12}$ \\
   512  & 0.002650& $4.307\times 10^{-6}$& $7.549\times 10^{-9}$ & $1.726\times 10^{-6}$ & $3.286\times 10^{-13}$ \\
    \hline
  \end{tabular}
  \label{tab:kadai3_error}
\end{table}

また,誤差比 $E(n)/E(n/2)$ の典型的な値として,
$n=256$ と $512$ の組に対しては
\[
\begin{array}{ll}
  \text{前進オイラー} & E(512)/E(256) \approx 0.5009,\\
  \text{AB2}           & E(512)/E(256) \approx 0.2508,\\
  \text{AB3}           & E(512)/E(256) \approx 0.1258,\\
  \text{ホイン}        & E(512)/E(256) \approx 0.2504,\\
  \text{RK4}           & E(512)/E(256) \approx 0.0625
\end{array}
\]
が得られた。

\subsection{考察}

表\ref{tab:kadai3_error} から,刻み数 $n$ を 2 倍にすると各手法の誤差は
ほぼ一定の比率で減少していることがわかる。
誤差比 $E(n)/E(n/2)$ の極限値は
\[
  2^{-p}
\]
となることから,$p$ 次精度の手法では
\[
  E(n)/E(n/2) \to 2^{-p}
\]
が期待される。本数値結果では,$n$ が大きい領域で

\begin{itemize}
  \item 前進オイラー法:$E(n)/E(n/2)\to 0.5 \approx 2^{-1}$,
  \item AB2,ホイン法:$E(n)/E(n/2)\to 0.25 \approx 2^{-2}$,
  \item AB3:$E(n)/E(n/2)\to 0.125 \approx 2^{-3}$,
  \item RK4:$E(n)/E(n/2)\to 0.0625 = 2^{-4}$
\end{itemize}
となっており,それぞれ 1 次,2 次,3 次,4 次精度であるという
理論的な収束次数と整合的な結果になっている。

$n=512$ での誤差の大きさを比較すると,
前進オイラー法は $E \approx 2.65\times 10^{-3}$ であるのに対し,
AB2,ホイン法はそれぞれ $10^{-6}$ オーダー,
AB3 は $10^{-9}$ オーダー,
RK4 は $10^{-13}$ オーダーまで誤差が減少している。
同じ刻み幅でも,高次の手法ほど誤差が急速に小さくなることが確認できる。

また,多段法である AB2,AB3 は開始ステップで解析解を用いて初期値を与えているため,
この問題設定では一段法よりも有利な初期条件になっている。
それにもかかわらず,AB2 とホイン法(ともに 2 次精度)の誤差比は
ほぼ同じ値に収束しており,「次数が同じであれば刻み幅に対する誤差の減少率も同程度」
であることが数値的に確かめられる。
RK4 は計算あたりの評価回数が多い一方で,
同じ $n$ に対して最も小さい誤差を与えており,
高次の一段法を用いることにより,
粗い刻みでも高精度な解が得られることがわかる。


