\section{課題5}

\subsection{問題設定}

常微分方程式
\begin{align}
  u'(t) = -2u(t) + 1, \qquad u(0)=1
  \label{eq:kadai5_ode}
\end{align}
を考える。
本課題では,ステップ幅 $\Delta t>0$ を用いて
\[
  t_n = n\Delta t,\qquad u_n \approx u(t_n)
\]
とし,2 段法
\begin{align}
  u_n = u_{n-2} + 2\Delta t\, f(t_{n-1},u_{n-1}),
  \qquad f(t,u)=-2u+1
  \label{eq:kadai5_scheme}
\end{align}
による数値解の安定性を解析する。

式 \eqref{eq:kadai5_scheme} は,区間 $[t_{n-2},t_n]$ で
中点 $t_{n-1}$ における微分係数 $u'(t_{n-1})$ を
$2\Delta t$ だけ積分した形になっており,
いわゆる中点公式(明示 2 段法)の一種である。

\subsection{差分スキームの形}

\eqref{eq:kadai5_ode} を \eqref{eq:kadai5_scheme} に代入すると
\begin{align}
  u_n
  &= u_{n-2} + 2\Delta t\,(-2u_{n-1}+1) \nonumber\\
  &= u_{n-2} - 4\Delta t\,u_{n-1} + 2\Delta t
  \label{eq:kadai5_un_recur}
\end{align}
を得る。

この差分方程式の定常解を考えると,
連続系の平衡解 $u_\ast$ は
\[
  -2u_\ast + 1 = 0 \quad\Rightarrow\quad u_\ast = \frac{1}{2}
\]
であるから,数値解についても
\[
  u_n \equiv \frac{1}{2}
\]
が定常解になっていることが分かる。

数値解の安定性を見るため,
平衡解からの偏差
\[
  v_n = u_n - u_\ast = u_n - \frac{1}{2}
\]
を導入する。\eqref{eq:kadai5_un_recur} に
$u_n = v_n + \tfrac{1}{2}$ を代入すると
\begin{align}
  v_n + \frac{1}{2}
  &= v_{n-2} + \frac{1}{2}
   - 4\Delta t\left(v_{n-1} + \frac{1}{2}\right) + 2\Delta t.
\end{align}
両辺から $\tfrac{1}{2}$ を消去すると
\begin{align}
  v_n
  &= v_{n-2} - 4\Delta t\,v_{n-1}
\end{align}
を得る。したがって誤差 $v_n$ は線形 2 階の斉次漸化式
\begin{align}
  v_n + 4\Delta t\,v_{n-1} - v_{n-2} = 0
  \label{eq:kadai5_v_recur}
\end{align}
に従う。

\subsection{特性方程式による安定性解析}

漸化式 \eqref{eq:kadai5_v_recur} の特性方程式は
\begin{align}
  D(\lambda) = \lambda^2 + 4\Delta t\,\lambda - 1 = 0
  \label{eq:kadai5_charpoly}
\end{align}
である。\eqref{eq:kadai5_charpoly} の 2 つの根を
$\lambda_1,\lambda_2$ とすると,
係数比較から
\begin{align}
  \lambda_1 + \lambda_2 &= -4\Delta t, \\
  \lambda_1 \lambda_2 &= -1
  \label{eq:kadai5_root_product}
\end{align}
が成り立つ。

特に,\eqref{eq:kadai5_root_product} より
\[
  |\lambda_1 \lambda_2| = 1
\]
であるから,もし一方の根が $|\lambda_1|<1$ を満たすならば,
もう一方は
\[
  |\lambda_2| = \frac{1}{|\lambda_1|} > 1
\]
を満たさざるを得ない。
すなわち,2 つの根の少なくとも一方は必ず $|\lambda|>1$ となる。

根の位置をもう少し具体的に確認するため,
$D(\lambda)$ の符号を調べる。
$\Delta t>0$ とすると
\begin{align}
  D(-1) &= (-1)^2 + 4\Delta t(-1) - 1
        = -4\Delta t < 0, \\
  D(0)  &= -1 < 0, \\
  D(1)  &= 1 + 4\Delta t - 1
        = 4\Delta t > 0.
\end{align}
したがって,連続性より
\[
  0 < \lambda_1 < 1,\qquad \lambda_2 < -1
\]
となる 2 つの実根が存在する。
このうち $\lambda_2$ は必ず $|\lambda_2|>1$ を満たす。

漸化式 \eqref{eq:kadai5_v_recur} の一般解は
\begin{align}
  v_n = C_1 \lambda_1^n + C_2 \lambda_2^n
\end{align}
と書けるので,
初期値 $(v_0,v_1)$ が特別に選ばれて
$C_2=0$ となる場合を除き,
$\lambda_2^n$ の項が支配的になり
\[
  |v_n| \sim |C_2|\,|\lambda_2|^n \to \infty
  \qquad (n\to\infty)
\]
となる。

\subsection{まとめ}

以上から,差分スキーム \eqref{eq:kadai5_scheme} を
\eqref{eq:kadai5_ode} に適用すると,
平衡解からの偏差 $v_n$ は必ず
\[
  |\lambda_2|>1
\]
を持つ固有値成分を含むため,
一般の初期値に対して $n\to\infty$ で発散する。

すなわち,任意の $\Delta t>0$ に対して,
この 2 段法は式 \eqref{eq:kadai5_ode} に対して
\emph{安定ではなく},$u_n$ は必ず発散する。

\subsection{数値解析による結果}

常微分方程式
\begin{align}
  u'(t) = -2u(t) + 1, \qquad u(0)=1
\end{align}
に対して,2 段法
\begin{align}
  u_n = u_{n-2} + 2\Delta t\,(-2u_{n-1}+1)
  \label{eq:kadai5_scheme_num}
\end{align}
を用いて数値解の時間発展を調べた。

時間区間は $[0,10]$ とし,刻み幅
\[
  \Delta t = 0.01,\ 0.05,\ 0.10
\]
の 3 通りについて計算を行った。  
2 段法であるため,$u_0=1$ に加え,$u_1$ は解析解
\[
  u_{\mathrm{exact}}(t)
  = \frac{1}{2} + \frac{1}{2} e^{-2t}
\]
を用いて
\[
  u_1 = u_{\mathrm{exact}}(\Delta t)
\]
と与えた。その後は式 \eqref{eq:kadai5_scheme_num} に従って
$n=2,3,\dots,N$ まで時間発展させた($N\Delta t = 10$)。

図\ref{fig:task5_twostep} に,真値と 3 種類の刻み幅に対する数値解を示す。
縦軸は $u\in[-200,200]$ に固定して描画した。
いずれの刻み幅でも,初期のうちは真値と同様に減衰しているように見えるが,
時間が進むにつれて振幅が増大し,$t\approx 10$ では
$u$ が $\pm 10^2$~$10^5$ のオーダで発散していることが分かる。
特に $\Delta t$ が大きいほど発散が急激であり,
$\Delta t=0.10$ の場合には数値解が短時間で縦軸の表示範囲を大きく超えている。

\begin{figure}[H]
  \centering
  \includegraphics[width=0.7\linewidth]{task5_two_step.png}
  \caption{課題5の2段法
           $u_n = u_{n-2} + 2\Delta t(-2u_{n-1}+1)$
           による式(4.29)の数値解
           ($\Delta t = 0.01,\,0.05,\,0.10$)。}
  \label{fig:task5_twostep}
\end{figure}


$t=10$ における数値解 $u_N$ と真値 $u_{\mathrm{exact}}(10)$ の絶対誤差
\[
  E = \left|u_N - u_{\mathrm{exact}}(10)\right|
\]
を表\ref{tab:kadai5_error} に示す(有効数字 4 桁)。

\begin{table}[H]
  \centering
  \caption{課題5 の 2 段法における $t=10$ での数値解と誤差}
  \begin{tabular}{c|c|c}
    \hline
    $\Delta t$ & $u_N$ & $E = |u_N-u_{\mathrm{exact}}(10)|$ \\
    \hline
    $0.01$ & $1.588\times 10^{2}$ & $1.583\times 10^{2}$ \\
    $0.05$ & $1.753\times 10^{4}$ & $1.753\times 10^{4}$ \\
    $0.10$ & $1.120\times 10^{5}$ & $1.120\times 10^{5}$ \\
    \hline
  \end{tabular}
  \label{tab:kadai5_error}
\end{table}

真値 $u_{\mathrm{exact}}(10)\approx 0.5$ が有界であるのに対し,
どの刻み幅でも $u_N$ は時間とともに急速に増大しており,
解析で得られた「本 2 段法の特性根の一つは常に $|\lambda|>1$ となるため,
一般の初期値に対して数値解は必ず発散する」という結果と
定性的にも定量的にも一致している。

