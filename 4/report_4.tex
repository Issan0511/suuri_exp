\documentclass[a4paper,11pt]{ltjsarticle}
\usepackage{geometry}
\geometry{top=25mm,bottom=25mm,left=25mm,right=25mm}
\usepackage{graphicx}
\usepackage{amsmath,amssymb,bm}
\usepackage{hyperref}
\usepackage{float}
\usepackage{caption}
\usepackage{subcaption}
\usepackage{booktabs}
\usepackage{siunitx}
\usepackage{listings}
\usepackage{xcolor}
\usepackage{physics}
\usepackage{microtype}

\hypersetup{
  pdftitle={数理工学実験レポート},
  pdfauthor={中塚一瑳},
  colorlinks=true,
  linkcolor=blue,
  citecolor=blue,
  urlcolor=blue
}

% -------------------------
% ユーザ定義マクロ
% -------------------------
\newcommand{\StudentID}{1029366161}
\newcommand{\AuthorName}{中塚一瑳}
\newcommand{\CourseName}{数理工学実験}
\newcommand{\ChapterTitle}{第4章(モンテカルロシミュレーション)}
\newcommand{\ExperimentDate}{\today}

% 図・表フォルダパス(必要なら変更)
\newcommand{\graphdir}{graphs}

% 数式用の簡易マクロ
\newcommand{\R}{\mathbb{R}}
% \newcommand{\norm}[1]{\left\lVert#1\right\rVert}

% listings 設定(Julia を例示)
\lstset{
  basicstyle=\ttfamily\footnotesize,
  breaklines=true,
  frame=single,
  numbers=left,
  numberstyle=\tiny,
  language=Python,
  keywordstyle=\color{blue},
  commentstyle=\color{gray},
  stringstyle=\color{teal},
  showstringspaces=false
}

% 実験表題用コマンド
\newcommand{\TaskSection}[2]{%
  \section{#1}%
  \textbf{目的:} #2\par\vspace{6pt}%
}

% キャプションのフォント調整
\captionsetup{font=small,labelfont=bf}

% -------------------------
% 本文
% -------------------------
\title{数理工学実験レポート\\[4pt]\large \ChapterTitle}
\author{学籍番号 1029366161 \quad 中塚一瑳}
\date{\ExperimentDate}

\begin{document}

\maketitle
\begin{abstract}
% 要旨をここに記述
\end{abstract}

\tableofcontents
\clearpage

% \section*{表記・略記}
% 本レポートにおける表記と略記を示す。必要に応じて編集すること。
% \begin{itemize}
%   \item $\R$:実数全体
%   \item $\|\cdot\|$:ユークリッドノルム(明示が必要な場合は添字を用いる)
%   \item 各種パラメータや記号は本文中で定義する
% \end{itemize}

% ===== 例:実験テンプレート =====
\section*{はじめに}
今回は

\section{課題1:疫病の確率的な伝染モデル}

\subsection{問題設定}
1次元拡散方程式
\begin{equation}
  \frac{\partial u}{\partial t} = \frac{\partial^2 u}{\partial x^2}
\end{equation}
を、区間 $[0,L]$($L=10$)上で数値的に解く。初期条件は
\begin{equation}
  u_0(x) = \frac{1}{\sqrt{2\pi}} \exp\left( -\frac{1}{2}(x-5)^2 \right)
\end{equation}
とし、Dirichlet境界条件($u(0,t)=u(L,t)=0$)およびNeumann境界条件($\partial u/\partial x |_{x=0} = \partial u/\partial x |_{x=L} = 0$)の両方について計算を行った。

\subsection{数値解法}
\subsubsection{オイラー陽解法}
時間微分を前進差分、空間2階微分を中心差分で近似すると、オイラー陽解法が得られる:
\begin{equation}
  u_j^{n+1} = u_j^n + c(u_{j-1}^n - 2u_j^n + u_{j+1}^n), \quad c = \frac{\Delta t}{(\Delta x)^2}
\end{equation}
この解法は $c \leq 1/2$ のとき安定である。

\subsubsection{クランク-ニコルソン法}
時間微分を中心差分、空間2階微分を $n$ ステップと $n+1$ ステップの平均で近似すると、クランク-ニコルソン法が得られる:
\begin{equation}
  -\frac{c}{2}u_{j-1}^{n+1} + (1+c)u_j^{n+1} - \frac{c}{2}u_{j+1}^{n+1} = \frac{c}{2}u_{j-1}^n + (1-c)u_j^n + \frac{c}{2}u_{j+1}^n
\end{equation}
この解法は無条件安定であり、各ステップで三重対角行列の連立方程式を解く必要がある。

\subsection{計算結果}
計算パラメータは $N=100$, $\Delta x = 0.1$, $\Delta t = 0.004$($c=0.4$)とした。

\subsubsection{Dirichlet境界条件}
図\ref{fig:dirichlet}にDirichlet境界条件における計算結果を示す。オイラー陽解法とクランク-ニコルソン法の結果はほぼ一致しており、初期のガウス分布が時間とともに拡散し、境界条件により最終的に消滅していく様子が観察できる。

\begin{figure}[H]
  \centering
  \begin{subfigure}{0.45\textwidth}
    \includegraphics[width=\textwidth]{graphs_dirichlet/comparison_t1.png}
    \caption{$t=1$}
  \end{subfigure}
  \begin{subfigure}{0.45\textwidth}
    \includegraphics[width=\textwidth]{graphs_dirichlet/comparison_t2.png}
    \caption{$t=2$}
  \end{subfigure}
  \begin{subfigure}{0.45\textwidth}
    \includegraphics[width=\textwidth]{graphs_dirichlet/comparison_t3.png}
    \caption{$t=3$}
  \end{subfigure}
  \begin{subfigure}{0.45\textwidth}
    \includegraphics[width=\textwidth]{graphs_dirichlet/comparison_t5.png}
    \caption{$t=5$}
  \end{subfigure}
  \caption{Dirichlet境界条件における拡散方程式の数値解}
  \label{fig:dirichlet}
\end{figure}

\subsubsection{Neumann境界条件}
図\ref{fig:neumann}にNeumann境界条件における計算結果を示す。Neumann境界条件では流束がゼロであるため、初期のガウス分布が境界で反射され、最終的には一様分布に近づいていく。

\begin{figure}[H]
  \centering
  \begin{subfigure}{0.45\textwidth}
    \includegraphics[width=\textwidth]{graphs_neumann/comparison_t1.png}
    \caption{$t=1$}
  \end{subfigure}
  \begin{subfigure}{0.45\textwidth}
    \includegraphics[width=\textwidth]{graphs_neumann/comparison_t2.png}
    \caption{$t=2$}
  \end{subfigure}
  \begin{subfigure}{0.45\textwidth}
    \includegraphics[width=\textwidth]{graphs_neumann/comparison_t3.png}
    \caption{$t=3$}
  \end{subfigure}
  \begin{subfigure}{0.45\textwidth}
    \includegraphics[width=\textwidth]{graphs_neumann/comparison_t5.png}
    \caption{$t=5$}
  \end{subfigure}
  \caption{Neumann境界条件における拡散方程式の数値解}
  \label{fig:neumann}
\end{figure}

\subsection{考察}
オイラー陽解法とクランク-ニコルソン法の結果を比較すると、同じパラメータでほぼ同一の解が得られた。これは、使用した $c=0.4 < 0.5$ がオイラー陽解法の安定条件を満たしており、十分小さな時間刻みを使用していることによる。クランク-ニコルソン法は無条件安定であるため、より大きな時間刻みを使用できるという利点がある。

境界条件の違いについては、Dirichlet境界条件では境界で$u=0$が課されるため、物質が境界から流出し最終的に消滅する。一方、Neumann境界条件では境界での流束がゼロであるため、物質は保存され、最終的に一様分布に近づく。


% ===== 複数課題の繰り返し =====
% 以下を必要数コピーして各課題を記述
\TaskSection{課題2:LU分解}{目的を記載}
% ... 同様に節を展開

% ===== 総合比較 =====
\section{アルゴリズム比較}
表や図を用いて複数手法の比較を行う。計算量・精度・安定性・実行時間の観点でまとめる。

% ===== 結論 =====
\section{結論}
本章での結論と今後の課題を箇条書きでまとめる。

% ===== 付録 =====
\appendix
\section{使用コード一覧}
主要スクリプトと入手先を列挙する。



% ===== 参考文献 =====
\section*{参考文献}
\begin{thebibliography}{9}
\bibitem{exp2025} 数理工学実験(2025年度配布資料).
\end{thebibliography}

\end{document}
