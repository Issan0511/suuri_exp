\subsection{原理と方法}
本節では,モンテカルロ法を用いて円周率 $\pi$ を推定する原理と,
実際に用いる計算手順をまとめる.

まず,次の定積分が成り立つことを用いる.
\begin{equation}
  \int_0^1 \frac{4}{1+x^2}\,\mathrm{d}x = \pi .
  \label{eq:int-pi}
\end{equation}
区間 $[0,1]$ 上の一様分布に従う確率変数 $X$ を考えると,
その確率密度関数は
\[
  p_X(x) =
  \begin{cases}
    1 & (0 \le x \le 1),\\
    0 & (\text{それ以外})
  \end{cases}
\]
である.このとき,任意の可積分関数 $g(x)$ に対して
\[
  \mathbb{E}[g(X)]
  = \int_0^1 g(x)\,p_X(x)\,\mathrm{d}x
  = \int_0^1 g(x)\,\mathrm{d}x
\]
が成り立つ.$g(x)=4/(1+x^2)$ とおけば,
式 \eqref{eq:int-pi} は
\begin{equation}
  \pi
  = \int_0^1 \frac{4}{1+x^2}\,\mathrm{d}x
  = \mathbb{E}\!\left[\frac{4}{1+X^2}\right]
  \label{eq:pi-as-expectation}
\end{equation}
と書き換えられる.したがって,
$\pi$ は確率変数 $Y = 4/(1+X^2)$ の期待値として表される.

ここで,$X$ の独立な標本 $X_1,\dots,X_N$ を
一様分布 $\mathcal{U}(0,1)$ に従って生成し,
対応する
\[
  Y_k = \frac{4}{1+X_k^2} \quad (k=1,\dots,N)
\]
を計算する.このとき,$Y_k$ は互いに独立同分布であり,
その期待値は式 \eqref{eq:pi-as-expectation} より $\mathbb{E}[Y_k]=\pi$ である.
よって,標本平均
\begin{equation}
  \hat{\pi}_N
  = \frac{1}{N}\sum_{k=1}^N Y_k
  = \frac{1}{N}\sum_{k=1}^N \frac{4}{1+X_k^2}
  \label{eq:mc-estimator}
\end{equation}
は $\pi$ の推定量として用いることができる.
大数の法則により,$N\to\infty$ の極限で
\[
  \hat{\pi}_N \to \pi \quad \text{(ほとんど確実に)}
\]
が成り立つため,式 \eqref{eq:mc-estimator} は
モンテカルロ法による円周率の一貫した推定量になっている.
さらに中心極限定理から,
$N$ が十分大きいとき $\hat{\pi}_N$ は平均 $\pi$,
分散が $O(N^{-1})$ の正規分布に近づき,
推定誤差の大きさは概ね $N^{-1/2}$ のオーダーで減少する.

本実験では,サンプル数 $N$ を
\[
  N \in \{10^2,\,10^3,\,10^5,\,10^7\}
\]
とし,各 $N$ について独立な試行を $10$ 回行う.
各試行で一様乱数列 $\{X_k\}_{k=1}^N$ を生成し,
式 \eqref{eq:mc-estimator} により推定値 $\hat{\pi}_N$ を求める.
得られた $10$ 個の $\hat{\pi}_N$ から,$N$ ごとに
\begin{itemize}
  \item 推定値の平均(系統誤差の確認),
  \item 推定値の標準偏差(統計誤差の大きさ)
\end{itemize}
を算出し,サンプル数 $N$ の増加に伴う精度向上の様子を評価する.

\subsection{実装上の工夫}
本実験ではサンプル数 $M$ を $10, 10^3, 10^5, 10^7$ と増加させながら,
式 (5.10) に基づくモンテカルロ積分を行う.
計算負荷が大きくなるため,以下の点に留意して実装した.

\begin{itemize}
  \item \textbf{完全ベクトル化による高速化}  
  Python の for ループは極端に遅いため,
  乱数生成から積分評価
  \[
    \frac{4}{1 + X^2}
  \]
  の計算までを NumPy の配列演算として一度に実行した.  
  これにより Python レベルのループを排除し,
  C 実装による高速な要素演算が可能になる.

  \item \textbf{形状 (10, M) の乱数生成}  
  10 回の独立試行を行うため,
  毎回乱数を生成するのではなく,
  あらかじめ形状 $(10, M)$ の乱数行列を生成した.  
  これにより,
  乱数生成コストを 1 回にまとめつつ,
  同じ処理を 10 行に対して並列に実行できる.

  \item \textbf{in-place 演算の利用}  
  NumPy の in-place 演算(例:\verb|s *= s|, \verb|np.reciprocal(out=s)|)
  を積極的に利用することで,
  中間配列の生成を避け,  
  メモリ消費とメモリアロケーション時間を削減した.  
  特に $M = 10^7$ のような大規模計算では,
  配列の再生成がボトルネックになるため有効である.

  \item \textbf{標本平均の同時計算}  
  10 行分の計算結果に対して,
  NumPy の \verb|mean(axis=1)| を用いて
  一度に標本平均を求めることで,
  試行ごとのループ処理を回避した.
\end{itemize}

上記の工夫により,大規模サンプルに対しても
純粋 Python 実装と比べて大幅に高速な実行が可能となった.


\subsection{結果}
本実験では,各サンプル数 $M$ について
10 回の独立試行により推定値 $\hat{\pi}$ を求め,
その平均と分散を算出した.  
得られた結果を以下に示す.
\begin{table}[h]
    \centering
    \caption{モンテカルロ法による円周率推定結果}
    \label{tab:pi-estimation}
    \begin{tabular}{cccc}
        \hline
        サンプル数 $M$ & 推定値の平均 & 分散 & 相対誤差 \\
        \hline
        $10$ & $3.1589$ & $1.3593 \times 10^{-2}$ & $0.55\%$ \\
        $10^3$ & $3.1332$ & $4.3925 \times 10^{-4}$ & $0.27\%$ \\
        $10^5$ & $3.1410$ & $3.6470 \times 10^{-6}$ & $0.02\%$ \\
        $10^7$ & $3.1415$ & $2.6389 \times 10^{-8}$ & $0.003\%$ \\
        \hline
    \end{tabular}
\end{table}

以上の結果より,
サンプル数 $M$ の増加に伴い推定値が真値 $\pi$ に収束し,
分散が概ね $O(M^{-1})$ で減少している様子が確認できる.

\subsubsection{考察}
表{tab:pi-estimation}に示す通り,分散が概ね $O(M^{-1})$ で減少していることから,
分散が $\displaystyle \frac{1}{M}\Bigl[\langle f^{2}(X)\rangle - \langle f(X)\rangle^{2}\Bigr]$ であることは裏付けられるが、$\Bigl[\langle f^{2}(X)\rangle - \langle f(X)\rangle^{2}\Bigr]$の値を求めるため、$f(X)=\frac{4}{1+X^{2}}$の2乗の平均値を求める必要がある。
そこで、追加のコードを実装し、関数pi\_pi\_monte\_carlo\_integrationを実装し、M=10,000,000で2乗の平均値を求めた。
その結果、 10.2826であり、真値$\pi^{2} \approx 9.8696$との差は0.4130であった。シミュレーションによる$\Bigl[\langle f^{2}(X)\rangle - \langle f(X)\rangle^{2}\Bigr]$の値はが0.3~0.5であったことから、確かに分散が$\displaystyle \frac{1}{M}\Bigl[\langle f^{2}(X)\rangle - \langle f(X)\rangle^{2}\Bigr]$であることが確かめられた。
