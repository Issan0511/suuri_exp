
\subsection{1. メトロポリス法が詳細つり合い条件を満たすことの証明}

メトロポリス法では,現状態 $X$ から次状態候補 $Y$ を
対称な確率分布 $Q(X,Y)=Q(Y,X)$ に従って提案し,
受理確率
\[
A(X\to Y)=\min\left(1,\frac{W(Y)}{W(X)}\right)
\]
で遷移する\cite{exp2025}.
したがって遷移確率は
\[
P(X\to Y)=Q(X,Y)A(X\to Y)
\]
である.

詳細つり合い条件
\[
P(X\to Y)W(X)=P(Y\to X)W(Y)
\tag{5.19}
\label{eq:detbal}
\]
を示すため,$W(Y)\ge W(X)$ と $W(Y)< W(X)$ の場合に分ける.

\paragraph{(i) $W(Y)\ge W(X)$ の場合}
このとき
\[
A(X\to Y)=1,\qquad
A(Y\to X)=\frac{W(X)}{W(Y)} .
\]
よって
\[
P(X\to Y)W(X)
=Q(X,Y)\cdot 1\cdot W(X),
\]
\[
P(Y\to X)W(Y)
=Q(Y,X)\frac{W(X)}{W(Y)}W(Y)
=Q(X,Y)W(X),
\]
となり \eqref{eq:detbal} が成り立つ.

\paragraph{(ii) $W(Y)< W(X)$ の場合}
このとき
\[
A(X\to Y)=\frac{W(Y)}{W(X)},\qquad
A(Y\to X)=1 .
\]
したがって
\[
P(X\to Y)W(X)
=Q(X,Y)\frac{W(Y)}{W(X)}W(X)
=Q(X,Y)W(Y),
\]
\[
P(Y\to X)W(Y)
=Q(Y,X)\cdot 1\cdot W(Y)
=Q(X,Y)W(Y),
\]
となる.

以上より任意の $X,Y$ について
\[
P(X\to Y)W(X)=P(Y\to X)W(Y)
\]
が成立し,メトロポリス法は詳細つり合い条件を満たす.


\subsection{2. 熱浴法が詳細つり合い条件を満たすことの証明}

状態を $X=(x_1,x_2,\ldots)$ とし,時刻 $t$ において添字 $i$ が
状態に依存しない確率 $p_i$ で選ばれるとする\cite{exp2025}.
熱浴更新では $i$ 番目の成分だけを更新し,
それ以外は不変である.

$X$ と $Y$ が 1 箇所だけ異なる場合
\[
X=(x_1,\ldots,x_i,\ldots),\qquad
Y=(x_1,\ldots,y_i,\ldots)
\]
となる.このとき熱浴法の条件付き確率は
\[
P(x_i'=y_i)
=
\frac{W(Y)}{\sum_{x_i}W(x_1,\ldots,x_i,\ldots)}
\equiv \frac{W(Y)}{Z_i},
\]
ここで $Z_i$ は $i$ 番目のみを動かしたときの重みの総和で,
$X$ と $Y$ に対して共通である.

よって遷移確率は
\[
P(X\to Y)=p_i\frac{W(Y)}{Z_i},\qquad
P(Y\to X)=p_i\frac{W(X)}{Z_i}.
\]

したがって
\[
P(X\to Y)W(X)
= p_i\frac{W(Y)}{Z_i}W(X),
\]
\[
P(Y\to X)W(Y)
= p_i\frac{W(X)}{Z_i}W(Y),
\]
より両辺は一致する.

$X$ と $Y$ が 2 箇所以上異なる場合には
熱浴法では 1 ステップで遷移できないため
\[
P(X\to Y)=P(Y\to X)=0
\]
となり,この場合も自明に詳細つり合いが成り立つ.

以上より熱浴法の遷移確率は詳細つり合い条件
\[
P(X\to Y)W(X)=P(Y\to X)W(Y)
\]
を満たす.


\subsection{3. 詳細つり合い条件 (5.19) からつり合い条件 (5.17) が従うことの証明}

詳細つり合い条件
\[
P(X\to Y)W(X)=P(Y\to X)W(Y)
\tag{5.19}
\]
がすべての $X,Y$ について成り立つと仮定する.

このとき任意の $X$ について
\[
\sum_Y P(Y\to X)W(Y)
\]
を計算する:
\[
\sum_Y P(Y\to X)W(Y)
=
\sum_Y P(X\to Y)W(X),
\]
ここで右辺の等号は詳細つり合い (5.19) より従う.

また遷移確率の和は $\sum_Y P(X\to Y)=1$ であるから,
\[
\sum_Y P(Y\to X)W(Y)
=
W(X)\sum_Y P(X\to Y)
=
W(X).
\]

よってつり合い条件
\[
W(X)=\sum_Y P(Y\to X)W(Y)
\tag{5.17}
\]
が示された.
詳細つり合いが成立すれば,$W$ はマルコフ過程の定常分布となる.

