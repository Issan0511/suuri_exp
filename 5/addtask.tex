
\[
\min_{x\in\mathbb{R}^n}\ \frac12 x^\top Qx + c^\top x
\quad \text{subject to}\quad Ax \le b,
\]
ただし $Q$ は対称半正定値行列とする。

\subsection{課題1}

\subsection{課題2}

\subsection{課題3}


本課題では, University of California, Irvine の\href{https://www.math.uci.edu/~qnie/Publications/NumericalOptimization.pdf}{Numerical Optimization} のチャプター16.6 "INTERIOR-POINT METHODS"を参考に,内点法を実装した。
文献で扱われている $Ax\ge b$ の形式に合わせるため,
\[
Ax\le b \;\;\Longleftrightarrow\;\; (-A)x \ge (-b)
\]
と左右反転し,$\tilde A=-A,\ \tilde b=-b$ を用いて
\[
\tilde A x \ge \tilde b
\]
として以下を進める(記号の簡略化のため,再び $A,b$ と書く)。

制約 $Ax\ge b$ に対し,スラック変数 $y\ge0$ を導入すると,
\[
Ax-y-b=0
\]
と書ける。これに対する KKT 条件は
\[
\begin{cases}
Qx-A^\top\lambda+c=0,\\
Ax-y-b=0,\\
y_i\lambda_i=0\quad (i=1,\dots,m),\\
y\ge0,\ \lambda\ge0
\end{cases}
\]
である。問題は凸であるため,これらの条件は必要十分である。

相補性条件 $y_i\lambda_i=0$ は境界条件であり直接扱いにくいため,
補完性の平均尺度
\[
\mu=\dfrac{y^\top\lambda}{m}
\]
を導入し,摂動 KKT 系
\[
F(x,y,\lambda;\sigma\mu)=
\begin{pmatrix}
Qx-A^\top\lambda+c\\
Ax-y-b\\
Y\Lambda e-\sigma\mu e
\end{pmatrix}
=0
\]
を考える。ここで
$Y=\mathrm{diag}(y_1,\dots,y_m)$,
$\Lambda=\mathrm{diag}(\lambda_1,\dots,\lambda_m)$,
$e=(1,\dots,1)^\top$,
$\sigma\in(0,1)$ は固定定数である。
$\sigma\mu>0$ の解は central path を成し,$\sigma\mu\to0$ により
元の KKT 解(最適解)に収束する。

現在の反復点 $z=(x,y,\lambda)$ において,
Newton 法により
\[
F(z+\Delta z)\approx F(z)+J_F(z)\Delta z=0
\]
を解く。ヤコビ行列は
\[
J_F(z)=
\begin{pmatrix}
Q & 0 & -A^\top\\
A & -I & 0\\
0 & \Lambda & Y
\end{pmatrix}
\]
であり,Newton 方程式は
\[
\begin{pmatrix}
Q & 0 & -A^\top\\
A & -I & 0\\
0 & 0 & Y
\end{pmatrix}
\begin{pmatrix}
\Delta x\\
\Delta y\\
\Delta\lambda
\end{pmatrix}
=
\begin{pmatrix}
-r_d\\
-r_p\\
-\Lambda Y e+\sigma\mu e
\end{pmatrix}
\]
となる。ここで
\[
r_d=Qx-A^\top\lambda+c,\qquad
r_p=Ax-y-b
\]
はそれぞれ双対残差および主残差である。

得られた方向に対して
\[
(x^+,y^+,\lambda^+)=(x,y,\lambda)+\alpha(\Delta x,\Delta y,\Delta\lambda)
\]
と更新する。ただし $y^+>0,\ \lambda^+>0$ を保つように
$\alpha\in(0,1]$ を選ぶ。

以上を反復し,
$\|r_p\|,\ \|r_d\|,\ \mu$ が十分小さくなった時点で停止する。
凸性より,停止点 $x$ は元の制約 $Ax\le b$ を満たす
二次計画問題の最適解である。
