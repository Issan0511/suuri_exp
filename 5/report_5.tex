\documentclass[a4paper,11pt]{ltjsarticle}
\usepackage{geometry}
\geometry{top=25mm,bottom=25mm,left=25mm,right=25mm}
\usepackage{graphicx}
\usepackage{amsmath,amssymb,bm}
\usepackage{hyperref}
\usepackage{float}
\usepackage{caption}
\usepackage{subcaption}
\usepackage{booktabs}
\usepackage{siunitx}
\usepackage{listings}
\usepackage{xcolor}
\usepackage{physics}
\usepackage{microtype}

\hypersetup{
  pdftitle={数理工学実験レポート},
  pdfauthor={中塚一瑳},
  colorlinks=true,
  linkcolor=blue,
  citecolor=blue,
  urlcolor=blue
}

% -------------------------
% ユーザ定義マクロ
% -------------------------
\newcommand{\StudentID}{1029366161}
\newcommand{\AuthorName}{中塚一瑳}
\newcommand{\CourseName}{数理工学実験}
\newcommand{\ChapterTitle}{第6章(連続最適化)}
\newcommand{\ExperimentDate}{\today}

% 図・表フォルダパス(必要なら変更)
\newcommand{\graphdir}{graphs}

% 数式用の簡易マクロ
\newcommand{\R}{\mathbb{R}}
% \newcommand{\norm}[1]{\left\lVert#1\right\rVert}

% listings 設定(Julia を例示)
\lstset{
  basicstyle=\ttfamily\footnotesize,
  breaklines=true,
  frame=single,
  numbers=left,
  numberstyle=\tiny,
  language=Python,
  keywordstyle=\color{blue},
  commentstyle=\color{gray},
  stringstyle=\color{teal},
  showstringspaces=false
}



% キャプションのフォント調整
\captionsetup{font=small,labelfont=bf}

% -------------------------
% 本文
% -------------------------
\title{数理工学実験レポート\\[4pt]\large \ChapterTitle}
\author{学籍番号 1029366161 \quad 中塚一瑳}
\date{\ExperimentDate}

\begin{document}

\maketitle

\tableofcontents
\clearpage



% ===== 例:実験テンプレート =====
\section*{はじめに}
今回は連続最適化の様々な手法を用いて、与えられた関数の最小値を求める課題に取り組む。


\section{課題15:ニュートン法および二分法による多項式の零点計算}


本課題では,多項式
\begin{align}
f(x) = x^{3}+2x^{2}-5x-6
\label{eq:f}
\end{align}
の零点を数値的に求める.
まず関数のグラフを描画して零点の存在を確認し,その後,二分法およびニュートン法を用いて零点を計算する.

\subsection{原理と方法}
\subsubsection{二分法}
二分法は,区間 $[a,b]$ において $f(a)$ と $f(b)$ の符号が異なるとき,その区間内に零点が存在することを利用した反復法である.
中点 $c=(a+b)/2$ を取り,$f(c)$ の符号に応じて零点を含む半区間に更新する操作を繰り返すことで,区間幅を徐々に縮小し零点へ収束させる.
零点を挟む区間が与えられれば必ず収束するが,収束速度は比較的遅い.

\subsubsection{ニュートン法}
ニュートン法は,関数をある点 $x_k$ の周りで一次近似し,その接線と $x$ 軸の交点を次の近似値とする方法である.
反復公式は
\begin{align}
x_{k+1}=x_k-\frac{f(x_k)}{f'(x_k)}
\end{align}
で与えられる.
一般に収束は速いが,初期値の選び方によっては収束しない場合がある.
本課題で用いた導関数は
\begin{align}
f'(x)=3x^{2}+4x-5
\end{align}
である.

\subsection{実験方法}
Python を用いて $x\in[-10,10]$ の範囲で $f(x)$ を描画し,グラフから零点が
$x\approx -3,-1,2$ 付近に存在することを確認した.
二分法では,それぞれの零点を挟む区間として
\[
[-4,-2],\ [-2,0],\ [1,3]
\]
を与えた.
ニュートン法では初期値として
\[
x_0=-2.5,\ -0.5,\ 1.5
\]
を用いた.
停止条件は $|f(x)|\le 10^{-10}$ とした.

\subsection{結果}
\subsubsection{関数のグラフ}
\begin{figure}[h]
  \centering
  \includegraphics[width=0.7\linewidth]{task15.png}
  \caption{$f(x)=x^{3}+2x^{2}-5x-6$ のグラフ}
  \label{fig:task15}
\end{figure}

\subsubsection{零点}
数値計算によって得られた零点を表\ref{tab:roots}に示す.
\begin{table}[h]
\centering
\caption{二分法およびニュートン法で求めた零点}
\begin{tabular}{l|c|c}
\hline
手法 & 近似解 $x$ & 残差 $f(x)$ \\
\hline
二分法 & $-3.000$ & $0$ \\
二分法 & $-1.000$ & $0$ \\
二分法 & $ 2.000$ & $0$ \\
\hline
ニュートン法 & $-3.000$ & $0$ \\
ニュートン法 & $-1.000$ & $0$ \\
ニュートン法 & $ 2.000$ & $1.421\times 10^{-14}$ \\
\hline
\end{tabular}
\label{tab:roots}
\end{table}

\subsection{考察}
$f(x)=0$ は因数分解により
\[
f(x)=(x+3)(x+1)(x-2)
\]
と書け,解析解は $x=-3,-1,2$ である.
数値計算によって得られた零点はこれらと一致しており,手法が正しく実装されていることが確認できる.
ニュートン法において残差が完全に $0$ とならない場合があるのは,浮動小数点演算誤差によるものである.



\section{課題16:最急降下法およびニュートン法による停留点計算}

\subsection{原理と方法}
本課題では,次の関数
\begin{align}
f(x) := \frac{1}{3}x^3 - x^2 - 3x + \frac{5}{3}
\end{align}
の停留点($f'(x)=0$ を満たす点)を,(a) 最急降下法,(b) ニュートン法により数値的に求める.
まず微分は
\begin{align}
f'(x) &= x^2 - 2x - 3 = (x-3)(x+1),\\
f''(x) &= 2x - 2
\end{align}
である.従って停留点は解析的には $x=-1,\,3$ の2点である.

\subsubsection{(a) 最急降下法}
1次元では勾配 $\nabla f(x)$ は $f'(x)$ に一致するため,最急降下法は
\begin{align}
x_{k+1} = x_k - t_k f'(x_k), \qquad t_k = \frac{1}{k+1}
\end{align}
で与えられる.初期点は $x_0=1/2$ とする.停止判定は $|f'(x_k)|\le \varepsilon$($\varepsilon=10^{-8}$)とし,最大反復回数も設ける.

\subsubsection{(b) ニュートン法}
停留点探索($f'(x)=0$ の零点探索)としてのニュートン法は
\begin{align}
x_{k+1} = x_k - t_k\frac{f'(x_k)}{f''(x_k)}, \qquad t_k=1
\end{align}
で与えられる.初期点は $x_0=5$ とし,(a) と同様に $|f'(x_k)|\le \varepsilon$ を停止判定とした.なお,本問題では $f''(x)=0$(すなわち $x=1$)で更新が不能となるため,実装では $f''(x_k)=0$ の場合を例外として扱う.

\subsection{実装}
Python により $f,f',f''$ をそれぞれ関数として実装し,(a) と (b) の更新式をそのまま反復した.
各反復で $(x_k, f(x_k), |f'(x_k)|)$ を履歴として保存し,収束挙動の可視化に用いた.

\subsection{結果}
(a) 最急降下法($x_0=0.5,\ t_k=1/(k+1)$)および (b) ニュートン法($x_0=5,\ t_k=1$)の結果を表\ref{tab:task16_result}に示す.
両手法とも停留点 $x^\ast=3$ に収束した(もう一つの停留点 $x=-1$ には到達しなかった).\\
\\



\begin{table}[h]
\centering
\caption{課題16の計算結果($\varepsilon=10^{-8}$)}
\label{tab:task16_result}
\begin{tabular}{lcccc}
\hline
手法 & 初期値 $x_0$ & 近似解 $x^\ast$ & $f(x^\ast)$ & 反復回数 $k$ \\
\hline
最急降下法 & 0.5000 & 3.000 & -7.333 & 193 \\
ニュートン法 & 5.000 & 3.000 & -7.333 & 5 \\
\hline
\end{tabular}
\end{table}

両方法による反復点の推移を,関数 $y=f(x)$ 上に重ねて図\ref{fig:task16_iter}に示す.

\begin{figure}[h]
 \centering
 \includegraphics[width=0.7\linewidth]{task16.png}
 \caption{関数 $y=f(x)$ 上における反復点の推移.
 青丸は最急降下法,橙四角はニュートン法による反復点を表す.}
 \label{fig:task16_iter}
\end{figure}

\subsection{考察}
図\ref{fig:task16_iter}より,最急降下法では反復点が谷に沿って
緩やかに移動しており,ステップサイズ $t_k=1/(k+1)$ が減少することで
後半の収束が遅くなっていることが視覚的に確認できる.
一方,ニュートン法では局所的な2次近似に基づき更新が行われるため,
停留点近傍では大きなジャンプを伴い,極めて高速に収束している.



\section{課題17:勾配およびヘッセ行列の解析的導出と実装との対応}

本課題では,次の関数
\begin{align}
f(x_0,x_1)
= x_0^2 + e^{x_0} + x_1^4 + x_1^2 - 2x_0x_1 + 3
\end{align}
について,勾配およびヘッセ行列を解析的に求め,それを実装したコードとの対応を示す.

\subsection{微分結果}

まず,各変数による1階微分は
\begin{align}
\frac{\partial f}{\partial x_0}
&= 2x_0 + e^{x_0} - 2x_1, \\
\frac{\partial f}{\partial x_1}
&= 4x_1^3 + 2x_1 - 2x_0
\end{align}
である.従って勾配ベクトルは
\begin{align}
\nabla f(x)
=
\begin{pmatrix}
2x_0 + e^{x_0} - 2x_1 \\
4x_1^3 + 2x_1 - 2x_0
\end{pmatrix}
\end{align}
と表される\cite{exp2025}.

次に,2階微分よりヘッセ行列は
\begin{align}
\nabla^2 f(x)
=
\begin{pmatrix}
\displaystyle \frac{\partial^2 f}{\partial x_0^2}
&
\displaystyle \frac{\partial^2 f}{\partial x_0 \partial x_1}
\\[6pt]
\displaystyle \frac{\partial^2 f}{\partial x_1 \partial x_0}
&
\displaystyle \frac{\partial^2 f}{\partial x_1^2}
\end{pmatrix}
=
\begin{pmatrix}
2 + e^{x_0} & -2 \\
-2 & 12x_1^2 + 2
\end{pmatrix}
\end{align}
となる.

\subsection{コードとの対応}

上記の解析結果は,以下の Python コードとして実装されている.

\begin{itemize}
 \item 目的関数
\begin{lstlisting}[language=Python]
f = x0**2 + exp(x0) + x1**4 + x1**2 - 2*x0*x1 + 3
\end{lstlisting}

 \item 勾配
\begin{lstlisting}[language=Python]
g0 = 2*x0 + exp(x0) - 2*x1
g1 = 4*x1**3 + 2*x1 - 2*x0
g  = [g0, g1]
\end{lstlisting}

 \item ヘッセ行列
\begin{lstlisting}[language=Python]
h00 = 2 + exp(x0)
h01 = -2
h10 = -2
h11 = 12*x1**2 + 2
H = [[h00, h01],
     [h10, h11]]
\end{lstlisting}
\end{itemize}

これにより,関数値・勾配・ヘッセ行列が理論式と一致して計算されていることが確認できる.


\section{課題18:最急降下法およびニュートン法による2変数関数の最小化}

\subsection{原理と方法}
本課題では,2変数関数
\begin{align}
f(x_0,x_1)
= x_0^2 + e^{x_0} + x_1^4 + x_1^2 - 2x_0x_1 + 3
\end{align}
を最小化する.勾配およびヘッセ行列は
\begin{align}
\nabla f(x)
&=
\begin{pmatrix}
2x_0 + e^{x_0} - 2x_1\\
4x_1^3 + 2x_1 - 2x_0
\end{pmatrix},\\
\nabla^2 f(x)
&=
\begin{pmatrix}
2+e^{x_0} & -2\\
-2 & 12x_1^2+2
\end{pmatrix}
\end{align}
である.更新は
\begin{align}
x_{k+1}=x_k+t_k d_k
\end{align}
とし,方向 $d_k$ は (a) 最急降下法 $d_k=-\nabla f(x_k)$,
(b) ニュートン法 $\nabla^2 f(x_k)d_k=-\nabla f(x_k)$ の解として定める\cite{exp2025}.

ステップ幅 $t_k$ はバックトラック法(Armijo 条件)で決定した\cite{exp2025}.すなわち,
$t\leftarrow t_{\mathrm{init}}$ から開始し,
\begin{align}
f(x_k+t d_k)\le f(x_k)+\xi t\,\langle \nabla f(x_k), d_k\rangle
\end{align}
を満たすまで $t\leftarrow \rho t$ により縮小する.

\subsection{実験方法}
初期点は $x_0=(1,1)^{\mathsf T}$ とし,共通パラメータを
\begin{align}
\xi=1.0\times 10^{-4},\qquad
\rho=0.5,\qquad
t_{\mathrm{init}}=1
\end{align}
とした.停止条件は $\|\nabla f(x_k)\|\le 1.0\times 10^{-6}$ とし,
最大反復回数は $1000$ とした.
各反復で $(x_k,f(x_k))$ を保存し,$x_0$--$x_1$ 平面上の等高線図に反復点列を重ねて可視化した
(図\ref{fig:task18_traj}).

\subsection{結果}
(a) バックトラック法付き最急降下法および (b) バックトラック法付きニュートン法の収束結果を
表\ref{tab:task18_result}に示す.両手法は同一の解に収束し,
ニュートン法の方が少ない反復回数で収束した.

\begin{table}[h]
\centering
\caption{課題18の計算結果(初期点 $x_0=(1,1)^{\mathsf T}$)}
\label{tab:task18_result}
\begin{tabular}{lcccc}
\hline
手法 & $x^\ast_0$ & $x^\ast_1$ & $f(x^\ast)$ & 反復回数 $k$ \\
\hline
最急降下法+BT & $-0.7335$ & $-0.4933$ & $3.597$ & $31$ \\
ニュートン法+BT & $-0.7335$ & $-0.4933$ & $3.597$ & $6$ \\
\hline
\end{tabular}
\end{table}

table\ref{tab:task18_result}より,ニュートン法は最急降下法に比べて訳5分の1の反復回数で収束している.
\\
\\
反復点列を等高線図上に重ねた結果を図\ref{fig:task18_traj}に示す.


\begin{figure}[h]
 \centering
 \includegraphics[width=0.75\linewidth]{task18.png}
 \caption{$x_0$--$x_1$ 平面における等高線図と反復点列.
 〇が最急降下法+BT,□がニュートン法+BTの反復点列を表す.}
 \label{fig:task18_traj}
\end{figure}

\subsection{考察}
図\ref{fig:task18_traj}より,最急降下法では反復点同士を結ぶ線分が等高線に対して直交しているのに対し,ニュートン法では直交していないものの、あたかも低い場所を知っているかのように、等高線の谷に沿って効率的に移動している様子がわかる.
これは,ニュートン法がヘッセ行列を用いて関数の曲率を考慮しているためであり,その結果として少ない反復回数で収束していると考えられる.

\section{課題19:総合演習}

\subsection{原理と方法}
本課題では,$x=(x_0,x_1)^{\mathsf T}\in\mathbb{R}^2$ に対し
\begin{align}
f(x)=\sum_{i=0}^{2} f_i(x)^2,\qquad
f_i(x)=y_i-x_0\left(1-x_1^{i+1}\right)
\end{align}
を最小化する.定数は $y_0=1.5,\ y_1=2.25,\ y_2=2.625$ とする(最適値は $0$,最適解は $x^\ast=(3,0.5)^{\mathsf T}$).

$f_i$ の勾配およびヘッセ行列は
\begin{align}
\nabla f_i(x)
&=
\begin{pmatrix}
-1+x_1^{i+1}\\
x_0(i+1)x_1^i
\end{pmatrix},\\
\nabla^2 f_0(x)
&=
\begin{pmatrix}
0 & 1\\
1 & 0
\end{pmatrix},\qquad
\nabla^2 f_i(x)
=
\begin{pmatrix}
0 & (i+1)x_1^i\\
(i+1)x_1^i & x_0(i+1)i\,x_1^{i-1}
\end{pmatrix}\ (i=1,2)
\end{align}
である.よって,勾配・ヘッセ行列は課題18と同様に
\begin{align}
\nabla f(x) &= 2\sum_{i=0}^{2} f_i(x)\nabla f_i(x),\\
\nabla^2 f(x) &= 2\sum_{i=0}^{2}\left( f_i(x)\nabla^2 f_i(x)+\nabla f_i(x)\nabla f_i(x)^{\mathsf T}\right)
\end{align}
より計算できる.更新式 $x_{k+1}=x_k+t_k d_k$,方向の選び方(最急降下法/ニュートン法),およびバックトラック法(Armijo 条件)によるステップ幅決定は課題18と同様である.

\subsection{実験方法}
初期点は $x_0=(2,0)^{\mathsf T}$ とし,共通パラメータを
\begin{align}
\xi=1.0\times 10^{-4},\qquad
\rho=0.5,\qquad
t_{\mathrm{init}}=1
\end{align}
とした.停止条件は $\|\nabla f(x_k)\|\le 1.0\times 10^{-6}$ とし,
最大反復回数は $1000$ とした.各反復で $(x_k,f(x_k))$ を保存し,$x_0$--$x_1$ 平面上の等高線図に反復点列を重ねて可視化した(図\ref{fig:task19_traj}).

\subsection{結果}
(a) バックトラック法付き最急降下法および (b) バックトラック法付きニュートン法の収束結果を表\ref{tab:task19_result}に示す.両手法はほぼ同一の解に収束し,ニュートン法の方が少ない反復回数で収束した.

\begin{table}[h]
\centering
\caption{課題19の計算結果(初期点 $x_0=(2,0)^{\mathsf T}$)}
\label{tab:task19_result}
\begin{tabular}{lcccc}
\hline
手法 & $x^\ast_0$ & $x^\ast_1$ & $f(x^\ast)$ & 反復回数 $k$ \\
\hline
最急降下法+BT & $2.99999781$ & $0.49999946$ & $7.68\times 10^{-13}$ & $791$ \\
ニュートン法+BT & $3.00000013$ & $0.50000005$ & $9.47\times 10^{-15}$ & $5$ \\
\hline
\end{tabular}
\end{table}

表\ref{tab:task19_result}より,ニュートン法は最急降下法に比べて大幅に少ない反復回数で収束している.反復点列を等高線図上に重ねた結果を図\ref{fig:task19_traj}に示す.

\begin{figure}[h]
 \centering
 \includegraphics[width=0.75\linewidth]{task19.png}
 \caption{$x_0$--$x_1$ 平面における等高線図と反復点列.
 〇が最急降下法+BT,□がニュートン法+BTの反復点列を表す.}
 \label{fig:task19_traj}
\end{figure}

\subsection{考察}
図\ref{fig:task19_traj}より,最急降下法では最適解近傍でジグザグに進む傾向が見られ,収束までに多くの反復を要した.一方,ニュートン法はヘッセ行列により曲率情報を用いるため,より適切な方向・スケールで更新でき,少ない反復回数で効率的に収束したと考えられる.





\section{追加課題 :制約付き凸最適化問題}

\[
\min_{x\in\mathbb{R}^n}\ \frac12 x^\top Qx + c^\top x
\quad \text{subject to}\quad Ax \le b,
\]
ただし $Q$ は対称半正定値行列とする。
\subsection{課題1}
二次計画問題として定式化することが自然な問題の一つに、最小二乗法による制約付き回帰問題がある。
与えられたデータ点 $(a_i,y_i)\in\mathbb{R}^n\times\mathbb{R}$ $(i=1,\dots,m)$ に対し、
\[
\min_{x\in\mathbb{R}^n}\ \frac12 \sum_{i=1}^m (y_i - a_i^\top x)^2
\quad \text{subject to}\quad Ax \le b
\]
を解くことで、線形モデル $y \approx a^\top x$ のパラメータ $x$ を求めることができる。
この問題は、行列 $A\in\mathbb{R}^{m\times n}$、ベクトル $c\in\mathbb{R}^m$ を用いて
\[\min_{x\in\mathbb{R}^n}\ \frac12 \|Ax - y\|^2 = \min_{x\in\mathbb{R}^n}\ \frac12 x^\top (A^\top A) x - y^\top A x + \frac12 y^\top y\]
と変形でき、$Q=A^\top A$、$c=-A^\top y$ とすれば二次計画問題の形に帰着できる。
\subsection{課題2}
課題3で後ほど説明するように、Qが半正定値である場合はKKT条件と内点法を用いて最適解を求めることができる。しかし、Qが正定値でない、例えばQの固有値に1つでも負の値がある場合は、KKT条件を満たす点が最適解とは限らず、NP困難な問題となることが知られている。\cite{pardalos1991}\cite{sahni1974}。
\subsection{課題3}


本課題では, University of California, Irvine の\href{https://www.math.uci.edu/~qnie/Publications/NumericalOptimization.pdf}{Numerical Optimization} のチャプター16.6 "INTERIOR-POINT METHODS"を参考に,内点法を実装した。
文献で扱われている $Ax\ge b$ の形式に合わせるため,
\[
Ax\le b \;\;\Longleftrightarrow\;\; (-A)x \ge (-b)
\]
と左右反転し,$\tilde A=-A,\ \tilde b=-b$ を用いて
\[
\tilde A x \ge \tilde b
\]
として以下を進める(記号の簡略化のため,再び $A,b$ と書く)。

制約 $Ax\ge b$ に対し,スラック変数 $y\ge0$ を導入すると,
\[
Ax-y-b=0
\]
と書ける。これに対する KKT 条件は
\[
\begin{cases}
Qx-A^\top\lambda+c=0,\\
Ax-y-b=0,\\
y_i\lambda_i=0\quad (i=1,\dots,m),\\
y\ge0,\ \lambda\ge0
\end{cases}
\]
である。問題は凸であるため,これらの条件は必要十分である。

相補性条件 $y_i\lambda_i=0$ は境界条件であり直接扱いにくいため,
補完性の平均尺度
\[
\mu=\dfrac{y^\top\lambda}{m}
\]
を導入し,摂動 KKT 系
\[
F(x,y,\lambda;\sigma\mu)=
\begin{pmatrix}
Qx-A^\top\lambda+c\\
Ax-y-b\\
Y\Lambda e-\sigma\mu e
\end{pmatrix}
=0
\]
を考える。ここで
$Y=\mathrm{diag}(y_1,\dots,y_m)$,
$\Lambda=\mathrm{diag}(\lambda_1,\dots,\lambda_m)$,
$e=(1,\dots,1)^\top$,
$\sigma\in(0,1)$ は固定定数である。
$\sigma\mu>0$ の解は central path を成し,$\sigma\mu\to0$ により
元の KKT 解(最適解)に収束する。

現在の反復点 $z=(x,y,\lambda)$ において,
Newton 法により
\[
F(z+\Delta z)\approx F(z)+J_F(z)\Delta z=0
\]
を解く。ヤコビ行列は
\[
J_F(z)=
\begin{pmatrix}
Q & 0 & -A^\top\\
A & -I & 0\\
0 & \Lambda & Y
\end{pmatrix}
\]
であり,Newton 方程式は
\[
\begin{pmatrix}
Q & 0 & -A^\top\\
A & -I & 0\\
0 & 0 & Y
\end{pmatrix}
\begin{pmatrix}
\Delta x\\
\Delta y\\
\Delta\lambda
\end{pmatrix}
=
\begin{pmatrix}
-r_d\\
-r_p\\
-\Lambda Y e+\sigma\mu e
\end{pmatrix}
\]
となる。ここで
\[
r_d=Qx-A^\top\lambda+c,\qquad
r_p=Ax-y-b
\]
はそれぞれ双対残差および主残差である。

得られた方向に対して
\[
(x^+,y^+,\lambda^+)=(x,y,\lambda)+\alpha(\Delta x,\Delta y,\Delta\lambda)
\]
と更新する。ただし $y^+>0,\ \lambda^+>0$ を保つように
$\alpha\in(0,1]$ を選ぶ。

以上を反復し,
$\|r_p\|,\ \|r_d\|,\ \mu$ が十分小さくなった時点で停止する。
凸性より,停止点 $x$ は元の制約 $Ax\le b$ を満たす
二次計画問題の最適解である。


\subsection{課題4}

本課題では,以下のデータを用いて内点法を実装し,制約付き二次計画問題を解いた。
\[Q=\begin{pmatrix}
2 & 0\\ 
0 & 1
\end{pmatrix},\quad
c=\begin{pmatrix}
-1 \\ -1
\end{pmatrix},\quad
A=\begin{pmatrix}
1 & 1
\end{pmatrix},\quad
b=\begin{pmatrix}
0
\end{pmatrix}.\]

初期点として
\[
x^{(0)} = (0,0)^\top
\]
を用い,スラック変数および双対変数は
\[
y^{(0)} = 1,\qquad \lambda^{(0)} = 1
\]
とした。これらはいずれも正であり,内点法の初期条件を満たしている。

この条件下で課題3のアルゴリズムをaddtask.pyに実装し,反復を行った。
以下が出力結果である。
\[
\mathbf{x}^* = \begin{pmatrix} -2.809 \times 10^{-13} \\ -5.618 \times 10^{-13} \end{pmatrix}
\]
\[
y = \begin{pmatrix} 8.427 \times 10^{-13} \end{pmatrix}
\]
\[
\lambda = \begin{pmatrix} 1.000 \end{pmatrix}
\]
\[
\text{info} = \{\text{iter}: 12,\ r_d: 3.140 \times 10^{-16},\ r_p: 4.039 \times 10^{-28},\ \mu: 8.427 \times 10^{-13}\}
\]

つまり,最適解は
\[x^* \approx \begin{pmatrix} 0 \\ 0 \end{pmatrix}\]
\[y \approx \begin{pmatrix} 0 \end{pmatrix}\]
\[\lambda \approx \begin{pmatrix} 1 \end{pmatrix}\]
であることがわかる。


% ===== 結論 =====
\section*{結論}


% ===== 付録 =====
\appendix
\section{ソースコード}
コード作成、レポート作成の一部にGitHub Copilotを使用した。
\subsection{課題15のコード}
\lstinputlisting[language=Python]{task15.py}
\subsection{課題16のコード}
\lstinputlisting[language=Python]{task16.py}
\subsection{課題17のコード}
\lstinputlisting[language=Python]{task17.py}
\subsection{課題18のコード}
\lstinputlisting[language=Python]{task18.py}
\subsection{課題19のコード}
\lstinputlisting[language=Python]{task19.py}
\subsection{追加課題のコード}
\lstinputlisting[language=Python]{addtask.py}




% ===== 参考文献 =====
\section*{参考文献}
\begin{thebibliography}{9}

\bibitem{exp2025}
数理工学実験(2025年度配布資料).

\bibitem{pardalos1991}
Pardalos, P.~M. and Vavasis, S.~A. (1991).
``Quadratic programming with one negative eigenvalue is NP-hard,''
\textit{Journal of Global Optimization}, 1(1), pp.~15--22.

\bibitem{sahni1974}
Sahni, S. (1974).
``Computationally related problems,''
\textit{SIAM Journal on Computing}, 3, pp.~262--279.

\end{thebibliography}

\end{document}
