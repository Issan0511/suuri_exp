\documentclass[a4paper,11pt]{ltjsarticle}
\usepackage{geometry}
\geometry{top=25mm,bottom=25mm,left=25mm,right=25mm}
\usepackage{graphicx}
\usepackage{amsmath,amssymb,bm}
\usepackage{hyperref}
\usepackage{float}
\usepackage{caption}
\usepackage{subcaption}
\usepackage{booktabs}
\usepackage{siunitx}
\usepackage{listings}
\usepackage{xcolor}
\usepackage{physics}
\usepackage{microtype}

\hypersetup{
  pdftitle={数理工学実験レポート},
  pdfauthor={中塚一瑳},
  colorlinks=true,
  linkcolor=blue,
  citecolor=blue,
  urlcolor=blue
}

% -------------------------
% ユーザ定義マクロ
% -------------------------
\newcommand{\StudentID}{1029366161}
\newcommand{\AuthorName}{中塚一瑳}
\newcommand{\CourseName}{数理工学実験}
\newcommand{\ChapterTitle}{第6章(連続最適化)}
\newcommand{\ExperimentDate}{\today}

% 図・表フォルダパス(必要なら変更)
\newcommand{\graphdir}{graphs}

% 数式用の簡易マクロ
\newcommand{\R}{\mathbb{R}}
% \newcommand{\norm}[1]{\left\lVert#1\right\rVert}

% listings 設定(Julia を例示)
\lstset{
  basicstyle=\ttfamily\footnotesize,
  breaklines=true,
  frame=single,
  numbers=left,
  numberstyle=\tiny,
  language=Python,
  keywordstyle=\color{blue},
  commentstyle=\color{gray},
  stringstyle=\color{teal},
  showstringspaces=false
}



% キャプションのフォント調整
\captionsetup{font=small,labelfont=bf}

% -------------------------
% 本文
% -------------------------
\title{数理工学実験レポート\\[4pt]\large \ChapterTitle}
\author{学籍番号 1029366161 \quad 中塚一瑳}
\date{\ExperimentDate}

\begin{document}

\maketitle

\tableofcontents
\clearpage



% ===== 例:実験テンプレート =====
\section*{はじめに}
今回は連続最適化の様々な手法を用いて、与えられた関数の最小値を求める課題に取り組む。


\section{課題15:ニュートン法および二分法による多項式の零点計算}


本課題では,多項式
\begin{align}
f(x) = x^{3}+2x^{2}-5x-6
\label{eq:f}
\end{align}
の零点を数値的に求める.
まず関数のグラフを描画して零点の存在を確認し,その後,二分法およびニュートン法を用いて零点を計算する.

\subsection{原理と方法}
\subsubsection{二分法}
二分法は,区間 $[a,b]$ において $f(a)$ と $f(b)$ の符号が異なるとき,その区間内に零点が存在することを利用した反復法である.
中点 $c=(a+b)/2$ を取り,$f(c)$ の符号に応じて零点を含む半区間に更新する操作を繰り返すことで,区間幅を徐々に縮小し零点へ収束させる.
零点を挟む区間が与えられれば必ず収束するが,収束速度は比較的遅い.

\subsubsection{ニュートン法}
ニュートン法は,関数をある点 $x_k$ の周りで一次近似し,その接線と $x$ 軸の交点を次の近似値とする方法である.
反復公式は
\begin{align}
x_{k+1}=x_k-\frac{f(x_k)}{f'(x_k)}
\end{align}
で与えられる.
一般に収束は速いが,初期値の選び方によっては収束しない場合がある.
本課題で用いた導関数は
\begin{align}
f'(x)=3x^{2}+4x-5
\end{align}
である.

\subsection{実験方法}
Python を用いて $x\in[-10,10]$ の範囲で $f(x)$ を描画し,グラフから零点が
$x\approx -3,-1,2$ 付近に存在することを確認した.
二分法では,それぞれの零点を挟む区間として
\[
[-4,-2],\ [-2,0],\ [1,3]
\]
を与えた.
ニュートン法では初期値として
\[
x_0=-2.5,\ -0.5,\ 1.5
\]
を用いた.
停止条件は $|f(x)|\le 10^{-10}$ とした.

\subsection{結果}
\subsubsection{関数のグラフ}
\begin{figure}[h]
  \centering
  \includegraphics[width=0.7\linewidth]{task15.png}
  \caption{$f(x)=x^{3}+2x^{2}-5x-6$ のグラフ}
  \label{fig:task15}
\end{figure}

\subsubsection{零点}
数値計算によって得られた零点を表\ref{tab:roots}に示す.
\begin{table}[h]
\centering
\caption{二分法およびニュートン法で求めた零点}
\begin{tabular}{l|c|c}
\hline
手法 & 近似解 $x$ & 残差 $f(x)$ \\
\hline
二分法 & $-3.000$ & $0$ \\
二分法 & $-1.000$ & $0$ \\
二分法 & $ 2.000$ & $0$ \\
\hline
ニュートン法 & $-3.000$ & $0$ \\
ニュートン法 & $-1.000$ & $0$ \\
ニュートン法 & $ 2.000$ & $1.421\times 10^{-14}$ \\
\hline
\end{tabular}
\label{tab:roots}
\end{table}

\subsection{考察}
$f(x)=0$ は因数分解により
\[
f(x)=(x+3)(x+1)(x-2)
\]
と書け,解析解は $x=-3,-1,2$ である.
数値計算によって得られた零点はこれらと一致しており,手法が正しく実装されていることが確認できる.
ニュートン法において残差が完全に $0$ とならない場合があるのは,浮動小数点演算誤差によるものである.



\section{課題16:}




% ===== 結論 =====
\section*{結論}


% ===== 付録 =====
\appendix
\section{ソースコード}
コード作成、レポート作成の一部にGitHub Copilotを使用した。
\subsection{課題15のコード}
\lstinputlisting[language=Python]{task15.py}


% ===== 参考文献 =====
\section*{参考文献}
\begin{thebibliography}{9}
\bibitem{exp2025} 数理工学実験(2025年度配布資料).
\end{thebibliography}

\end{document}
