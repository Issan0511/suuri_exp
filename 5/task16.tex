
\subsection{原理と方法}
本課題では,次の関数
\begin{align}
f(x) := \frac{1}{3}x^3 - x^2 - 3x + \frac{5}{3}
\end{align}
の停留点($f'(x)=0$ を満たす点)を,(a) 最急降下法,(b) ニュートン法により数値的に求める.
まず微分は
\begin{align}
f'(x) &= x^2 - 2x - 3 = (x-3)(x+1),\\
f''(x) &= 2x - 2
\end{align}
である.従って停留点は解析的には $x=-1,\,3$ の2点である.

\subsubsection{(a) 最急降下法}
1次元では勾配 $\nabla f(x)$ は $f'(x)$ に一致するため,最急降下法は
\begin{align}
x_{k+1} = x_k - t_k f'(x_k), \qquad t_k = \frac{1}{k+1}
\end{align}
で与えられる.初期点は $x_0=1/2$ とする.停止判定は $|f'(x_k)|\le \varepsilon$($\varepsilon=10^{-8}$)とし,最大反復回数も設ける.

\subsubsection{(b) ニュートン法}
停留点探索($f'(x)=0$ の零点探索)としてのニュートン法は
\begin{align}
x_{k+1} = x_k - t_k\frac{f'(x_k)}{f''(x_k)}, \qquad t_k=1
\end{align}
で与えられる.初期点は $x_0=5$ とし,(a) と同様に $|f'(x_k)|\le \varepsilon$ を停止判定とした.なお,本問題では $f''(x)=0$(すなわち $x=1$)で更新が不能となるため,実装では $f''(x_k)=0$ の場合を例外として扱う.

\subsection{実装}
Python により $f,f',f''$ をそれぞれ関数として実装し,(a) と (b) の更新式をそのまま反復した.
各反復で $(x_k, f(x_k), |f'(x_k)|)$ を履歴として保存し,収束挙動の可視化に用いた.

\subsection{結果}
(a) 最急降下法($x_0=0.5,\ t_k=1/(k+1)$)および (b) ニュートン法($x_0=5,\ t_k=1$)の結果を表\ref{tab:task16_result}に示す.
両手法とも停留点 $x^\ast=3$ に収束した(もう一つの停留点 $x=-1$ には到達しなかった).\\
\\



\begin{table}[h]
\centering
\caption{課題16の計算結果($\varepsilon=10^{-8}$)}
\label{tab:task16_result}
\begin{tabular}{lcccc}
\hline
手法 & 初期値 $x_0$ & 近似解 $x^\ast$ & $f(x^\ast)$ & 反復回数 $k$ \\
\hline
最急降下法 & 0.5000 & 3.000 & -7.333 & 193 \\
ニュートン法 & 5.000 & 3.000 & -7.333 & 5 \\
\hline
\end{tabular}
\end{table}

両方法による反復点の推移を,関数 $y=f(x)$ 上に重ねて図\ref{fig:task16_iter}に示す.

\begin{figure}[h]
 \centering
 \includegraphics[width=0.7\linewidth]{task16.png}
 \caption{関数 $y=f(x)$ 上における反復点の推移.
 青丸は最急降下法,橙四角はニュートン法による反復点を表す.}
 \label{fig:task16_iter}
\end{figure}

\subsection{考察}
図\ref{fig:task16_iter}より,最急降下法では反復点が谷に沿って
緩やかに移動しており,ステップサイズ $t_k=1/(k+1)$ が減少することで
後半の収束が遅くなっていることが視覚的に確認できる.
一方,ニュートン法では局所的な2次近似に基づき更新が行われるため,
停留点近傍では大きなジャンプを伴い,極めて高速に収束している.

