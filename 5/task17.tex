
本課題では,次の関数
\begin{align}
f(x_0,x_1)
= x_0^2 + e^{x_0} + x_1^4 + x_1^2 - 2x_0x_1 + 3
\end{align}
について,勾配およびヘッセ行列を解析的に求め,それを実装したコードとの対応を示す.

\subsection{微分結果}

まず,各変数による1階微分は
\begin{align}
\frac{\partial f}{\partial x_0}
&= 2x_0 + e^{x_0} - 2x_1, \\
\frac{\partial f}{\partial x_1}
&= 4x_1^3 + 2x_1 - 2x_0
\end{align}
である.従って勾配ベクトルは
\begin{align}
\nabla f(x)
=
\begin{pmatrix}
2x_0 + e^{x_0} - 2x_1 \\
4x_1^3 + 2x_1 - 2x_0
\end{pmatrix}
\end{align}
と表される\cite{exp2025}.

次に,2階微分よりヘッセ行列は
\begin{align}
\nabla^2 f(x)
=
\begin{pmatrix}
\displaystyle \frac{\partial^2 f}{\partial x_0^2}
&
\displaystyle \frac{\partial^2 f}{\partial x_0 \partial x_1}
\\[6pt]
\displaystyle \frac{\partial^2 f}{\partial x_1 \partial x_0}
&
\displaystyle \frac{\partial^2 f}{\partial x_1^2}
\end{pmatrix}
=
\begin{pmatrix}
2 + e^{x_0} & -2 \\
-2 & 12x_1^2 + 2
\end{pmatrix}
\end{align}
となる.

\subsection{コードとの対応}

上記の解析結果は,以下の Python コードとして実装されている.

\begin{itemize}
 \item 目的関数
\begin{lstlisting}[language=Python]
f = x0**2 + exp(x0) + x1**4 + x1**2 - 2*x0*x1 + 3
\end{lstlisting}

 \item 勾配
\begin{lstlisting}[language=Python]
g0 = 2*x0 + exp(x0) - 2*x1
g1 = 4*x1**3 + 2*x1 - 2*x0
g  = [g0, g1]
\end{lstlisting}

 \item ヘッセ行列
\begin{lstlisting}[language=Python]
h00 = 2 + exp(x0)
h01 = -2
h10 = -2
h11 = 12*x1**2 + 2
H = [[h00, h01],
     [h10, h11]]
\end{lstlisting}
\end{itemize}

これにより,関数値・勾配・ヘッセ行列が理論式と一致して計算されていることが確認できる.
