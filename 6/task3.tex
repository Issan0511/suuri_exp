% =========================
% 課題3(修正版):限定操作(下界値テスト)の導入
%  ※ 課題4で実装した「残り辺数に応じたLB(反復途中結果の保存)」に合わせて整合を取った版
% =========================

\subsection{限定操作の方針}
課題1で設計した全単純路探索(分枝)に対し,本課題では限定操作として \textbf{下界値テスト}を導入する。
すなわち,探索途中の部分問題(探索木の節点)に対して,下界が暫定最良値(incumbent)以上であるとき,
その部分問題の子孫探索を省略(終端)することで探索木を縮小する。\cite{exp2025}

\subsection{下界(LB)の定義:事前計算による弱い下界(辺数制限付き)}
単純路では同一節点を2回訪れないため,始点から終点までに使用できる辺数は高々 $|V|-1$ である。
この点を下界にも反映させるため,本課題では各節点 $v\in V$ に対し
\[
\mathrm{LB}_k[v] \quad (k=0,1,\dots,|V|-1)
\]
を次の意味で定義する:
\begin{quote}
\textbf{$\mathrm{LB}_k[v]$:} 「\textbf{単純路制約を無視}した緩和問題において,
$v$ から $t$ へ \textbf{高々 $k$ 本の辺}で到達する歩道(頂点重複を許す)を考えたときの最短距離」。
\end{quote}
単純路制約を無視することは制約緩和であるため,任意の単純路の残余コストは $\mathrm{LB}_k[v]$ 以上となり,
$\mathrm{LB}_k$ は元の問題に対する下界として安全である。

\subsubsection{(正規の Bellman--Ford 法との違い:負閉路検出の省略)}
本課題では枝の長さが負でもよいので,$\mathrm{LB}_k$ の計算は Bellman--Ford 法の反復緩和と同型の手続きで行える。
ただし \textbf{正規の Bellman--Ford 法が行う負閉路検出は本実装では省略する}。理由は以下の通りである。
\begin{itemize}
\item 元問題は単純路であり,使用可能な辺数は高々 $|V|-1$ である。本課題の下界も $\mathrm{LB}_k$ として
\textbf{辺数を $k$ で制限して定義}しているため,閉路を何度でも回って距離を無限に下げる状況($-\infty$)は,
この $k$ 制限の枠内では本質的に問題になりにくい。
\item 下界値テストで必要なのは「過小評価(楽観的)な下界」である。負閉路検出は「さらに反復すれば距離が下がり続け得る」
ことを示すが,本課題では $k$ を固定して $\mathrm{LB}_k$ を用いるため,負閉路検出の追加ステップを入れても
枝刈り効果の改善には直結しにくい(むしろ下界が過度に小さくなる方向に働き得る)。
\end{itemize}
以上より,本課題では \textbf{$k=0,1,\dots,|V|-1$ の範囲での緩和結果}を下界として利用し,負閉路検出は省略する。

\subsection{下界値テストの妥当性}
探索中,現在節点を $u$,始点 $s$ から $u$ までの累積路長を $L$ とする。
現在までに訪れた節点数が $|P|$ のとき,単純路制約の下で今後使用できる辺数は高々
\[
r := |V| - |P|
\]
である(これ以上進むと必ず頂点重複が生じるため)。
よって,$u$ から $t$ への任意の単純路の追加コストは,
制約を緩めた緩和問題の「高々 $r$ 本の辺での最短距離」$\mathrm{LB}_r[u]$ 以上である。
したがって,この部分問題から得られる任意の $s\to t$ 単純路の長さは
\[
L+\mathrm{LB}_r[u]
\]
以上である。ここで暫定最良値(上界)を $\texttt{bestValue}$ とすると,
\[
L+\mathrm{LB}_r[u]\ \ge\ \texttt{bestValue}
\]
が成り立つ場合,この部分問題の子孫をどれだけ探索しても $\texttt{bestValue}$ より良い解は得られない。
よってこの条件を満たすとき探索を終端してよい(下界値テスト)。
また $\mathrm{LB}_r[u]=+\infty$($r$ 本以内では $u$ から $t$ に到達不能)の場合も実行可能解が存在しないため,
同様に終端してよい。

\subsection{擬似コード}

\subsubsection{前処理:下界配列 $\mathrm{LB}_k$ の計算(負閉路検出は省略)}
以下は Bellman--Ford 法と同様に緩和を反復するが,
\textbf{各反復 $k$ の結果を保存}し,$\mathrm{LB}_k$ を得る手続きである。

\begin{quote}
\textbf{Algorithm: PrecomputeLBSteps}($G=(V,E)$, $d$, $t$)

\textbf{Input:} 有向グラフ $G=(V,E)$, 枝長 $d:E\to\mathbb{R}$, 終点 $t$\\
\textbf{Output:} 下界配列 $\mathrm{LB}_k[v]$($k=0,\dots,|V|-1$)
\begin{enumerate}
\item \textbf{for each} $v \in V$ \textbf{do} $\mathrm{LB}_0[v] := +\infty$
\item $\mathrm{LB}_0[t] := 0$
\item \textbf{for} $k := 1$ \textbf{to} $|V|-1$ \textbf{do}
\begin{enumerate}
\item[(a)] \textbf{for each} $v \in V$ \textbf{do} $\mathrm{LB}_k[v] := \mathrm{LB}_{k-1}[v]$
\item[(b)] \textbf{for each} $(u,v)\in E$ \textbf{do}
\begin{enumerate}
\item[(i)] \textbf{if} $\mathrm{LB}_{k-1}[v] \neq +\infty$ \textbf{and}
$\; d(u,v) + \mathrm{LB}_{k-1}[v] < \mathrm{LB}_k[u]$ \textbf{then}
\begin{enumerate}
\item $\mathrm{LB}_k[u] := d(u,v) + \mathrm{LB}_{k-1}[v]$
\end{enumerate}
\end{enumerate}
\end{enumerate}
\item // 注:正規の Bellman--Ford 法の負閉路検出(追加の緩和チェック)は省略する
\item \textbf{return} $\{\mathrm{LB}_k\}_{k=0}^{|V|-1}$
\end{enumerate}
\end{quote}

\subsubsection{限定操作:下界値テスト}
\begin{quote}
\textbf{Procedure: LowerBoundTest}($u$, $L$, $r$)

\textbf{Input:} 現在節点 $u$, 累積長 $L$, 残り許容辺数 $r$\\
\textbf{Output:} 枝刈りするなら \textbf{true},続行するなら \textbf{false}
\begin{enumerate}
\item \textbf{if} $\mathrm{LB}_r[u] = +\infty$ \textbf{then return true}
\item \textbf{if} $L + \mathrm{LB}_r[u] \ge \texttt{bestValue}$ \textbf{then return true}
\item \textbf{return false}
\end{enumerate}
\end{quote}

\subsubsection{全体:限定操作込み探索}
\begin{quote}
\textbf{Algorithm: ShortestSimplePathBB}($G=(V,E)$, $d$, $s$, $t$)

\textbf{Input:} $G=(V,E)$, $d:E\to\mathbb{R}$, 始点 $s$, 終点 $t$\\
\textbf{Output:} 最短単純路 \texttt{bestPath} と長さ \texttt{bestValue}
\begin{enumerate}
\item $\{\mathrm{LB}_k\} := \texttt{PrecomputeLBSteps}(G,d,t)$
\item $P := \emptyset$
\item $\texttt{bestValue} := +\infty$
\item $\texttt{bestPath} := \emptyset$
\item $\texttt{ExploreBB}(P, s, 0)$
\item \textbf{return} $(\texttt{bestPath}, \texttt{bestValue})$
\end{enumerate}
\end{quote}

\begin{quote}
\textbf{Procedure: ExploreBB}($P$, $u$, $L$)

\textbf{Input:} 単純路 $P$, 現在節点 $u \in V\setminus P$, 累積長 $L$
\begin{enumerate}
\item $P.\texttt{push}(u)$
\item $r := |V| - |P|$ \quad // 残り許容辺数
\item \textbf{if} $\texttt{LowerBoundTest}(u, L, r)$ \textbf{then}
\begin{enumerate}
\item[(a)] $P.\texttt{pop}()$
\item[(b)] \textbf{return}
\end{enumerate}
\item \textbf{if} $u = t$ \textbf{then}
\begin{enumerate}
\item[(a)] \textbf{if} $L < \texttt{bestValue}$ \textbf{then}
\begin{enumerate}
\item[(i)] $\texttt{bestValue} := L$
\item[(ii)] $\texttt{bestPath} := \texttt{Copy}(P)$
\end{enumerate}
\end{enumerate}
\item \textbf{else}
\begin{enumerate}
\item[(a)] \textbf{for each} $(u,v)\in E$, $v \notin P$ \textbf{do}
\begin{enumerate}
\item[(i)] $\texttt{ExploreBB}(P, v, L + d(u,v))$
\end{enumerate}
\end{enumerate}
\item $P.\texttt{pop}()$
\item \textbf{return}
\end{enumerate}
\end{quote}
