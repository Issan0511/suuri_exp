\subsection{実装}

課題3で設計した限定操作(下界値テストおよび到達不能判定)を Python で実装した。下界値 $\mathrm{LB}[u]$ は、$u$ から終点 $t$ までの「単純路制約を無視した最短距離」として Bellman--Ford 法により前処理で計算する。探索中は、現在節点 $u$ と累積路長 $L$ に対し、$L + \mathrm{LB}[u] \geq \texttt{bestValue}$ が成り立つ場合に部分問題を打ち切る(下界値テスト)。また、$\mathrm{LB}[u] = +\infty$ の場合($u$ から $t$ へ到達不能)も同様に打ち切る。

探索ノード数と枝刈り数は、DFS 関数に入った回数(探索ノード数)と、限定操作により打ち切った回数(枝刈り数)としてそれぞれカウンタを用いて計測した。実行時間は \texttt{time.process\_time()} により CPU 時間を測定し、ミリ秒に変換した。

グラフ生成方法および試行回数は課題2と同一とし、各 $(n,m)$ について 5 回試行して平均および標準偏差を算出した。枝数は $m = \frac{n(n-1)}{2}$、$n=5$ から $15$ まで計測した。

\subsection{結果}

限定操作なし(課題2)と限定操作あり(課題4)の実行時間と探索ノード数を表\ref{tab:task4}に示す。

\begin{table}[h]
\centering
\caption{限定操作の有無による実行時間・探索ノード数の比較(平均±標準偏差、試行回数5)}
\label{tab:task4}
\small
\begin{tabular}{r|rr|rr||rr|rr|rr}
\hline
\multirow{2}{*}{$n$} & \multicolumn{4}{c||}{限定なし (Plain)} & \multicolumn{6}{c}{限定あり (BB)} \\
\cline{2-11}
 & \multicolumn{2}{c|}{time (ms)} & \multicolumn{2}{c||}{nodes} & \multicolumn{2}{c|}{time (ms)} & \multicolumn{2}{c|}{nodes} & \multicolumn{2}{c}{pruned} \\
 & mean & std & mean & std & mean & std & mean & std & mean & std \\
\hline
5  & 0.016 & 0.024 & 4.4 & 2.1 & 0.024 & 0.007 & 3.2 & 2.0 & 0.8 & 0.4 \\
6  & 0.008 & 0.002 & 10.6 & 3.0 & 0.038 & 0.006 & 8.6 & 2.6 & 2.8 & 1.2 \\
7  & 0.039 & 0.029 & 69.8 & 33.3 & 0.058 & 0.009 & 26.6 & 11.3 & 11.6 & 5.5 \\
8  & 0.033 & 0.019 & 73.4 & 45.4 & 0.101 & 0.033 & 40.4 & 22.1 & 17.8 & 15.0 \\
9  & 0.122 & 0.083 & 240.8 & 160.3 & 0.177 & 0.037 & 109.0 & 50.4 & 41.8 & 30.2 \\
10 & 0.468 & 0.255 & 1283.0 & 609.1 & 0.306 & 0.112 & 336.8 & 166.0 & 140.8 & 46.1 \\
11 & 2.355 & 0.781 & 5725.0 & 1741.0 & 0.742 & 0.223 & 805.2 & 318.2 & 419.6 & 175.4 \\
12 & 14.980 & 5.121 & 26190.0 & 8652.0 & 1.598 & 0.826 & 1315.0 & 703.0 & 689.6 & 365.6 \\
13 & 86.410 & 43.880 & 192400.0 & 92210.0 & 10.450 & 9.409 & 14090.0 & 13080.0 & 7991.0 & 7887.0 \\
14 & 497.500 & 254.500 & 1093000.0 & 487600.0 & 14.380 & 4.662 & 19770.0 & 6362.0 & 11560.0 & 3256.0 \\
15 & 4163.000 & 1294.000 & 8474000.0 & 2836000.0 & 47.330 & 55.020 & 65050.0 & 75650.0 & 41510.0 & 47960.0 \\
\hline
\end{tabular}
\end{table}

限定操作の導入により、探索ノード数が大幅に削減され、それに伴い実行時間も劇的に短縮された。特に $n=15$ では、限定なしで約 4163ms、探索ノード約 $8.47 \times 10^6$ であったのに対し、限定ありでは約 47ms、探索ノード約 $6.51 \times 10^4$ となり、探索ノード数が約 130 分の 1、実行時間が約 88 分の 1 に削減された。

\subsection{考察}

\subsubsection{限定操作の効果}

下界値テストの導入により、暫定解を超える可能性のない部分問題を早期に打ち切ることができ、探索木の大部分を枝刈りできた。表\ref{tab:task4}の ``pruned'' 列は、限定操作により打ち切られた部分問題数を示しており、$n$ が大きくなるほど枝刈り数も増加している。例えば $n=15$ では約 41510 個の部分問題が枝刈りされ、全体の探索ノード数(約 65050)のうち約 64\% が枝刈りによって削減されたことになる。


\subsubsection{実行時間の削減}

探索ノード数の削減に伴い、実行時間も大幅に短縮された。$n=10$ までは両手法ともミリ秒単位で完了するが、$n \geq 11$ からは差が顕著になる。$n=13$ 以降では、限定なしでは秒単位の計算時間を要するのに対し、限定ありではミリ秒単位で完了している。これは、限定操作により実質的に解けるグラフのサイズが拡大したことを意味する。


\subsubsection{限定操作のオーバーヘッド}

$n$ が非常に小さい場合(例:$n=5, 6$)、限定ありの方が実行時間が長くなることがある。これは、下界値計算(Bellman--Ford 法)や下界値テストの判定に伴うオーバーヘッドが、枝刈りによる削減効果を上回るためである。しかし、$n \geq 7$ では枝刈り効果がオーバーヘッドを上回り、限定操作の導入が明確に有効となる。



