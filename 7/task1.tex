\subsection{問題設定}
1次元拡散方程式
\begin{equation}
  \frac{\partial u}{\partial t}(x,t) = \frac{\partial^2 u}{\partial x^2}(x,t)
\end{equation}
を、区間 $[0,L]$($L=10$)上で数値的に解く。ここで$u(x,t)$は位置$x$・時刻$t$における濃度を表す。初期条件は
\begin{equation}
  u_0(x) = \frac{1}{\sqrt{2\pi}} \exp\left( -\frac{1}{2}(x-5)^2 \right)
\end{equation}
とし、Dirichlet境界条件およびNeumann境界条件の両方について計算を行った。

\subsection{原理と方法}

\subsubsection{離散化と無次元パラメータ$c$の導出}
空間区間$[0,L]$を幅$\Delta x$で分割し、$L=N\Delta x$とする。時間も刻み幅$\Delta t$で離散化する。
区間$[(j-1)\Delta x, j\Delta x]$における平均値として
\begin{equation}
  u_j^n = \frac{1}{\Delta x} \int_{(j-1)\Delta x}^{j\Delta x} u(y, n\Delta t) \, dy
\end{equation}
を定義する。\cite{exp2025}

拡散方程式を時間区間$[n\Delta t, (n+1)\Delta t]$で積分すると、
\begin{equation}
  u(x,(n+1)\Delta t) - u(x,n\Delta t) = \int_{n\Delta t}^{(n+1)\Delta t} \frac{\partial^2 u}{\partial x^2}(x,s) \, ds
\end{equation}
を得る。右辺を時刻$n\Delta t$の値で近似すると、
\begin{equation}
  u(x,(n+1)\Delta t) - u(x,n\Delta t) \simeq \Delta t \frac{\partial^2 u}{\partial x^2}(x,n\Delta t)
\end{equation}
となる。さらに空間微分を中心差分で近似すると、
\begin{equation}
  \frac{\partial^2 u}{\partial x^2}(x,n\Delta t) \simeq \frac{u_{j-1}^n - 2u_j^n + u_{j+1}^n}{\Delta x^2}
\end{equation}
である。これらをまとめると、
\begin{equation}
  u_j^{n+1} - u_j^n = \frac{\Delta t}{\Delta x^2} \left( u_{j-1}^n - 2u_j^n + u_{j+1}^n \right)
\end{equation}
を得る。ここで
\begin{equation}
  \boxed{c \equiv \frac{\Delta t}{\Delta x^2}}
\end{equation}
と定義する。この$c$は時間刻みと空間刻みの比を表し、数値安定性を支配する重要な無次元量である。

\subsubsection{オイラー陽解法}
差分方程式をそのまま用いると、
\begin{equation}
  u_j^{n+1} = u_j^n + c\left( u_{j-1}^n - 2u_j^n + u_{j+1}^n \right)
\end{equation}
となる。これは既知の時刻$n$の値から直接$n+1$を計算できるため、オイラー陽解法と呼ばれる。\cite{exp2025}

\paragraph{特徴}
\begin{itemize}
  \item 実装が非常に簡単
  \item ただし安定性条件$c \le 1/2$を満たさないと数値解が発散する
  \item $\Delta x$を小さくすると$\Delta t$も強く制限される
\end{itemize}

\subsubsection{クランク–ニコルソン法}
時間積分を台形公式で近似すると、
\begin{equation}
  u_j^{n+1} - u_j^n = \frac{c}{2} \left[ (u_{j-1}^{n+1} - 2u_j^{n+1} + u_{j+1}^{n+1}) + (u_{j-1}^{n} - 2u_j^{n} + u_{j+1}^{n}) \right]
\end{equation}
を得る。整理すると、
\begin{equation}
  -\frac{c}{2}u_{j-1}^{n+1} + (1+c)u_j^{n+1} - \frac{c}{2}u_{j+1}^{n+1} = (1-c)u_j^n + \frac{c}{2}(u_{j-1}^n + u_{j+1}^n)
  \label{eq:cn}
\end{equation}
となる。未知数$u_j^{n+1}$が連立方程式として現れるため、陰解法に分類される。\cite{exp2025}

\paragraph{行列表示と三重対角構造}
内部格子点$j=1,\dots,N$に対して、未知ベクトル$\mathbf{x} = (u_1^{n+1}, u_2^{n+1}, \dots, u_N^{n+1})^T$を定義すると、式\eqref{eq:cn}は
\begin{equation}
  A\mathbf{x} = \mathbf{z}
\end{equation}
と書ける。係数行列$A$は三重対角行列となる。

\paragraph{LU分解による解法}
三重対角行列$A$を
\begin{equation}
  A = LU
\end{equation}
と下三角行列$L$と上三角行列$U$の積に分解する。$L$と$U$の係数は漸化式により前処理として一度だけ計算できる。

連立方程式$A\mathbf{x} = \mathbf{z}$は、$LU\mathbf{x} = \mathbf{z}$とおき、
\begin{enumerate}
  \item 前進代入:$L\mathbf{y} = \mathbf{z}$を解く
  \item 後退代入:$U\mathbf{x} = \mathbf{y}$を解く
\end{enumerate}
により効率的に求まる。各時間ステップの計算量は$O(N)$であり、係数行列$A$は時間に依存しないため、LU分解は最初に1回だけ行えばよい。\cite{exp2025}

\paragraph{特徴}
\begin{itemize}
  \item 全ての$c > 0$に対して安定(無条件安定)
  \item オイラー法より高精度
  \item LU分解により効率的に解ける
\end{itemize}

\subsubsection{境界条件の扱い}

\paragraph{Dirichlet境界条件}
境界での値を直接指定する:\cite{exp2025}
\begin{equation}
  u_0^n = u_L, \quad u_{N+1}^n = u_R
\end{equation}
物理的には「境界の温度(濃度)が固定されている」状況を表す。本課題では$u_L = u_R = 0$とした。

\paragraph{Neumann境界条件}
境界での微分(流束)を指定する:\cite{exp2025}
\begin{equation}
  \frac{\partial u}{\partial x}(0,t) = J_L, \quad \frac{\partial u}{\partial x}(L,t) = J_R
\end{equation}
これを差分化すると、
\begin{equation}
  u_0^n = u_1^n - J_L \Delta x, \quad u_{N+1}^n = u_N^n + J_R \Delta x
\end{equation}
となる。本課題では$J_L = J_R = 0$としており、境界での流束がゼロ、すなわち外へ物質が流出しない条件を表す。

\subsection{計算結果}
計算パラメータは $N=100$, $\Delta x = 0.1$, $\Delta t = 0.004$($c=0.4$)とした。

\subsubsection{Dirichlet境界条件}
図\ref{fig:dirichlet}にDirichlet境界条件における計算結果を示す。オイラー陽解法とクランク-ニコルソン法の結果はほぼ一致しており、初期のガウス分布が時間とともに拡散し、境界条件により最終的に消滅していくと予想される。

\begin{figure}[H]
  \centering
  \begin{subfigure}{0.45\textwidth}
    \includegraphics[width=\textwidth]{graphs_dirichlet/comparison_t1.png}
    \caption{$t=1$}
  \end{subfigure}
  \begin{subfigure}{0.45\textwidth}
    \includegraphics[width=\textwidth]{graphs_dirichlet/comparison_t2.png}
    \caption{$t=2$}
  \end{subfigure}
  \begin{subfigure}{0.45\textwidth}
    \includegraphics[width=\textwidth]{graphs_dirichlet/comparison_t3.png}
    \caption{$t=3$}
  \end{subfigure}
  \begin{subfigure}{0.45\textwidth}
    \includegraphics[width=\textwidth]{graphs_dirichlet/comparison_t4.png}
    \caption{$t=4$}
  \end{subfigure}
  \begin{subfigure}{0.45\textwidth}
    \includegraphics[width=\textwidth]{graphs_dirichlet/comparison_t5.png}
    \caption{$t=5$}
  \end{subfigure}
  \caption{Dirichlet境界条件における拡散方程式の数値解}
  \label{fig:dirichlet}
\end{figure}

\subsubsection{Neumann境界条件}
図\ref{fig:neumann}にNeumann境界条件における計算結果を示す。Neumann境界条件では流束がゼロであるため、初期のガウス分布が境界で反射され、最終的には一様分布に近づいていくと予想される。

\begin{figure}[H]
  \centering
  \begin{subfigure}{0.45\textwidth}
    \includegraphics[width=\textwidth]{graphs_neumann/comparison_t1.png}
    \caption{$t=1$}
  \end{subfigure}
  \begin{subfigure}{0.45\textwidth}
    \includegraphics[width=\textwidth]{graphs_neumann/comparison_t2.png}
    \caption{$t=2$}
  \end{subfigure}
  \begin{subfigure}{0.45\textwidth}
    \includegraphics[width=\textwidth]{graphs_neumann/comparison_t3.png}
    \caption{$t=3$}
  \end{subfigure}
  \begin{subfigure}{0.45\textwidth}
    \includegraphics[width=\textwidth]{graphs_neumann/comparison_t4.png}
    \caption{$t=4$}
  \end{subfigure}
  \begin{subfigure}{0.45\textwidth}
    \includegraphics[width=\textwidth]{graphs_neumann/comparison_t5.png}
    \caption{$t=5$}
  \end{subfigure}
  \caption{Neumann境界条件における拡散方程式の数値解}
  \label{fig:neumann}
\end{figure}

\subsection{考察}
オイラー陽解法とクランク-ニコルソン法の結果を比較すると、同じパラメータでほぼ同一の解が得られた。これは、使用した $c=0.4 < 0.5$ がオイラー陽解法の安定条件を満たしており、十分小さな時間刻みを使用していることによる。クランク-ニコルソン法は無条件安定であるため、より大きな時間刻みを使用できるという利点がある。

境界条件の違いについては、Dirichlet境界条件では境界で$u=0$が課されるため、物質が境界から流出し最終的に消滅する。一方、Neumann境界条件では境界での流束がゼロであるため、物質は保存され、最終的に一様分布に近づく。
