\subsection{問題設定}
1次元シュレディンガー方程式
\begin{equation}
  i\hbar\frac{\partial \psi}{\partial t} = -\frac{\hbar^2}{2m}\frac{\partial^2 \psi}{\partial x^2} + V(x)\psi
\end{equation}
を、調和振動子ポテンシャル $V(x) = \frac{1}{2}kx^2$ のもとで解く。$\hbar = m = k = 1$ とし、$\psi = R + iI$($R$, $I$は実数)と分解して計算を行った。

初期条件は
\begin{equation}
  R_j^0 = \frac{\sqrt{2}}{\pi^{1/4}} \exp(-2(x_j - 5)^2), \quad I_j^0 = 0
\end{equation}
とした。

\subsection{数値解法}
実部 $R$ と虚部 $I$ に分離した差分スキーム:
\begin{align}
  R_j^{n+1} &= R_j^n + \Delta t \left( -\frac{1}{2} \frac{I_{j-1}^n - 2I_j^n + I_{j+1}^n}{(\Delta x)^2} + \frac{1}{2}x_j^2 I_j^n \right) \\
  I_j^{n+1} &= I_j^n - \Delta t \left( -\frac{1}{2} \frac{R_{j-1}^{n+1} - 2R_j^{n+1} + R_{j+1}^{n+1}}{(\Delta x)^2} + \frac{1}{2}x_j^2 R_j^{n+1} \right)
\end{align}
周期境界条件を適用した。

\subsection{計算結果}
計算パラメータは $N=400$, $\Delta x = 0.05$, $\Delta t = 0.001$ とした。

図\ref{fig:schrodinger}に確率密度 $P = R^2 + I^2$ の時間発展を示す。調和振動子ポテンシャル中で波束が振動する様子が観察できる。振動周期は $T = 2\pi$($\omega = 1$)であり、$t = 0, \pi, 2\pi$ では波束が元の位置に戻っている。

\begin{figure}[H]
  \centering
  \begin{subfigure}{0.45\textwidth}
    \includegraphics[width=\textwidth]{graphs_schrodinger/probability_density.png}
    \caption{確率密度の時間発展($t=1,2,\ldots,8$)}
  \end{subfigure}
  \begin{subfigure}{0.45\textwidth}
    \includegraphics[width=\textwidth]{graphs_schrodinger/probability_density_pi.png}
    \caption{確率密度の時間発展($t=0,\pi/2,\pi,\ldots$)}
  \end{subfigure}
  \begin{subfigure}{0.45\textwidth}
    \includegraphics[width=\textwidth]{graphs_schrodinger/P_max_evolution.png}
    \caption{確率密度最大値の時間発展}
  \end{subfigure}
  \begin{subfigure}{0.45\textwidth}
    \includegraphics[width=\textwidth]{graphs_schrodinger/probability_density_contour.png}
    \caption{確率密度の等高線図}
  \end{subfigure}
  \caption{シュレディンガー方程式の数値解}
  \label{fig:schrodinger}
\end{figure}

\subsection{考察}
調和振動子ポテンシャル中の波束は、古典的な調和振動子と同様に振動運動を行う。初期位置 $x_0 = 5$ から出発した波束は、$t = \pi/2$ で原点を通過し、$t = \pi$ で $x = -5$ に達する。その後、$t = 3\pi/2$ で再び原点を通過し、$t = 2\pi$ で元の位置に戻る。

確率密度の総和(規格化)は計算を通じてほぼ1に保たれており、使用した差分スキームが確率保存の性質を持つことが確認できた。確率密度の最大値は振動中に変化するが、これは波束の拡がりが振動中に変化するためである。
